%!TEX root = thesis.tex

% Set german to default language and load english as well
\usepackage[english,ngerman]{babel}

% Set UTF8 as input encoding
\usepackage[utf8]{inputenc}

% Set T1 as font encoding
\usepackage[T1]{fontenc}
% Load a slightly more modern font
\usepackage{lmodern}
% Use the symbol collection textcomp, which is needed by listings.
\usepackage{textcomp}
% Load a better font for monospace.
\usepackage{courier}
% Load package to configure header and footer
\usepackage{scrlayer-scrpage}
\usepackage{longtable}
\usepackage{acronym}
\usepackage{setspace}
\setstretch{1.2}


% Set some options regarding the document layout. See KOMA guide
\KOMAoptions{%
  paper=a4,
  fontsize=11.5pt,
  parskip=half,
  headings=normal,
  BCOR=1cm,
  headsepline,
  headsepline=0.5pt,
  DIV=13}

% do not align bottom of pages
\raggedbottom

% set style of captions
\setcapindent{24pt} % do not indent second line of captions
\setkomafont{caption}{\small}
\setkomafont{caption}{\vspace*{3mm}}
\setkomafont{captionlabel}{\textbf}
\setcapwidth[c]{1\textwidth}

% load extended tabulars used in the list of abbreviation
\usepackage{tabularx}

% load the color package and enable colored tables
\usepackage[table]{xcolor}

\usepackage{pdfpages}


% define new environment for zebra tables
\usepackage{colortbl}
\newcommand{\mainrowcolors}{\rowcolors{1}{white}{white}}
\newenvironment{zebratabular}{\mainrowcolors\begin{tabular}}{\end{tabular}}
\newcommand{\setrownumber}[1]{\global\rownum#1\relax}
%\newcommand{\headerrow}{\normalfont\sffamily\color{maincolor}\setrownumber1}
%\rowcolor{maincolor!50}
\renewcommand{\arraystretch}{1.3}

% add main color to section headers
\usepackage[outermarks,noindentafter,nobottomtitles]{titlesec}
\titleformat{\section}[hang]{\Large\sffamily\color{maincolor}}{\thetitle}{8pt}{\vspace{-2mm}}{}
\titleformat{\subsection}[hang]{\large\sffamily\color{maincolor}}{\thetitle}{8pt}{}
\titleformat{\subsubsection}[hang]{\normalfont\sffamily\color{maincolor}}{\thetitle}{8pt}{}
\titleformat{\paragraph}[hang]{\bfseries\sffamily\color{maincolor}}{\thetitle}{8pt}{}

%Etwas aufwendiger für die Kapitelüberschriften:
\titleformat{\chapter}[display]
{\huge\sffamily\color{maincolor}} 
{\color{maincolor!30}\fontsize{60pt}{40pt}\selectfont\thechapter}
{-0.2cm} %is vertical space in [display] mode
{\vspace{-2mm}} %1ex ist die Hähe von x im aktuellen Font
[\vspace{-4mm}]

% do not print numbers next to each formula
\usepackage{mathtools}
\mathtoolsset{showonlyrefs}
% left align formulas
\makeatletter
\@fleqntrue\let\mathindent\@mathmargin \@mathmargin=\leftmargini
\makeatother

% Allow page breaks in align environments
\allowdisplaybreaks

% header and footer
\pagestyle{scrheadings}
\setkomafont{pagenumber}{\normalfont\sffamily\color{maincolor}}
\setkomafont{pageheadfoot}{\normalfont\sffamily}
\setkomafont{headsepline}{\color{maincolor}}

% German guillemets quotes
\usepackage[german=guillemets]{csquotes}

% load TikZ to draw diagrams
\usepackage{tikz}

% load additional libraries for TikZ
\usetikzlibrary{%
  automata,%
  positioning,%
}

% set some default options for TikZ -- in this case for automata
\tikzset{
  every state/.style={
      draw=maincolor,
      thick,
      fill=maincolor!18,
      minimum size=0pt
    }
}

% load listings package to typeset sourcecode
\usepackage{listings}

% set some options for the listings package
\lstset{%
  upquote=true,%
  showstringspaces=false,%
  captionpos=b,%
  basicstyle=\ttfamily,%
  keywordstyle=\color{keywordcolor}\slshape,%
  commentstyle=\color{commentcolor}\itshape,%
  stringstyle=\color{stringcolor}}
\renewcommand{\lstlistingname}{Quelltext}
\renewcommand{\lstlistlistingname}{Quelltextverzeichnis}

% enable german umlauts in listings
\lstset{
literate={ö}{{\"o}}1
{Ö}{{\"O}}1
{ä}{{\"a}}1
{Ä}{{\"A}}1
{ü}{{\"u}}1
{Ü}{{\"U}}1
{ß}{{\ss}}1
}

% define style for pseudo code
\lstdefinestyle{pseudo}{language={},%
  basicstyle=\normalfont,%
  morecomment=[l]{//},%
  morekeywords={for,to,while,do,if,then,else},%
  mathescape=true,%
  columns=fullflexible}

% load the AMS math library to typeset formulas
\usepackage{amsmath}
\usepackage{amsthm}
\usepackage{thmtools}
\usepackage{amssymb}

% load the paralist library to use compactitem and compactenum environment
\usepackage{paralist}

% load varioref and hyperref to create nicer references using vref
\usepackage[ngerman]{varioref}
\PassOptionsToPackage{hyphens}{url} % allow line break at hyphens in URLs

\usepackage{varioref}
\usepackage{hyperref}
\newcommand{\ssecref}[1]{\hyperref[{#1}]{\autoref{#1}: „\nameref{#1}“}}
\newcommand*{\secref}[1]{\hyperref[{#1}]{\ref*{#1} \nameref*{#1}}}


% setup hyperref
\hypersetup{breaklinks=true,
  pdfborder={0 0 0},
  ngerman,
  pdfhighlight={/N},
  pdfdisplaydoctitle=true}

% Fix bugs in some package, e.g. listings and hyperref
\usepackage{scrhack}

% set the style of the bibliography
\usepackage{apacite}
\bibliographystyle{apacite}

% Allow todos
\usepackage{todonotes}

% define german names for referenced elements
% (vref automatically inserts these names in front of the references)
\labelformat{figure}{Abbildung\ #1}
\labelformat{table}{Tabelle\ #1}
\labelformat{appendix}{Anhang\ #1}
\labelformat{chapter}{Kapitel\ #1}
\labelformat{section}{Abschnitt\ #1}
\labelformat{subsection}{Unterabschnitt\ #1}
\labelformat{subsubsection}{Unterunterabschnitt\ #1}
\AtBeginDocument{\labelformat{lstlisting}{Quelltext\ #1}}

% define theorem environments
\declaretheorem[numberwithin=chapter,style=plain]{Theorem}
\labelformat{Theorem}{Theorem\ #1}

\declaretheorem[sibling=Theorem,style=plain]{Lemma}
\labelformat{Lemma}{Lemma\ #1}

\declaretheorem[sibling=Theorem,style=plain]{Korollar}
\labelformat{Korollar}{Korollar\ #1}

\declaretheorem[sibling=Theorem,style=definition]{Definition}
\labelformat{Definition}{Definition\ #1}

\declaretheorem[sibling=Theorem,style=definition]{Beispiel}
\labelformat{Beispiel}{Beispiel\ #1}

\declaretheorem[sibling=Theorem,style=definition]{Bemerkung}
\labelformat{Bemerkung}{Bemerkung\ #1}


\setcounter{secnumdepth}{2}
\setcounter{tocdepth}{1}