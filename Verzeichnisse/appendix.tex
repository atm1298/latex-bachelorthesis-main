%!TEX root = thesis.tex

\chapter{Anhang}


\section{Analyse: Interviewleitfaden}
\label{appendix:interview}

Dieser Anhang enthält tiefergehende Informationen, die nicht zur eigentlichen Arbeit gehören.

\begin{itemize}
    \item Begrüßung und Danken für die Zeit
    \item Kurzer Umriss des Themas: Ziel in den Interviews ist es, zu erfahren, wie die Organisation
          und Planung unter den Mitarbeiter:innen untereinander sowie zwischen den Mitarbeiter:innen und
          Studierenden aktuell abläuft und wie diese Kommunikation verbessert werden kann, in Bezug auf
          das Ausleihen von Assets + in 2 Teile geteilt
    \item Tätigkeit
    \item Datenschutz
    \item Wenn einverstanden sind, Interview aufzeichnen und würden Sie bitten, dazu das
          Datenschutzformular auszufüllen. Die Aufzeichnungen werden diskret behandelt und nach der
          Auswertung gelöscht
\end{itemize}

\subsection{Verleihende}
{\sffamily\color{maincolor}{Abschnitt: Jetzt}}
\begin{enumerate}
    \item Interessieren, wie Sie vorgehen, wenn Sie eine Anfrage erhalten zum Ausleihen eines
          Assets.
          \begin{enumerate}
              \item Bekommen Sie anfragen per Mail
              \item oder mündlich
              \item Kalender Eintragen (Outlook)/Zettel
              \item Einfach nehmen
          \end{enumerate}
    \item Gibt es eine öffentliche Übersicht (für Studierende) der auszuleihenden Geräte? Nein: Hat
          das schon einmal für Probleme gesorgt im Ausleihprozess? Im Nachhinein gehört, dass Assets
          benötigt wurden
    \item Können Personen vorläufig Systeme reservieren, z. B. in 2 Wochen für 4 Tage?
          Warum nicht? Probleme
    \item Wird sichergestellt, dass Ausleihende mit dem Gerät umgehen können?
    \item Wie sind die Assets versichert bzw. was passiert, wenn es kaputtgeht?
    \item Sehen Sie Probleme oder sind Sie mit dem derzeitigen Ablauf zufrieden?
    \item Fehlgeleitet Anfragen, direkte Ansprechpartner:innen
\end{enumerate}

{\sffamily\color{maincolor}{Abschnitt: Visionen und Ziele}}
\begin{enumerate}
    \item[8.] Vorstellen, es gibt eine webbasierte Anwendung (System)- Online-Plattform
        \begin{enumerate}
            \item Übersicht – Vorstellung?
                  Informationen werden benötigt
                  Form der Darstellung
            \item Was wären weitere Funktionen?
            \item Vorausplanen? Reservieren?
            \item Fehlgeleitet Anfragen, direkte Ansprechpartner:innen
        \end{enumerate}
    \item[9.] Sind Ihnen bis hierher noch Gedanken gekommen, die Sie gerne mit auf den Weg geben wollen?
    \item[10.] Vielen Dank, dass Sie uns für dieses Interview zur Verfügung gestanden haben, wir wären jetzt am Ende des Interviews angelangt
    \item[11.] Im Nachhinein noch melden können
\end{enumerate}

\subsection{Ausleihende}
\begin{itemize}
    \item Begrüßung und Danken für die Zeit
    \item Kurzer Umriss des Themas
    \item Vorerfahrung (HiWi,...)
    \item Datenschutz
\end{itemize}

{\sffamily\color{maincolor}{Abschnitt: Jetzt}}
\begin{enumerate}
    \item Ist Dir/Ihnen bekannt, welche Assets sie/du am IMIS ausleihen können?
          \begin{enumerate}
              \item Nein, was hätten sie/du dann gebraucht? Jetzt, wo du es weißt, würdest du es gerne
                    nutzen? oder Listenübersicht
              \item Vorgehen, wenn Sie eine Anfrage stellen zum Ausleihen eines Assets.
                    \begin{itemize}
                        \item Schauen Sie auch spontan in den Laboren, nach den Geräten vorbei?
                        \item Wie häufig ist Ihr spontaner Besuch (nicht) erfolgreich?
                        \item Was machen Sie, wenn Sie keine Person antreffen, der sie mitteilen, dass sie das Gerät mitnehmen?
                        \item Was wäre für Sie der einfachste Weg, Informationen zu hinterlassen, was würden Sie sich wünschen?
                    \end{itemize}
              \item Sehen Sie hier Probleme oder sind Sie mit dem derzeitigen Ablauf zufrieden?
                    \begin{itemize}
                        \item Darstellung von auszuleihenden Inhalten hilfreich (Übersicht)
                        \item Wie gehen sie vor, wenn sie etwas in 2 Wochen ausleihen wollen?
                    \end{itemize}
          \end{enumerate}
\end{enumerate}

{\sffamily\color{maincolor}{Abschnitt: Visionen und Ziele}}
\begin{enumerate}
    \item[8.] Vorstellen, es gibt eine webbasierte Anwendung (System)- Online-Plattform
        \begin{enumerate}
            \item Übersicht – Vorstellung?
                  Informationen werden benötigt
                  Form der Darstellung
            \item Was wären weitere Funktionen?
            \item Vorausplanen? Reservieren?
            \item Fehlgeleitet Anfragen, direkte Ansprechpartner:innen
        \end{enumerate}
    \item[9.] Sind Ihnen bis hierher noch Gedanken gekommen, die Sie gerne mit auf den Weg geben wollen?
    \item[10.] Vielen Dank, dass Sie uns für dieses Interview zur Verfügung gestanden haben, wir wären jetzt am Ende des Interviews angelangt
    \item[11.] Im Nachhinein noch melden können
\end{enumerate}