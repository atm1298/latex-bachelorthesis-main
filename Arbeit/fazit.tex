%!TEX root = thesis.tex

\chapter{Zusammenfassung und Ausblick}
\label{chapter-fazit}

Abschließen werden die wischtigsten Ergebnisse sowie Antworten auf die Forschungsfragen
(\ref{sec:Forschungsfragen}) der vorliegenden Arbeit zusammengefasst. Im Anschluss werden offene
Punkte aus dem Konzept, welche nicht wie geplant umgesetzt werden konnten, erläutert. Zum Abschluss
auf die möglichen Weiterentwicklungen des Frameworks eingegangen.


\section{Zusammenfassung}
Für einen Ausleih- und Reservierungsprozess ist die Unterstützung durch ein Reservierungstool eine
zielführende Möglichkeit, um die Planung in die Zukunft zu erleichtern und ermöglichen. Hierbei ist
insbesondere eine strukturierte und übersichtliche Ansicht der auszuleihenden Assets von hoher
Bedeutung.

Um ein solches Konzept entwickeln zu können wuden zunächst aus vorangestellten Untersuchungen die
objektiven Anforderungen an den Snipe-IT Companion formalisiert (\ref{section:anforderung}). Die
wesentliche Informationsquelle stellten hierbei die Stakeholder-Interviews mit Mitarbeitenden des
\ac{imis} und Studierenden der Medieninformatik. Zusätzlich wurde eine Recherche zu vergleichbaren
Systemen vorgenommen, welche Aufschluss über mögliche Fehlerquellen geben sollte. Hierbei wurde sich
nach einer weiteren Recherche auf zwei aus dem Interview genannte Apps beschränkt, da die vorherige
Recherche wenig Aufschluss für den spezifischen Einsatzfall gab. Aus den Qullen konnten zwei
Nutzendengruppen (Verleihende und Ausleihende) festgehalten (\ref{section:Nutzenden}). Für
Forschungsfrage F1 wurden die Probleme und Herausforderungen des aktuellen Vorgehens und den
unterschiedlichen Ausleihprozessen, in einer Problemanalyse, erarbeitet und beantwortet
(\ref{section:iststand}). Anschließend wurden für die Vorbereitung auf Forschungsfrage F2 die
Aufgaben, die Verleihende und Ausleihende im Reservierungsprozess erledigen müssen diskutiert
(\ref{section:aufgaben}). 

In der Spezifikationsphase wurden die Anforderungen an das System weiter spezialisiert
(\ref{chapter-konzept}). Dazu wurden zunächst gemäß der Forschungsfrage F2 Funktionalitäten
definiert. Die Funktionalitäten wurden entsprechend den Anforderungen entwickelt und in einer
priorisierten Feature-Liste festgehalten. Anschließend wurde die Systemarchitektur, aufgeteilt in
Frontend, Reservierungsinterface (Backend) und dem bestehenden Snipe-IT Server, mithilfe des
C4-Models dargestellt (\ref{section:architektur}). Aufbauend darauf wurden passende Frameworks zur
Entwicklung ausgewählt. 

Aufbauend auf dem Konzept wurde das Interface-Design erarbeitet (\ref{chapter-design}). Durch das
regelmäßige einarbeiten von Interviews konnte ein iteratives Vorgehen ermöglicht werden. Daraus
resultierten zentrale Designentscheidungen wie Begrifflichkeiten oder die Navigation, welche in
Folgearbeiten berücksichtigt werden sollten. 

Mithilfe der vorangehstellten Phasen wurde das Reservierungstools entsprechend umgesetzt
(\ref{chapter-implementierung}). Hierbei wurden die in der Konzeptionsphase festgelegten Frameworks
genutzt. \ref{chapter-dialogbeispiel} präsentiert das realisierte System anhand von
Dialogbeispielen.

In der abschließenden Phase wurde das realisierte System mithilfe von Interviews und Umfragen
evaluiert (\ref{chapter-evaluation}). Die Ergebnisse der Phase dienten zur Beantwortung von
Forschungsfrage F3, inwieweit ein aus F2 resultierender Prototyp die in F1 identifizierten
Schwierigkeiten reduzieren kann und gaben Aufschluss über die Wirksamkeit des entwickelten Systems.
Generell wurde die Oberfläche als übersichtlich beschrieben und der Wunsch, die Anwendung im
Universitätsalltag für Projekte nutzen zu können geäußert. Die eingangs formulierte Forschungsfrage
F3 konnte lediglich bedingt beantwortet werden, da das positive Ergebnis der durchgeführten Studie
nicht verglichen werden konnte. Eine weitere ausführliche Studie mit einem Vergleichsystem wird
somit empfohlen. Trotz des fehlenden Vergleichs konnte die Arbeit ein qualitativ hochwertiges System
hervorbringen, welches sich als übersichtlich und unterstützend herausstellt. Für Forschungsfrage 2
konnten weitere Anforderungen an das System herausgearbeitet werden. 

\section{Offene Punkte}
\label{sec:punkte}
Die in den Anforderungen festgelegten Funktionalitäten mit einer hohen Prioriät konnten umgesetzt
werden. Die mittle prioriseirten Funktionalitäten stellten sich als Herausforderung dar.
Insbesondere \anfref{F90}, die Einbindung des LDAP-Systems führte zu Schwierigkeiten und wurde
dementsprechend nicht umgesetzt. Zentrale Gründe hierfür waren der \textit{persönlichen
Zugriffstoken}, welcher die einzige Authentifizierungsmöglichkeit der Snipe-IT API ist und lediglich
manuell im Dashboard generiert werden können. Demzufolge konnte das LDAP-System nicht ohne Umstände
eingebunden und entsprechend genutzt werden. Dis führte auch dazu, dass eine Unterscheidung der
Zugriffsrechte zunächst nicht sichergestellt werden konnten (\anfref{Q20}). Durch die eingeschränkte
Beispieldatenbank konnte eine Nutzen-Suche (\anfref{F100}) nicht ermöglicht werden. Aus der
Evaluation ließ sich jedoch schließen, dass diese Funktion erwünscht sei.

Die erarbeite Funktionalität Ft-VA-7 umfasst das Filtern von Materialen. Das heißt unter anderem,
dass Nutzende verschiebene Preferezen, wie Sortierung, Status, etc. für die Suche einstellen können.
Im Interface-Design wurden bereits Entwürfe für diese Komponente erarbeitet \todo{Anhang}.

Funktionalitäten zur Pflege und Wartung von Assets (F-V-5) ist durch das Assetmangement
Snipe-IT bereits gegeben, diese Prozesse sollten jedoch am \ac{imis} klar kommuniziert werden.


\section{Ausblick}
Die Resultate der Untersuchungen im Rahmen der Arbeit können als Indikator und Grundlage für
weiterführende Arbeiten und Untersuchungen eingesetzt werden. Insbesondere für Erhebungen mit der
Nutzendengruppe wurden zahlreiche Vorarbeiten geleistet, an die es anzuknüpfen gilt. Ebenso
zentrall sind die offenen Punkte für die Weiterentwicklung des Systems.

\subsection{Weiterentwicklung der Funktionalitäten}
Das Filtern von Material (F-VA-7) wurde ebenfalls in den Evaluationsergebnissen herausgearbeitet und
sollte bei der Weiterentwicklung des Systems mit berücksichtigt werden. 

Aus der Evaluation ließen sich die Funktionalität \enquote{Nach Zweck suchen} und
\enquote{Set-Vorschläge} ableiten, welche in der späteren Entwicklung implementiert und evaluiert
werden könnten.

Bevor die ausgeliehenen Assets an Verleihende zurückgeben werden, sollte eine Checkliste
für das jeweilige Asset angezeigt. Dort werden Hinweise angezeigt, wie zum Beispiel
SD-Karte geleert, Assets auf Ursprungseinstellungen zurückgestellt oder Akkus geladen. Diese
Funktionalität soll insbesondere dafür sorgen, dass nachfolgende Ausleihende die Assets
direkt nutzen können.

\subsection{Weiterentwicklung der Kalenderkomponente}
Die im Rahmen dieser Arbeit verwendete Kalenderkomponente
\textit{V-Calendar}\footnote{\url{https://vcalendar.io/layouts.html}} ermöglicht eine leichte
Erweiterbarkeit und sollt insbesondere in der mobile Ansicht beim Bearbeiten des
Reseriveurngszeitraums angepasst werden.

\subsection{LDAP-System Einbindung}
Wie bereits in \ref{sec:punkte} erläutert konnte die Accountbasierte Nutzung des Systems nicht
umgesetzt werden. Für den realen Einsatz des Systems ist diese Funktionalität unabdingbar. Die
Einbindung des  LDAP-Systems deckt zudem die Datenschutz und Zugriffsrechte der Nutzenden ab, sodass
lediglich Nutzende der Universität zugriff auf die Listen der am \ac{imis} bestehenden Assets haben.
Eine mögliche Umsetzung ist in \ref{subsec:heraus} aufgeführt. 

\subsection{Einbindung in das Labormanagementsystem}
Abschließend ist eine Möglicheit, um die Prozesse und die aus der Evaluation erarbeiteten Probleme
der Begrifflichkeiten in Bezug auf den Assetstatus lösen zu können, das erarbeitet Konzept in das
Labormanagementsystem zu \todo{Pabst Zitieren} integrieren. Da sich das Labormanagementsystem
bereits in der zweiten Iteraktion befindet, ist die Integration auf diesem Wege sinnvoll. Dies
ermöglicht ein Reservierungs- und Ausleihsystem für Räume und Assets. Demzufolge können Assets,
welche den Status \enquote{Fest verbaut} aufweisen, über die Raumbuchung genutzt werden. Außerdem
haben Nutzende so nur ein System. Dies löst somit ebenfalls die Funktionalität der Accountbasierten
Nutzung.


