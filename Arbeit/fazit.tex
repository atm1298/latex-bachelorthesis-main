%!TEX root = thesis.tex

\chapter{Zusammenfassung und Ausblick}
\label{chapter-fazit}

Die Zusammenfassung greift die in der Einleitung angerissenen Bereiche wieder auf und erläutert, zu
welchen Ergebnissen diese Arbeit kommt. Dabei wird insbesondere auf die neuen Erkenntnisse und den
Nutzen der Arbeit eingegangen.

Im anschließenden Ausblick werden mögliche nächste Schritte aufgezählt, um die Forschung an diesem
Thema weiter voranzubringen. Hier darf man sich nicht scheuen, klar zu benennen, was im Rahmen
dieser Arbeit nicht bearbeitet werden konnte und wo noch weitere Arbeit notwendig ist.

Die Resultate der Untersuchungen im Rahmen der Arbeit können als Indikator und Grundlage für
weiterführende Projekte und Untersuchungen eingesetzt werden. Insbesondere für Erhebungen mit der
Zielgruppe wurden zahlreiche Vorarbeiten geleistet, an die es anzuknüpfen gilt:



\section{Zusammenfassung}

\section{Offene Punkte}
Funktionalitäten erwähnen! alles was soll und sollte in zukunft + Konzeption!

\section{Ausblick}

\subsection{Weiterentwicklung des Interfacedesigns}

\subsection{Weiterentwicklung der Funktionalitäten}
\subsection{LDAP Einbindung}

\subsection{Einbindung in das Labormanagementsystem}
Labormanagementsystem \todo{Pabst Zitieren}
\section{Abschließendes Fazit}

