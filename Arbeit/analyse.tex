%!TEX root = thesis.tex

\chapter{Anforderungen}
\label{chapter-analyse}

Um die relevanten Anforderungen, die Einfluss auf die Entwicklung und die spätere Nutzung des
Reservierungstools nehmen, identifizieren und einordnen zu können, wurde eine Analyse nach dem
menschenzentrierten Gestaltungsprozess durchgeführt, um unter anderem Benutzende, Aufgaben und
Kontext des Projektes genauer zu verstehen. Nach einer Ausführung der Datenquellen
(\ref{section:daten}) wurden zunächst zwei Benutzergruppen (Verleihende und Ausleihende) mittels
einer Benutzeranalyse (\ref{section:benutzer}) unter der zu Hilfenahme der durchgeführten Interviews
klassifiziert und eingegrenzt. Daraufhin wurden die Probleme und Herausforderungen des aktuellen
Vorgehens und den unterschiedlichen Ausleihprozessen (\ref{section:iststand}) geschildert.
Anschließend wurden die Aufgaben, die Verleihende und Ausleihende mithilfe der Anwendung bewältigen
möchten, diskutiert (\ref{section:aufgaben}) sowie der organisatorische und zeitlich-räumliche
Kontext des Verleihens und Ausleihens am \ac{imis} (\ref{section:kontext}) untersucht. Aufbauend auf
den Resultaten der vorangestellten Untersuchungen wurden die objektiven Anforderungen an den SnipeIT
Companion formalisiert (\ref{section:anforderung}).

\section{Datenquellen}
\label{section:daten}
Im Rahmen der Analyse wurden Stakeholder-Interviews durchgeführt und ein vergleichbares Projekt
konnte im Rahmen eines Interviews als weitere Quelle genutzt werden. Die Befragten wurden durch ein
semi-strukturiertes Interview geführt. Im Vorfeld wurde dafür ein Interviewleitfaden entwickelt
(Anhang Verlinken). Es wurde eine Unterteilung in Verleihende und Ausleihende von Assets vorgenommen
(genauere Definition der Benutzergruppen in \ref{section:benutzer}). Bei den Teilnehmenden handelt
es sich um Mitarbeitende, welche am \ac{imis} tätig sind und Studierende der Medieninformatik an der
Universität zu Lübeck. In \ref{table:v} ist der jeweilige (Haupt-)Zuständigkeitsbereich sowie der
Verleihenden aufgeführt. Verleihenden der Assets können gleichzeitig die Position eines Ausleihenden
einnehmen. Die Ausleihenden umfassen sowohl Bachelor als auch Master Studierende, diese sind in
\ref{table:a} dargestellt. Durch den geplanten Einsatz am \ac{imis} wurde sich zunächst
ausschließlich auf die \ac{wimi} des \ac{imis} und Studierenden der Medieninformatik begrenzt.
Hierbei wurde insbesondere der Fokus auf die Probleme am \ac*{imis} gelegt (genauere Untersuchungen
in der Problemanalyse in \ref{section:iststand}). Die IDs der Teilnehmenden werden als Verweise in
den folgenden Abschnitten verwendet\footnote{die mit * gekennzeichneten Personen wurden gemeinsam
        interviewt}.

Die Interviews wurden aufgezeichnet und anschließend in Teilen, mithilfe eines
vereinfachten Transkriptionssystems, überführt. Daraufhin wurde eine qualitative Inhaltsanalyse
durchgeführt. Hierfür wurden die Transkripte erneut gelesen und die für die Arbeit und
Forschungsfragen relevanten Textstellen kommentiert. Infolgedessen wurden die erarbeiteten
Kommentare in ein Ordnungssystem strukturiert und unter den einzelnen Analysen zusammengefasst
\cite{dresing_praxisbuch_2016}. Des Weiteren wurde mithilfe des \ac{ati} das technische Interesse
und Verständnis der Teilnehmenden festgestellt (\ref{table:ati}), sodass das Reservierungstools
entsprechend an die Benutzergruppen angepasst werden kann \cite{attig_assessing_2017}.

\begin{table}[h]
        \centering
        \caption{Teilnehmende der Interviews, Verleihende}
        \begin{tabular}{lll}
                \arrayrulecolor{maincolor}\arrayrulecolor{maincolor}\hline
                \sffamily\color{maincolor}ID & \sffamily\color{maincolor}Alter &
                \sffamily\color{maincolor}Zuständigkeitsbereich                                   \\
                \arrayrulecolor{maincolor}\arrayrulecolor{maincolor}\hline
                V1                           & 25 - 35 J.                      & Keine direkte
                Zuständigkeit, Zugänge zu verschiedenen Laboren                                   \\
                V2                           & 25 - 35 J.                      & Multimedialabor  \\
                V3                           & 25 - 35 J.                      & VR-Labor         \\
                V4                           & 40 - 59 J.                      & Administratives
                Personal                                                                          \\
                V5                           & 25 - 35 J.                      & Innovationslabor \\
                \arrayrulecolor{maincolor}\arrayrulecolor{maincolor}\hline
        \end{tabular}
        \label{table:v}
\end{table}

\begin{table}[h]
        \centering
        \caption{Teilnehmende der Interviews, Ausleihende}
        \begin{tabular}{lll}
                \arrayrulecolor{maincolor}\arrayrulecolor{maincolor}\hline
                \sffamily\color{maincolor}ID & \sffamily\color{maincolor}Alter &
                \sffamily\color{maincolor}Rolle                                                                            \\
                \arrayrulecolor{maincolor}\arrayrulecolor{maincolor}\hline
                A1                           & 19 - 25 J.                      & Bachelorstudentin, Hilfswissenschaftlerin \\
                A2                           & 19 - 25 J.                      & Bachelorstudent                           \\
                A3                           & 19 - 25 J.                      & Masterstudent, Hilfswissenschaftler       \\
                A4*                          & 19 - 25 J.                      & Bachelorstudentin                         \\
                A5*                          & 19 - 25 J.                      & Bachelorstudentin                         \\
                A6                           & 19 - 25 J.                      & Masterstudentin                           \\
                \arrayrulecolor{maincolor}\arrayrulecolor{maincolor}\hline
        \end{tabular}
        \label{table:a}
\end{table}

\begin{table}[h]
        \centering
        \caption{Werte der \ac{ati}-Skala}
        \begin{tabular}{lccc}
                \arrayrulecolor{maincolor}\arrayrulecolor{maincolor}\hline
                \sffamily\color{maincolor}Benutzergruppe & \sffamily\color{maincolor}Mittelwert
                $(M)$                                    & \sffamily\color{maincolor}Standardabweichung $(SD)$ &
                \sffamily\color{maincolor}Teilnehmende $(N)$                                                              \\
                \arrayrulecolor{maincolor}\arrayrulecolor{maincolor}\hline
                Verleihende                              & 5,00                                                & 0,58 & 3 \\
                Ausleihende                              & 5,13                                                & 0,48 & 6 \\
                \arrayrulecolor{maincolor}\arrayrulecolor{maincolor}\hline
        \end{tabular}
        \label{table:ati}
\end{table}

\section{Benutzeranalyse}
\label{section:benutzer}
Um eine zielgruppengerechte Gestaltung des SnipeIT Companion voraussetzen zu können, werden in
diesem Abschnitt die Benutzergruppen des Reservierungstools eruiert und näher untersucht.
Resultierend aus den Stakeholder-Interviews wurden zwei Benutzergruppen, zu denen Verleihende sowie
Ausleihende eines Assets gehören, für das System erarbeitet. Die Zielgruppe beschränkt sich im
Rahmen dieser Arbeit auf die Mitarbeitenden des \ac{imis} sowie die Studierenden der
Medieninformatik an der Universität zu Lübeck. Für den SnipeIT Companion konnte eine Zielgruppe mit
einer Alterspanne von 17 - 35 Jahren festgelegt werden. Aus den Interviews (\ref{section:daten})
konnte entnommen werden, dass beide Benutzergruppen täglich ein Smartphone, Tablet, Laptop oder
Desktop PC nutzen und somit ein grundlegendes technisches Verständnis vorausgesetzt werden kann.
Diese Behauptung konnte mit den Ergebnissen der \ac{ati}-Skala bestärkt werden (\ref{table:ati}).


%Mitarbeitende, sowie administratives Personal des \ac{imis}, welche verantwortlich für die Ausgabe
%von Assets sind, werden im folgenden als Verleihende bezeichnet (\ref{table:v}).
\subsection{Verleihende}
Die Verleihenden eines Assets unterscheiden sich gering bis gar nicht in ihren soziodemografischen
Daten. Im Kontext der vorliegenden Arbeit sind besonders Technikaffinität und Alter zu beachten.

Mit $N=3$ und einem durchschnittlichen Alter von XX, lag der Wert der \ac{ati}-Skala, aufseiten der
Verleihenden, bei $M=5,00$ mit einer $SD=0,58$, dies weist auf eine geringe $SD$ hin und den
Nutzenden kann eine hohe Technikaffinität vorausgesetzt werden  (\ref{table:ati}).

Die \ac{wimi} mit einem Zuständigkeitsbereich für Assets des \ac{imis} lagen im Großteil bei einer
Alterspanne von 25 - 35 Jahren. Folglich lässt sich eine Alterspanne von 25 - 35 Jahren aufseiten
der \ac{wimi} für den SnipeIT Companion festlegen.

Verleihende umfassen ausschließlich Mitarbeitende, sowie administratives Personal des \ac{imis},
welche verantwortlich für die Ausgabe von Assets sind (\ref{table:v}). Verleihende sind in der Lage
einzelne Assets herauszugeben. Allerdings haben nicht alle Verleihende auf alle Assets den gleichen
Zugriff, da dies von Forschungsgruppe zu Forschungsgruppe unterschiedlich ist. Zudem liegen in den
Forschungsgruppen unterschiedliche Vorgänge vor (genauere Unterschiede zu den Vorgängen in
\ref{section:iststand}). Folglich können Verleihende der Assets gleichzeitig die Position der
Ausleihenden einnehmen.

\subsection{Ausleihende}
Die Ausleihende eines Assets unterscheiden sich nur bedingt in ihren soziodemografischen Daten,
dadurch das auch Verleihende, also Mitarbeitende Ausleihende sein können. Im Kontext der
vorliegenden Arbeit sind besonders Technikaffinität und Alter zu beachten.

Mit $N=6$ und einem durchschnittlichen Alter von XX, lag der Wert der \ac{ati}-Skala, aufseiten der
Ausleihenden, bei $M=5,13$ mit einer $SD=0,48$, dies weist auf eine geringe $SD$ hin und den
Nutzenden kann eine hohe Technikaffinität vorausgesetzt werden  (\ref{table:ati}).

Das Alter der studierenden Deutschen betrug im Sommersemester 2012 im Durchschnitt 24,4
Jahre. \cite{middendorff2017wirtschaftliche}. In den Jahren 2019/20 lag das Durchschnittsalter der
insgesamt 2,9 Millionen Studierenden bei 23,4 Jahren \cite{noauthor_studierende_nodate}. Mit einem
durchschnittlichen Alter von XX der befragten Personen lässt sich eine Alterspanne von 17 - 25
Jahren aufseiten der Studierenden für den SnipeIT Companion festlegen.

Bei den Ausleihenden handelt es sich insbesondere um Studierende, welche keinen direkten Zugang zu
den Assets haben (\ref{table:a}). Ausleihende suchen Mitarbeitende aktiv auf oder kontaktieren jene,
um Informationen über ausleihbare Assets zu erhalten. Wie bereits geschildert, können durch die
Forschungsgruppen auch Mitarbeitende, also Verleihende zu Ausleihenden werden, wobei für dergleichen
häufig ein anderer Ausleihprozess als für Studierende vorliegt (\ref{section:iststand}).


\section{Problemanalyse}
\label{section:iststand}

Um die Relevanz des Verleihens am \ac{imis} sowie die Prozesse und damit einhergehenden
Problematiken besser nachvollziehen zu können, wurde eine Problemanalyse auf Basis des aktuellen
Vorgehens getroffen. Der Übersichtlichkeit wegen werden im folgenden Probleme, welche Verleihende
und Ausleihende betreffen thematisiert. Daraufhin werden Probleme der einzelnen Benutzergruppen
näher erläutert.

\subsection{Probleme: Allgemein}
\label{section:probleme-allgemein}
Eines der größten Probleme im derzeitigen Ablauf ist das nicht vorhanden sein einer öffentlichen
Liste für Studieren, über die ausleihbare Assets eingesehen werden können. Auch aufseiten der
Verleihenden, also unter den \ac{wimi} ist keine vollständige interne Übersicht vorhanden (alle).
Dies führt dazu, dass aufgrund von Unwissenheit wenig Assets ausgeliehen werden können (alle V). Durch
die verschiedenen Forschungsgruppen am \ac{imis} und die damit verbundenen Labore, gibt es
verschiedene Ansprechpartner:innen für die jeweiligen Assets in den Laboren. Die Zuständigkeit
dieser Ansprechpartner:innen ist jedoch nicht ausreichend ersichtlich, sodass es häufig zu
Weiterverweisen, an andere Ansprechpartner:innen kommt, wobei auch hier unter \ac{wimi} nicht immer
klare Ansprechpartner:innen klar sind (V1,V3, V4, A1, A2, A3).

Studierende, welche als \ac{hiwi} am \ac{imis} angestellt sind, haben beim Ausleihen häufig einen
Vertrauensvorschuss, sodass bei kurzer Ausleihe für Tätigkeiten im Gebäude, Assets verliehen oder
entnommen werden, ohne dies zu Vermerken. Dies gilt auch für \ac{wimi} (A1, V1, V2). Dies erschwert
insbesondere das Planen mit Assets (A3, A6).

Beim Verleihen von Assets kommt es häufig vor, dass insbesondere Studierende, ohne Wissen ein Asset
ausleihen wollen. Aufseiten der Verleihenden herrscht hier ein Ungleichgewicht, während einige auf
die Selbstaneignung und Google verweisen, ist es anderen wichtig, den Use Case der Nutzung zu
verstehen und die damit verbundenen Einstellungen der Assets sowie weiteres Zubehör zu empfehlen
(V1, V2, V4, V5, A3, A6).

Sollten Beschädigungen an einem Assets vorliegenden, sind diese nicht festgehalten, sodass ein
Verzeichnis für Gebrauchsspuren oder Macken von Asset keine Auskunft geben kann (A1, V1, V5).

\subsection{Probleme: Verleihende}
\label{section:probleme-verleihende}
Eine zentrale Schwachstelle des aktuellen Vorgehens sind die uneinheitlichen Prozesse. Das Ausleihen
von Assets wird von Forschungsgruppe zu Forschungsgruppe unterschiedlich gehandhabt, bzw. innerhalb
der Forschungsgruppe liegen auch Unterschiede im Prozess vor (V1, V2,V3). Wie bereits in den
Allgemeinen Problemen (\ref{section:probleme-allgemein}) geschildert ist es einigen Verleihenden
wichtig, dass Ausleihende über die Assets Bescheid wissen und der Use Case detailliert besprochen
und erläutert wird, für wiederum andere ist die Vermittlung dieses Wissens nicht von Bedeutung
(V1,V2,V3,V4) (siehe Probleme: Ausleihende).

Um ein Asset ausleihen zu können, müssen Ausleihende auf einem Formular unterschreiben, in anderen
Fällen wird auf einem beliebigen Zettel unterschrieben (V1, V2, V3, V4), dies führt mitunter zu einer
unübersichtlichen Zettelwirtschaft (V4, V5). Wiederum wird durch das Vertrauen am \ac{imis} und bei
kurzen Dauern nicht dokumentiert, wer oder wie lange das Gerät genutzt wird (V1, V2). Die
Mitarbeitenden, welche von anderen Forschungsgruppen ein Asset ausleihen möchten, haben häufig einen
anderen Ausleihprozess als Studierende, da der Vertrauens- und Bekanntheitsgrad ein anderer ist
(V1,V2,V3). Durch die mangelnde und uneinsichtige Dokumentation des Ausleihens kann ein spontanes
Planen erschwert werden (V1, V2, V3).

Da auch eine interne Übersicht der verfügbaren Assets für Verleihende nicht vorhanden ist, kommt es
durch Unwissenheit zu Doppelbeschaffung (V1, V2, V3). Folglich kann es zu Anschaffungen kommen,
welche mit dem vorhanden Assets nicht kompatible sind (V2, V3). Eine Schwachstelle liegt im
Beschaffen von Assets, ohne dass diese vermerkt werden. Dies führt dazu, dass Assets in einzelnen
Büros liegen oder vergessen werden, obwohl diese für laufende Studien o. Ä. sinnvoll sein können
(V3). Es ist unter anderem vorgekommen, dass angeschaffte aber verschwunden geglaubte Assets nach
zwei Jahren wieder gefunden wurden (V1, V2, V3).

Eine weitere Schwachstelle lässt sich in der Wartung von Assets feststellen. Mitarbeitenden, welche
ein Asset angeschafft sind für dieses zuständig. Jedoch Fehlen Übersicht und Erinnerungen für
Wartungen und Updates der Assets, sodass es zu Entladung von Akkus kommen kann (V1, V2, V5).

\subsection{Probleme: Ausleihende}
\label{section:probleme-Ausleihende}
Für Studierende ist das Ausleihen insofern schwer, als das höchstens über Flurfunk, Videoworkshop
oder vereinzelnd Übungen über einzelne Assets berichtet wird (alle As). Der Vorgang, per E-Mail sei
für einige Studierende entspannt, trotzdem erschwert es das Ausleihen, wenn nicht Bescheid gegeben
wird was ausleihbar ist (A4, A5). Wiederum andere empfinden das E-Mail schreiben als störend,
insbesondere, wenn bewusst ist, wie viel \ac{wimi} zu tun haben und die Antwortzeit entsprechend
hoch ist. Daher ist spontanes Planen unzuverlässig (A1, A6). Spontanes nachfragen nach Assets ist
aufseiten der Studierenden weniger zu sehen. Hier wird vorher eine E-Mail an einen \ac{wimi}, in den
meisten Fällen an die verantwortlichen Übungsleiter oder die Studiengangkoordination geschrieben,
weil das Risiko zu hoch sei, das Asset nicht verfügbar sind (A3). Stark bemängelt wird, dass
es keine Garantie gibt, dass etwas zum gewünschten Zeitpunkt ausleihbar ist (A1, A3) und noch dazu
kein schnelles Herankommen ermöglicht werden kann (A3). Außerdem wird kritisiert, dass das
Nachfragen als nervig wahrgenommen wird, weil \ac{wimi}, für simple Fragen, die eine Liste der
Assets beantworten würde, aus der Arbeit gerissen werden (A6). Dazu kommt, dass Ausleihende nicht
genau wissen, um was für ein Gerät es sich genau handelt, so kann es zu Kompatibilitätsfehlern
kommen, weil die Auskunft fehlt und eine Selbstaneignung vorausgesetzt wird (A3). Dies führt dazu,
das Projekt nicht mit dem Asset umgesetzt werden konnten (A3). Folglich können Ausleihende nur
bedingt mit den Geräten umgehen.


\section{Aufgabenanalyse}
\label{section:aufgaben}
Durch die Schritte, welche Verleihende und Ausleihende im Ausleihprozess durchlaufen, konnten
resultierend auf Basis der Interviews (\ref{section:daten}) Aufgaben erarbeitet werden, welche von
dem Reservierungstools übernommen oder unterstützt werden können. Die Aufgaben wurden anhand des
aktuell idealen und vorgesehenen Ausleihprozess in drei Bereiche eingeordnet. Im ersten Bereich
handelt es sich um die Vorbereitung, welche zum Ausleihen eines Assets getroffen werden müssen.
Darauffolgend werden die Aufgaben der Ausgabe definiert. Der dritte Bereich umfasst die Rückgabe der
Assets. Ergänzend zu den zuvor genannten Bereichen wurden Aufgaben für die Wartung der Assets
dargestellt.

% Aufgabe im Bereich der Vorbereitung
\subsection{Aufgaben im Bereich der Vorbereitung}
Um ein Asset ausleihen zu können, müssen bestimmte Vorbereitungen getroffen werden, diese werden im
Folgenden näher erläutert (\ref{table:Ag-Vt}).

\begin{table}[h]
        \centering
        \caption{Aufgaben im Bereich der Vorbereitung}
        \begin{tabular}{ll}
                \arrayrulecolor{maincolor}\hline
                \sffamily\color{maincolor}ID & \sffamily\color{maincolor}Aufgabe \\
                \arrayrulecolor{maincolor}\hline
                Ag-Vt-1                      & Verfügbarkeit anfragen            \\
                Ag-Vt-2                      & Verfügbarkeit einsehen            \\
                Ag-Vt-3                      & Reservierungen von Assets         \\
                Ag-Vt-4                      & Beratungsgespräch                 \\
                \arrayrulecolor{maincolor}\hline
        \end{tabular}
        \label{table:Ag-Vt}
\end{table}

Um ein Asset ausleihen zu können, muss eine Anfrage an die Verantwortlichen gesendet werden, dies
geschieht meist per E-Mail. Ausleihende fragen, aufgrund des mangelnden Wissen, nach einem direkten
Asset, welches über bspw. den Flurfunk an diese gelangt ist (\textit{Ag-Vt-1 | Verfügbarkeit
        anfragen}). Wie bereits in der Problemanalyse geschildert (\nameref{section:probleme-allgemein}),
gibt es keine Übersicht über ausleihbare Assets. Dies zeigt die Dringlichkeit des SnipeIT Companion
für eine bessere Vorbereitung.\\
Um ein Asset ausleihen zu können ist es sowohl für Verleihende als auch Ausleihende von Bedeutung,
ob Assets verfügbar sind. Verleihende überprüfen, ob das angefragte Asset im Schrank vorhanden ist.
Hierbei ist zu berücksichtigen, dass keine langfristige Planung in die Zukunft gewährleistet werden
kann. Dies kann jedoch mit dem SnipeIT Companion mittels eines Kalenders und einer
Reservierungsfunktion ermöglicht werden (\textit{Ag-Vt-2 | Verfügbarkeit einsehen}).\\
Wie in \textit{Ag-Vt-2 | Verfügbarkeit einsehen} bereits geschildert, kann eine langfristige Planung
in die Zukunft nicht gewährleistet werden. Liegt eine Anfrage in der Zukunft vor, wird diese mittels
eines Klebezettels am gewünschten Asset vermerkt. Am Ende muss gehofft werden, dass die Notiz von
\ac{wimi} berücksichtigt wird und das Asset zum gewünschten Zeitraum verfügbar ist (\textit{Ag-Vt-3
        | Reservierungen von Assets}). \\
Die zuvor erhaltene Anfrage aus \textit{Ag-Vt-1 | Verfügbarkeit anfragen} wird von Verleihenden
verarbeitet, wobei in den einigen Fällen der Use Case des Ausleihens eruiert wird, um den
Ausleihenden Tipps zu vermitteln. Um den Ausleihprozess für Verleihende zu erleichtern, kann der
vorangestellt Use Case mittels eines Dialogs im SnipeIT Companion ermöglicht werden. So können
aufbauend auf dem Dialog, direkt Vorschläge an Ausleihende gegeben werden (\textit{Ag-Vt-4 |
        Beratungsgespräch}).


% Aufgabe der Ausgabe 
\subsection{Aufgaben im Bereich der Ausgabe}
Im nächsten Abschnitt werden alle zentralen Aufgaben aufgeführt, welche für die Übergabe von
Assets relevante sind (\ref{table:Ag-Au}).

\begin{table}[h]
        \centering
        \caption{Aufgaben im Bereich der Ausgabe}
        \begin{tabular}{ll}
                \arrayrulecolor{maincolor}\hline
                \sffamily\color{maincolor}ID & \sffamily\color{maincolor}Aufgabe \\
                \arrayrulecolor{maincolor}\hline
                Ag-Au-1                      & Asset-Abholung                    \\
                Ag-Au-2                      & Nutzung von Assets erläutern      \\
                Ag-Au-3                      & Formular unterschreiben           \\
                \arrayrulecolor{maincolor}\hline
        \end{tabular}
        \label{table:Ag-Au}
\end{table}

Ausleihende holen die Assets zum Besprochenen Zeitpunkt in den Laboren des \ac{imis} ab. Verleihende
schaffen hierfür Zugriff zum Labor oder Schrank (\textit{Ag-Au-1 | Abholung}). \\
Ausleihende wissen vorher häufig nicht genau um was für ein Gerät es sich genau handelt, so kann es
zu Kompatiblitätsfehlern kommen und das Asset ist für den ursprünglichen Gebrauch nicht nutzbar.
Daher kann in einigen Fällen die \textit{Ag-Au-2 | Nutzung von Assets erläutern} für den
Ausleihenden hilfreich sein. Im besten Fall ist eine Übersicht mit Informationen wie: Name,
Seriennummer, usw. sowie die dazugehörige Anleitung bereits vor dem Ausleihprozesse verfügbar,
sodass Ausleihende sich im Vorhinein selbständig besser informieren können.\\
Im Idealfall werden die ausgeliehenen Assets in einem Formular dokumentiert, welches von den
Ausleihenden unterzeichnet wird. Das Formular wird bis zur Rückgabe aufbewahrt. Durch die
unterschiedlichen Vorgänge ist insbesondere diese Aufgabe nicht einheitlich (\textit{Ag-Au-3 |
        Formular unterschreiben}).

% Aufgaben der Rückgabe
\subsection{Aufgaben im Bereich der Rückgabe}
Die nachfolgenden Aufgaben umfassen die Rückgabe der ausgeliehenen Assets (\ref{table:Ag-Rg}).
\begin{table}[h]
        \centering
        \caption{Aufgaben im Bereich der Rückgabe}
        \begin{tabular}{ll}
                \arrayrulecolor{maincolor}\hline
                \sffamily\color{maincolor}ID & \sffamily\color{maincolor}Aufgabe \\
                \arrayrulecolor{maincolor}\hline
                Ag-Rg-1                      & Asset-Rückgabe                    \\
                Ag-Rg-2                      & Asset-Überprüfung                 \\
                \arrayrulecolor{maincolor}\hline
        \end{tabular}
        \label{table:Ag-Rg}
\end{table}

Für \textit{Ag-Rg-1 | Rückgabe} der ausgeliehenen Assets wird auf dem Formular dokumentiert, wann und was
zurückgegeben wurde. Die Rückgabe wird meist während der Abholung besprochen oder per E-Mail. In den
Fällen, in denen kein Formular ausgefüllt wurde, wird der Klebezettel mit der Unterschrift entsorgt
und in wieder anderen Fällen, werden die Assets ohne weitere Dokumentation in die Schränke und Räume
zurückgebracht.

Für einige Verleihende fällt \textit{Ag-Rg-2 | Überprüfung} der Assets nach einer Rückgabe an. Hier
wird geschaut, dass bspw. SD Karten gelehrt wurden und Akkus geladen. Außerdem sollten Einstellungen
an den Assets zurückgesetzt werden. Diese Aufgaben sollte im besten Fall von Ausleihenden selbst
erledigt werden, jedoch fehlt auch hier wieder die Aufklärung, welche vom SnipeIT Companion
übernommen werden kann.
% Aufgaben der Wartung
\subsection{Aufgaben im Bereich der Wartung}
\label{subsec:wartung}
Im Folgenden werden Aufgaben, welche für Verleihende auf administrativer Ebene von Bedeutung sind
näher erläutert (\ref{table:Ag-Wt}).

\begin{table}[h]
        \centering
        \caption{Aufgaben im Bereich der Wartung}
        \begin{tabular}{ll}
                \arrayrulecolor{maincolor}\hline
                \sffamily\color{maincolor}ID & \sffamily\color{maincolor}Aufgabe \\
                \arrayrulecolor{maincolor}\hline
                Ag-Wt-1                      & Pflege von Assets                 \\
                Ag-Wt-2                      & Pflege von Neuanschaffung         \\
                \arrayrulecolor{maincolor}\hline
        \end{tabular}
        \label{table:Ag-Wt}
\end{table}

Assets, die längere Zeit nicht genutzt werden, müssen von Verleihenden gewartet werden \textit{(Ag-Wt-1)}. Dieses gerät
in manchen Fällen eher in Vergessenheit. Bei Neuanschaffung, sollten \ac{wimi} über diese informiert werden. Dies gerät auf Ebene der
Forschungsgruppe häufig in Vergessenheit, weil keine Übersicht vorhanden ist \textit{(Ag-Wt-2)}.



\section{Kontextanalyse}
\label{section:kontext}

Für die Ermittlung der Nutzungsumgebung, in der das System verwendet werden soll, wurde eine
Kontextanalyse, basierend auf den vorangehenden Analysen sowie den Interviews durchgeführt. Zunächst
wurde der organisatorische Kontext des Systems festgehalten. Anschließend wird der
zeitlich-räumliche Kontext eruiert \cite{herczeg_software-ergonomie_2018}.

\subsection{Organisatorischer Kontext}
Unter Berücksichtigung von sozialen Strukturen kann maßgeblich die Qualität des Systems positiv
beeinflusst werden \cite{herczeg_software-ergonomie_2018}. Aus diesem Grund wurden die Strukturen untersucht
und in formelle und informelle organisatorische Strukturen unterteilt.

\subsubsection{Formelle Organisation}
Innerhalb des Universitäts-Kontextes gibt es aus formaler Sicht eine überwiegend flache Hierarchie
zwischen Studierenden und Mitarbeitenden, wobei zwischen Hilfswissenschaftlern:innen,
wissenschaftlichen Mitarbeitenden sowie Professor:innen unterschieden werden kann. Diese Gruppen
weisen teilweise verschiedene Zugriffe auf Labor und Schränke, welche die Assets
beinhaltet auf.

Um die \nameref{subsec:wartung} berücksichtigen zu können, sollten Verleihende einen administrativen
Zugang zum System erhalten, um neue Assets eintragen zu können \textit{(Ag-Wt-2)}. Des Weiteren
sollte so ein Überblick über Updates ö.Ä. gegeben werden können \textit{(Ag-Wt-1)}.

\subsubsection{Informelle Organisation}
Aus informeller Sicht ist die Verbindung zwischen Mitarbeitenden, meist kollegial, was zu Problemen
im Ausleihprozess führen kann \nameref{section:probleme-verleihende}.

\subsection{Zeitlich-Räumlicher Kontext}
\label{section:zeit}
Der zeitlich-räumliche Kontext sollte sowohl aus Sicht der Verleihenden als auch aus Sicht der
Ausleihenden analysiert werden, da die Anwendung einen einheitlichen Ausleihprozess schaffen soll.

\subsubsection{Verleihende}
Mitarbeitende halten sich entweder in ihren Büros oder an einem anderen Ort auf, daher werden bei
der Analyse des zeitlich-räumlichen Kontextes beide Fälle betrachtet.

Befinden sich Mitarbeitende im Büro, arbeiten diese an einem Desktop-Arbeitsplatz. Der Computer ist
dabei die meiste Zeit eingeschaltet, daher wäre ein System in Form einer Web-App denkbar sinnvoll
(V1,V5). Wenn Mitarbeitende das Büro verlassen, um ein Asset zu Verleihen, können sich \ac{wimi} in
verschiedenen Laboren befinden. Da der Ort der Nutzung unter anderem durch Home-Office variiert,
liegt ein mobiler Nutzungskontext vor, welcher beispielsweise durch die Nutzung einer Web-App auf
dem Smartphone ermöglicht werden kann (V1, V2, V3, V5). Die Bedienung der Anwendung sollte
niedrigschwellig sein, da Mitarbeitende häufig nicht viel Zeit für die Bedienung haben oder
investieren möchten (V1, V2, V3). Das System sollte einen pragmatischen Zweck erfüllen und kein zu
großes Konzept umfassen, sodass womöglich neue Anläufe dazu kommen und sich die Arbeit zweckmäßig
eher erhört, als verringert (V2). Außerdem sollte durch das Reservieren

\subsubsection{Ausleihende}
Studierende arbeiten viel im Home-Office und selten im \ac{imis}. Daher sollte das Buchen jederzeit
und ortsunabhängig möglich sein. Das Buchen von Assets sollte auch in spontanen Momenten möglich
sein. Daher liegt ein mobiler Nutzungskontext vor, welcher beispielsweise durch die Nutzung einer
Web-App auf dem Smartphone ermöglicht werden kann (A1, A3, A6).


\section{Formalisierte Anforderungen}
\label{section:anforderung}
%Wie eingangs er-wähnt, definieren die Anforderungen, was das System zu leisten hat, während die
%Funktionalitä-ten definieren, wie das System diese gewährleistet.

Im Folgenden werden systematisch formalisierte Anforderungen präsentiert, welche die Ergebnisse der
Analysen abschließend zusammenfassen. Es werden zunächst die Visionen und Ziele
(\ref{section:visionziel}) definiert, des Weiteren werden die Rahmenbedingungen
(\ref{section:rahmen}) und der Kontext des Systems (\ref{section:kontextueberblick}) dargestellt.
Darauf aufbauend wird eine funktionale Anforderung erstellt (\ref{section:funktionale}).
Abschließend werden die Qualitätsanforderungen formuliert (\ref{section:qualität}).


\subsection{Vision und Ziele}
\label{section:visionziel}
Zunächst werden die Visionen und Ziele des Systems konkretisiert, an denen sich die Anforderungen
auf Zielgerichtetheit überprüfen lassen \cite{balzert2009}. Diese setzen sich aus der Analyse der
Benutzenden sowie Aufgaben und des Kontextes zusammen. Zunächst werden die Visionen für die Zukunft
realitätsnah festgelegt.

\begin{center}
        \renewcommand{\arraystretch}{1.5}
        \begin{longtable}{lp{0.85\textwidth}} \arrayrulecolor{maincolor}\hline
                \sffamily\color{maincolor}/V10/ & Der Ausleihprozess von Assets am \ac{imis}
                verläuft einheitlich.                                                           \\
                \sffamily\color{maincolor}/V20/ & Ausleihbare Assets des \ac{imis} sind allen
                Studierenden und Mitarbeitenden bekannt und werden von beiden Gruppen genutzt.  \\
                \sffamily\color{maincolor}/V30/ & Der SnipeIT Companion unterstützt Ausleihende
                effizient mit individuellen und anwendungsspezifischen Assetvorschlägen.        \\
                \sffamily\color{maincolor}/V40/ & Die Planung und Kommunikation zwischen
                Verleihenden und Ausleihenden verläuft reibungsloser.                           \\
                \sffamily\color{maincolor}/V50/ & Verleihende fühlen sich durch den SnipeIT
                Companion unterstützt.                                                          \\
                \arrayrulecolor{maincolor}\hline
        \end{longtable}
\end{center}
\vspace*{-1.5cm}

Basierend auf diese Visionen lassen sich die Ziele formulieren, welche die Visionen
operationalisieren. Diese folgen dabei den standardisierten Regeln zur Formulierung von Zielen
\cite{pohl_requirements_2008}.

\begin{center}
        \renewcommand{\arraystretch}{1.5}
        \begin{longtable}{lp{0.85\textwidth}} \arrayrulecolor{maincolor}\hline
                \sffamily\color{maincolor}/Z10/ & Verleihende und Ausleihende sollen jederzeit in
                der Lage sein, ein gebrauchstaugliches, niedrigschwelliges Interface zum Ausleihen
                von Assets verwenden.                                                                \\
                \sffamily\color{maincolor}/Z20/ & Ausleihende sollen jederzeit standortunabhängig in
                der Lage sein, die Verfügbarkeit von Assets einsehen zu können und diese zu buchen,
                (damit die direkt und ohne Umwege für den Ausleihenden effizient innerhalb einer
                Anwendung erledigt werden kann).                                                     \\
                \sffamily\color{maincolor}/Z30/ & Ausleihende eines Assets sollen jederzeit
                zielgerichtete und aktuelle Informationen zum Asset erhalten.                        \\
                \sffamily\color{maincolor}/Z40/ & Ausleihende sollen jederzeit in der Lage sein,
                sich über Assets zu informieren, um initiale Nutzungsbarrieren zu überwinden und auf
                die Nutzung des Assets vorzubereiten.                                                \\
                \sffamily\color{maincolor}/Z50/ & Verleihende sollen jederzeit in der Lage sein, vom
                System gesammelte Daten übersichtlich und strukturiert einzusehen.                   \\
                \sffamily\color{maincolor}/Z51/ & Verleihende sollen jederzeit standortunabhängig in
                der Lage sein, alle vorhandenen Assets am \ac{imis} einzusehen, sodass es zu keinen
                unbeabsichtigten Doppelbeschaffungen kommen kann.                                    \\
                \sffamily\color{maincolor}/Z60/ & Das System soll Informationen zugänglich
                präsentieren.                                                                        \\
                \arrayrulecolor{maincolor}\hline
        \end{longtable}
\end{center}
\vspace*{-1.5cm}
\subsection{Rahmenbedingungen}
\label{section:rahmen}
Die Rahmenbedingungen legen organisatorische und technische Restriktionen für das System oder den
Entwicklungsprozess fest \cite{balzert2009}. Die Bedingungen wurden aus der Benutzer- und
Kontextanalyse abgeleitet.

\begin{center}
        \renewcommand{\arraystretch}{1.5}
        \begin{longtable}{lp{0.85\textwidth}} \arrayrulecolor{maincolor}\hline
                \sffamily\color{maincolor}/R10/ & Das System ist eine informative Web-Anwendung
                (\secref{section:kontext}).                                                        \\
                \sffamily\color{maincolor}/R20/ & Die Zielgruppe sind Mitarbeitende des \ac{imis}
                und Studierende (\secref{section:benutzer}).                                       \\
                \sffamily\color{maincolor}/R21/ & Die Zielgruppe teilt sich in zwei Nutzergruppen:
                die Verleihenden und Ausleihende von Assets. Die Definitionen der Nutzergruppen sind
                in \secref{section:benutzer} zu finden.                                            \\
                \sffamily\color{maincolor}/R30/ & Das System wird von Verleihenden in einem
                Arbeitsplatzsystemkontext und mobilen Kontext genutzt. Von Ausleihenden vorwiegend
                nur im mobilen Kontext (\secref{section:kontext}).                                 \\
                \sffamily\color{maincolor}/R40/ & Das System soll sich vorwiegend im Dauerbetrieb
                befinden (\secref{section:zeit}).                                                  \\
                \sffamily\color{maincolor}/R50/ & Das System muss unbeaufsichtigt zuverlässig
                lauffähig sein.                                                                    \\
                \sffamily\color{maincolor}/R60/ & Die eingesetzte Software ist clientseitig ein
                Webbrowser. Die marktführenden Webbrowser müssen unterstützt werden: Chrome,
                Firefox, Safari \cite{noauthor_browser_nodate}.                                    \\
                \arrayrulecolor{maincolor}\hline
        \end{longtable}
\end{center}

\vspace*{-1.5cm}
\subsection{Kontext und Überblick}
\label{section:kontextueberblick}
Ein System ist in einer technischen Umgebung eingebettet \cite{balzert2009}. Es wurde im folgenden
Bezug auf das aktuelle Vorgehen mithilfe des \secref{section:iststand} geschlossen.

\begin{center}
        \renewcommand{\arraystretch}{1.5}
        \begin{longtable}{lp{0.85\textwidth}} \arrayrulecolor{maincolor}\hline
                \sffamily\color{maincolor}/K10/ & Der aktuelle Ausleihprozess umfasst keine
                vollständige Liste der Assets für Verleihende und Ausleihende.                    \\
                \sffamily\color{maincolor}/K20/ & Es existieren von Forschungsgruppe zu
                Forschungsgruppe unterschiedliche Ausleihprozesse.                                \\
                \sffamily\color{maincolor}/K30/ & Es existieren Formular, mit denen das Verleihen
                dokumentiert wird.                                                                \\
                \sffamily\color{maincolor}/K40/ & Im Rahmen eines Pilotprojekts existiert das
                Asset-Management-System SnipeIT.                                                  \\
                \arrayrulecolor{maincolor}\hline
        \end{longtable}
\end{center}

\vspace*{-1.5cm}

\subsection{Funktionale Anforderungen}
\label{section:funktionale}
Im Folgenden werden die Kernfunktionalitäten des Systems aufgeführt \cite{balzert2009}. Diese
ergeben sich aus allen Teilanalysen und den festgelegten Zielen. Um die Anforderungen mit einer
eindeutigen Semantik zu formulieren, wurde eine Anforderungsschablone (\ref{fig:schablone})
verwendet, um natürlichsprachliche Anforderungen zu definieren \cite{balzert2009}.

\begin{figure}[h]
        \centering
        \includegraphics[scale=0.45]{Bilder/anforderungsschablone.pdf}
        \label{fig:schablone}
        \caption[Anforderungsschablone]{Anforderungsschablone \cite{balzert2009}}
\end{figure}


\begin{center}
        \renewcommand{\arraystretch}{1.5}
        \begin{longtable}{lp{0.85\textwidth}} \arrayrulecolor{maincolor}\hline
                \sffamily\color{maincolor}/F10/  & Das System \textit{muss} Verleihenden und
                Ausleihenden die Möglichkeit bieten, alle Assets jederzeit mittels einer Übersicht
                einsehen zu können \textit{(Ag-Vt-1, /Z10/)}.                                           \\
                \sffamily\color{maincolor}/F20/  & Das System \textit{muss} Verleihenden und
                Ausleihenden die Möglichkeit bieten, die Verfügbarkeit eines Assets einsehen zu
                können \textit{(Ag-Vt-2, /Z20/)}.                                                       \\
                \sffamily\color{maincolor}/F30/  & Das System \textit{muss} Verleihenden und
                Ausleihenden die Möglichkeit bieten, nach Assets zu Filtern und zu Suchen.              \\
                \sffamily\color{maincolor}/F40/  & Das System \textit{muss}  Verleihenden und
                Ausleihenden relevante Informationen zu den Assets anzeigen (Bild, Name,
                Beschreibung und Seriennummer) \textit{(Ag-Vt-4, Ag-Au-2, /Z20/, /Z40/)}.               \\
                \sffamily\color{maincolor}/F50/  & Das System \textit{muss}  Ausleihenden relevante
                Informationen zu Ansprechpartner:innen anzeigen \textit{(Ag-Vt-4)}.                     \\
                \sffamily\color{maincolor}/F60/  & Das System \textit{muss} Verleihenden und
                Ausleihenden die Möglichkeit bieten, Reservierungen von Assets vornehme zu können \textit{(Ag-Vt-3)}.
                \\
                \sffamily\color{maincolor}/F61/  & Das System \textit{muss} Verleihenden und
                Ausleihenden die Möglichkeit bieten, die verfügbaren Zeitslots der Assets einsehen
                zu können \textit{(Ag-Vt-2)}.                                                           \\
                \sffamily\color{maincolor}/F70/  & Das System \textit{muss} Verleihenden und
                Ausleihenden die Möglichkeit bieten, sich mit dem vorhanden IDM (LDAP) Account
                einzuloggen.                                                                            \\
                \sffamily\color{maincolor}/F80/  & Das System \textit{soll} Ausleihenden die
                Möglichkeit bieten, mittels eines Assistenten individuelle und personalisierte
                Asset-Vorschläge zu erhalten \textit{(Ag-Vt-3, Ag-Au-2)}.                               \\
                \sffamily\color{maincolor}/F90/  & Das System \textit{soll} Ausleihende daran
                erinnern, die ausgeliehenen Assets abzuholen, zurückzubringen oder zu verlängern
                sind \textit{(Ag-Au-1, Ag-Rg-1)}.                                                       \\
                \sffamily\color{maincolor}/F100/ & Das System \textit{soll} Verleihende
                automatisch kontaktieren, wenn ein Asset reserviert wurde (\textit{Ag-Vt-1}).           \\
                \sffamily\color{maincolor}/F110/ & Das System \textit{soll} Verleihende
                automatisch erinnern, wenn der Zugriff zu einem Asset benötigt wird (\textit{Ag-Au-1}). \\
                \sffamily\color{maincolor}/F120/ & Das System \textit{soll} Verleihenden und
                Ausleihenden die Möglichkeit bieten, Mängel, Schäden oder Macken am Asset zu
                Kennzeichen.                                                                            \\
                \sffamily\color{maincolor}/F130/ & Das System \textit{soll} Verleihenden die
                Möglichkeit geben administrative Aufgaben zu erledigen und an Wartungen zu
                erinnern (\textit{Ag-Rg-2}).                                                            \\
                \sffamily\color{maincolor}/F140/ & Das System \textit{sollte in Zukunft}
                Verleihenden und Ausleihenden die Möglichkeit bieten, Assets mithilfe eines
                QR-Scans in den Warenkorb zu packen.                                                    \\
                \sffamily\color{maincolor}/F150/ & Das System \textit{sollte in Zukunft}
                Verleihenden und Ausleihenden die Möglichkeit bieten, Kommentare und
                Erfahrungsberichte unter Assets zu schreiben.                                           \\
                \sffamily\color{maincolor}/F160/ & Das System \textit{sollte in Zukunft}
                Kommunikation mit Verleihenden ermöglichen, sodass keine extra Instanz benötigt
                wird.                                                                                   \\
                \arrayrulecolor{maincolor}\hline
        \end{longtable}
\end{center}

\vspace*{-1.5cm}
\subsection{Qualitätsanforderungen}
\label{section:qualität}
Im letzten Schritt werden die nicht-funktionalen Anforderungen festgelegt, welche die qualitativen
oder quantitativen Eigenschaften eines Systems darstellen \cite{balzert2009}. Auch hier wird, falls
möglich, die Anforderungsschablone aus \ref{fig:schablone} verwendet.

\begin{center}
        \renewcommand{\arraystretch}{1.5}
        \begin{longtable}{lp{0.85\textwidth}} \arrayrulecolor{maincolor}\hline
                \sffamily\color{maincolor}/Q10/ & Das System \textit{muss} den Grundsätzen der DIN
                EN ISO 9241-110:2019-09 (Ergonomie der Mensch-System-Interaktion - Teil 110:
                Interaktionsprinzipien) folgen (\textit{DIN EN ISO 9241-110}, 2019).               \\
                \sffamily\color{maincolor}/Q20/ & Das System \textit{muss} die definierten
                Nutzungsklassen aus \ref{section:benutzer} unterscheiden und die dazugehörigen
                Zugriffsrechte sicherstellen.                                                      \\
                \sffamily\color{maincolor}/Q30/ & Das System \textit{muss} zuverlässig und ohne
                Störung im Dauerbetrieb laufen (\nameref{section:zeit}).                           \\
                \sffamily\color{maincolor}/Q40/ & Das System \textit{soll} modular strukturiert
                sein, damit Inhalte und Funktionalitäten effizient eingebunden werden können und das
                System einfach erweiterbar ist.                                                    \\
                \sffamily\color{maincolor}/Q50/ & Das System \textit{soll} beim Zugriff über das
                Internet eine gesicherte Übertragung (bspw. \ac{HTTPS}) ermöglichen.               \\
                \sffamily\color{maincolor}/Q60/ & Das System \textit{soll} alle
                Benutzerinteraktionen in unter fünf Sekunden ausführen.                            \\
                \arrayrulecolor{maincolor}\hline
        \end{longtable}
\end{center}
