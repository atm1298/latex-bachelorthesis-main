%!TEX root = thesis.tex
\renewcommand{\thesection}{\Alph{section}}
\labelformat{section}{Anhang\ #1}
\chapter{Anhang}
Im Folgenden werden ...

\section{Interviewleitfaden der Analyse}
\label{appendix:interview}

\begin{itemize}
    \item Begrüßung und Danken für die Zeit
    \item Kurzer Umriss des Themas: Ziel in den Interviews ist es, zu erfahren, wie die Organisation
          und Planung unter den Mitarbeiter:innen untereinander sowie zwischen den Mitarbeiter:innen und
          Studierenden aktuell abläuft und wie diese Kommunikation verbessert werden kann, in Bezug auf
          das Ausleihen von Assets + in 2 Teile geteilt
    \item Tätigkeit
    \item Datenschutz
    \item Wenn einverstanden sind, Interview aufzeichnen und würden Sie bitten, dazu das
          Datenschutzformular auszufüllen. Die Aufzeichnungen werden diskret behandelt und nach der
          Auswertung gelöscht
\end{itemize}

\subsection{Verleihende}
{\sffamily\color{maincolor}{Abschnitt: Jetzt}}
\begin{enumerate}
    \item Interessieren, wie Sie vorgehen, wenn Sie eine Anfrage erhalten zum Ausleihen eines
          Assets.
          \begin{enumerate}
              \item Bekommen Sie anfragen per Mail
              \item oder mündlich
              \item Kalender Eintragen (Outlook)/Zettel
              \item Einfach nehmen
          \end{enumerate}
    \item Gibt es eine öffentliche Übersicht (für Studierende) der auszuleihenden Geräte? Nein: Hat
          das schon einmal für Probleme gesorgt im Ausleihprozess? Im Nachhinein gehört, dass Assets
          benötigt wurden
    \item Können Personen vorläufig Systeme reservieren, z. B. in 2 Wochen für 4 Tage?
          Warum nicht? Probleme
    \item Wird sichergestellt, dass Ausleihende mit dem Gerät umgehen können?
    \item Wie sind die Assets versichert bzw. was passiert, wenn es kaputtgeht?
    \item Sehen Sie Probleme oder sind Sie mit dem derzeitigen Ablauf zufrieden?
    \item Fehlgeleitet Anfragen, direkte Ansprechpartner:innen
\end{enumerate}

{\sffamily\color{maincolor}{Abschnitt: Visionen und Ziele}}
\begin{enumerate}
    \item[8.] Vorstellen, es gibt eine webbasierte Anwendung (System)- Online-Plattform
        \begin{enumerate}
            \item Übersicht – Vorstellung?
                  Informationen werden benötigt
                  Form der Darstellung
            \item Was wären weitere Funktionen?
            \item Vorausplanen? Reservieren?
            \item Fehlgeleitet Anfragen, direkte Ansprechpartner:innen
        \end{enumerate}
    \item[9.] Sind Ihnen bis hierher noch Gedanken gekommen, die Sie gerne mit auf den Weg geben wollen?
    \item[10.] Vielen Dank, dass Sie uns für dieses Interview zur Verfügung gestanden haben, wir wären jetzt am Ende des Interviews angelangt
    \item[11.] Im Nachhinein noch melden können
\end{enumerate}

\subsection{Ausleihende}
\begin{itemize}
    \item Begrüßung und Danken für die Zeit
    \item Kurzer Umriss des Themas
    \item Vorerfahrung (HiWi,...)
    \item Datenschutz
\end{itemize}

{\sffamily\color{maincolor}{Abschnitt: Jetzt}}
\begin{enumerate}
    \item Ist Dir/Ihnen bekannt, welche Assets sie/du am IMIS ausleihen können?
          \begin{enumerate}
              \item Nein, was hätten sie/du dann gebraucht? Jetzt, wo du es weißt, würdest du es gerne
                    nutzen? oder Listenübersicht
              \item Vorgehen, wenn Sie eine Anfrage stellen zum Ausleihen eines Assets.
                    \begin{itemize}
                        \item Schauen Sie auch spontan in den Laboren, nach den Geräten vorbei?
                        \item Wie häufig ist Ihr spontaner Besuch (nicht) erfolgreich?
                        \item Was machen Sie, wenn Sie keine Person antreffen, der sie mitteilen, dass sie das Gerät mitnehmen?
                        \item Was wäre für Sie der einfachste Weg, Informationen zu hinterlassen, was würden Sie sich wünschen?
                    \end{itemize}
              \item Sehen Sie hier Probleme oder sind Sie mit dem derzeitigen Ablauf zufrieden?
                    \begin{itemize}
                        \item Darstellung von auszuleihenden Inhalten hilfreich (Übersicht)
                        \item Wie gehen sie vor, wenn sie etwas in 2 Wochen ausleihen wollen?
                    \end{itemize}
          \end{enumerate}
\end{enumerate}

{\sffamily\color{maincolor}{Abschnitt: Visionen und Ziele}}
\begin{enumerate}
    \item[8.] Vorstellen, es gibt eine webbasierte Anwendung (System)- Online-Plattform
        \begin{enumerate}
            \item Übersicht – Vorstellung?
                  Informationen werden benötigt
                  Form der Darstellung
            \item Was wären weitere Funktionen?
            \item Vorausplanen? Reservieren?
            \item Fehlgeleitet Anfragen, direkte Ansprechpartner:innen
        \end{enumerate}
    \item[9.] Sind Ihnen bis hierher noch Gedanken gekommen, die Sie gerne mit auf den Weg geben wollen?
    \item[10.] Vielen Dank, dass Sie uns für dieses Interview zur Verfügung gestanden haben, wir wären jetzt am Ende des Interviews angelangt
    \item[11.] Im Nachhinein noch melden können
\end{enumerate}

\section{Die zehn Usability Heuristiken nach Jakob Nielsen}
\label{appendix:Heuristiken}
\begin{enumerate}
    \item {\sffamily\color{maincolor}{Sichtbarkeit des Systemstatus}} Das System informiert den
Nutzer immer darüber, was gerade passiert – rechtzeitig und durch angemessenes Feedback.
\item {\sffamily\color{maincolor}{Übereinstimmung von System und Wirklichkeit}} Das System spricht
die Sprache des Nutzers – mit ihm vertrauten Wörtern, Phrasen und Konzepten. Entlehnt aus der echten
Welt erscheinen Informationen in ihrer natürlichen und logischen Ordnung. 
\item {\sffamily\color{maincolor}{Nutzerkontrolle und Freiheit}} Nutzer führen Aktionen oft
unbeabsichtigt durch. Auswege wie „Rückgängig”, „Wiederholen” und „ESC” sind deshalb immer möglich
und sichtbar.  
\item {\sffamily\color{maincolor}{Beständigkeit und Standards}} Nutzer müssen nicht
überlegen, ob unterschiedliche Wörter, Situationen und Aktionen das Gleiche meinen. Die Konventionen
des Betriebssystems werden eingehalten. 
\item {\sffamily\color{maincolor}{Fehlervermeidung}} Besser als jede gute Fehlermeldung
ist ein sorgfältiges Design, welches Fehler gar nicht erst auftreten lässt. Das System vermeidet
fehleranfällige Situationen oder warnt den Nutzer und lässt ihn die Aktion bestätigen.
\item {\sffamily\color{maincolor}{Wiedererkennung statt Erinnerung}} Durch sichtbare Objekte, Aktionen und Optionen muss der
Nutzer weniger im Gedächtnis behalten. Anleitungen zum Gebrauch des Systems sind sichtbar oder
leicht zu erreichen.
\item {\sffamily\color{maincolor}{Flexibilität und Effizienz}} Kurzbefehle und andere Abkürzungen –
unsichtbar für Neulinge – beschleunigen bei fortgeschrittenen Nutzern die Bedienung. Zusätzlich sind
häufige Aktionen individuell anpassbar. 
\item {\sffamily\color{maincolor}{Ästhetisches und minimalistisches Design}} Dialogfenster enthalten
keine überflüssigen oder nur selten gebrauchten Informationen. Denn jede zusätzliche Information
steht in Konkurrenz mit den relevanten Informationen und mindert deren Sichtbarkeit.
\item {\sffamily\color{maincolor}{Hilfestellung beim Erkennen, Bewerten und Beheben von Fehlern }} Fehlermeldungen sollten in klarer Sprache (kein Code) formuliert sein, das Problem exakt beschreiben und eine konstruktive Lösung vorschlagen.
\item {\sffamily\color{maincolor}{Hilfe und Dokumentation }} Obwohl es besser ist, wenn der Nutzer
ein System ohne Hilfe benutzten kann, ist es manchmal Nötig, eine Dokumentation bereitzustellen. In
dem Fall sind die Informationen einfach zu finden und konzentrieren sich auf die Aufgabe des
Nutzers. Die Dokumentation enthält konkrete Schritte zur Ausführung und beschränkt sich auf das
Wesentliche.
\end{enumerate}


\section{Leitfaden und Aufgaben der Evaluation}
\label{appendix:Evaluation}

\begin{enumerate}
    \item Begrüßung
    \item Erklärung des Evaluationsablauf (Nennung von Erhebungsmethoden, Rahmenbedingungen, Datenschutz)
    \item Zeit für Fragen zu dem Evaluationsablauf
    \item Einleitung
    \item Einführung in den Rahmen der Arbeit/des Projektes
    \item Zeit für Fragen zu der Einführung
    \item Erklärung, Ziel und Ablauf der Evaluationsaufgaben (Erläuterung Think-Aloud)
    \item Zeit für Fragen zu den Evaluationsaufgaben (Zeit dafür, dass Teilnehmer:in sich die Fragen kurz durchlesen kann)
    \item Durchführung der Evaluationsaufgaben (Anwendung Think-Aloud Methode, Beobachtung, Notizen)
    \item Zeit für Fragen zu dem Fragebogen
    \item Fragebogen
    \item Am Ende Bedanken für die Teilnahme
\end{enumerate}

\subsection{Verleihende}
{\sffamily\color{maincolor}{Szenario:}} Mitarbeitende und möchtest überprüfen, wann Abholungen anstehen und ob du Material für eine Studie gebrauchen kannst
\begin{enumerate}
    \item Einloggen in die App
            \begin{itemize}
                \item voraussichtliche Aufgabendauer: 20 Sekunden
                \item Obergrenze: 1:30 Minuten
            \end{itemize}
    \item Entwicklungslaptop wurde bereits abgeholt
            \begin{itemize}
                \item voraussichtliche Aufgabendauer: 2 Minuten
                \item Obergrenze: 3:30 Minuten
            \end{itemize}
    \item Finde das Gerät XY und bringe in erfahre, wo du das Gerät nutzen kannst
            \begin{itemize}
                \item voraussichtliche Aufgabendauer: 1 Minute
                \item Obergrenze: 1:30 Minuten
            \end{itemize}
    \item Zeitsprung: Laptop zurückgebracht
            \begin{itemize}
                \item voraussichtliche Aufgabendauer: 1 Minute
                \item Obergrenze: 2:00 Minuten
            \end{itemize}
    \item Entwicklungslaptop im Zeitraum 24. Oktober-27. Oktober ausleihen (Uhrzeit wie es in den Alltag passt)
            \begin{itemize}
                \item voraussichtliche Aufgabendauer: 30 Sekunden
                \item Obergrenze: 1:30 Minuten
            \end{itemize}
    \item Zeit zum Umschauen geben
\end{enumerate}

\subsection{Ausleihende}

{\sffamily\color{maincolor}{Szenario:}} Studierende und möchtest für ein Projekt Material ausleihen
\begin{enumerate}
    \item Einloggen in die App
            \begin{itemize}
                \item voraussichtliche Aufgabendauer: 20 Sekunden
                \item Obergrenze: 1:30 Minuten
            \end{itemize}
    \item Entwicklungslaptop im Zeitraum 24. Oktober-27. Oktober ausleihen (Uhrzeit wie es in den Alltag passt)
            \begin{itemize}
                \item voraussichtliche Aufgabendauer: 2 Minuten
                \item Obergrenze: 3:30 Minuten
            \end{itemize}
    \item Informationen zur Abholung des Laptops (Kontaktdaten, Ort)
            \begin{itemize}
                \item voraussichtliche Aufgabendauer: 20 Sekunden
                \item Obergrenze: 1:30 Minuten
                \begin{itemize}
                    \item Frage zu dem Laptop
                \end{itemize}
            \end{itemize}
    \item Material: Gurte Status herausfinden
            \begin{itemize}
                \item voraussichtliche Aufgabendauer: 1 Minute
                \item Obergrenze: 1:30 Minuten
            \end{itemize}
    \item Zeitsprung: Laptop abgeholt: Reservierungszeitraum des Laptops verlängern
            \begin{itemize}
                \item voraussichtliche Aufgabendauer: 1 Minute
                \item Obergrenze: 2:00 Minuten
            \end{itemize}
    \item Webcam Reservierung löschen
            \begin{itemize}
                \item voraussichtliche Aufgabendauer: 30 Sekunden
                \item Obergrenze: 1:30 Minuten
            \end{itemize}
    \item Zeitsprung: Laptop erneut ausleihen
            \begin{itemize}
                \item voraussichtliche Aufgabendauer: 30 Sekunden
                \item Obergrenze: 1:30 Minuten
            \end{itemize}
    \item Zeit zum Umschauen geben
\end{enumerate}

\section{Digitale Medien}
Der vorliegenden Arbeit ist ein digitaler Anhang auf einer CD beigefügt. Darauf befinden sich:
{\begin{enumerate}
    \item Interview- und Evaluationsergebnisse (Unterverzeichnis \textit{Befragungen})
    \item Die Mockups der Konzeption (Unterverzeichnis \textit{Mockups})
    \item Bilder und kreativ Einheiten (Unterverzeichnis \textit{Whiteboard})
    \item Die finale Version der Software (Unterverzeichnis \textit{Software})
    \item Eine digitale Ausgabe der Arbeit als PDF-Datei (Hauptverzeichnis)
\end{enumerate}}