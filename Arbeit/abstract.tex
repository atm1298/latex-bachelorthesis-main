%!TEX root = thesis.tex

\cleardoublepage
\thispagestyle{plain}

\pdfbookmark{Kurzfassung}{kurzfassung}
\section*{Kurzfassung}
\todo{Neu: bitte rüberlesen}
Am \ac{imis} werden Assets ohne Anmeldung oder nur mit einer mündlichen Absprache verwendet. Für die
vorausschauende Planung von anstehenden Projekten gibt es keine feste Reservierung oder einen
Überblick, wann die gewünschten Assets wieder verfügbar sind. Zudem ist vielen Mitarbeitenden und
Studierenden der Universität zu Lübeck unklar, welche Assets sich in den Laboren des \ac{imis}
befinden.

Das Ziel der vorliegenden Arbeit ist die Entwicklung eines wirksamen Systems, welches den
Reservierungs- und Ausleihprozess am \ac{imis} einheitlicher, effizienter und zufriedenstellender
lösen lässt. Die Basis für das Reservierungstool schafft die Asset-Managementsoftware
\textit{Snipe-IT}, welche bereits am \ac{imis} eingesetzt wurde.

Aus diesem Grund wurde in der vorliegenden Arbeit den Forschungsfragen nachgegangen, welche zentrale
Schwierigkeiten die aktuelle Planung und Reservierung von Assets für Mitarbeitende und Studierende
mit sich bringt und welche Anforderungen ein System mitbringen sollte, um diese zu adressieren und
reduzieren. Dazu wurden in der vorliegenden Arbeit systematische und standardisierte Anforderungen
an ein solches Reservierungstool auf Basis von Nutzenden, Aufgaben und Kontexten entwickelt und in
einer formalisierten Anforderungsanalyse erfasst. Teile des Konzeptes wurden in Form eines
technologisch reifen Prototyps realisiert. Die Ergebnisse der Evaluation zeigten, dass das System
als Unterstützung wahrgenommen wurde. Aufseiten der Ausleihenden wurden alle Funktionalitäten
überwiegend als hilfreich eingestuft. Auch aus Sicht der Verleihenden war die Nutzung des Systems
überwiegend positiv eingestuft worden.  Da das positive Ergebnis der durchgeführten Studie nicht mit
einem weiteren System verglichen werden konnte, wurde eine weitere ausführlichere Studie mit einem
Vergleichsystem empfohlen. Abschließend wurde die Weiterentwicklung des Systems und Folgestudien zur
Evaluation der Anwendung vorgeschlagen und diskutiert.

\subsection*{Schlüsselwörter}
Mensch-Computer-Interaktion, Reservieren, Assetmanagement, Snipe-IT, Dashboard, Web-App, Framework

\cleardoublepage
\thispagestyle{plain}

\foreignlanguage{english}{%
  \pdfbookmark{Abstract}{abstract}
  \section*{Abstract} 


  \subsection*{Keywords}
  Human-Computer Interaction, Reservation, Asset Management, Snipe-IT, Dashboard, Web App, Framework

  }