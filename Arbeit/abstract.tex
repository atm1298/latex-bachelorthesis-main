%!TEX root = thesis.tex

\cleardoublepage
\thispagestyle{plain}

\pdfbookmark{Kurzfassung}{kurzfassung}
\section*{Kurzfassung}

Am \ac{imis} werden Assets ohne Anmeldung oder nur mit einer mündlichen
Absprache verwendet. Für die vorausschauende Planung von anstehenden Projekten
gibt es keine Reservierungsmöglichkeiten oder einen Überblick, wann die
benötigten Assets wieder verfügbar sind. Zudem ist vielen Mitarbeitenden und
Studierenden der Universität zu Lübeck unklar, welche Assets sich in den Laboren
des \ac{imis} befinden.

Das Ziel der vorliegenden Arbeit ist die Entwicklung eines wirksamen Systems,
welches den Reservierungs- und Ausleihprozess am \ac{imis} einheitlicher,
effizienter und zufriedenstellender gestaltet. Die Basis für das
Reservierungstool schafft die Asset-Managementsoftware \textit{Snipe-IT}, welche
bereits am \ac{imis} eingesetzt wird.

Für die Entwicklung eines Asset-Reservierungstools wurden Forschungsfragen
nachgegangen, welche zentralen Schwierigkeiten die aktuelle Planung und
Reservierung von Assets für Mitarbeitende und Studierende mit sich bringen und
welche Anforderungen ein System mitbringen sollte, um diese zu adressieren und
reduzieren. Dazu werden in der vorliegenden Arbeit systematisch, standardisierte
Anforderungen an ein solches Reservierungstool auf Basis von Nutzenden, Aufgaben
und Kontext entwickelt und in einer formalisierten Anforderungsanalyse erfasst.
Teile des Konzeptes werden in Form eines technologisch reifen Prototyps
realisiert. Die Evaluation des Prototypens konnte aufzeigen, dass ein solches
System als Unterstützung wahrgenommen wird. Abschließend wird die
Weiterentwicklung des Prototyps und Folgestudien zur Evaluation der Anwendung
vorgeschlagen und diskutiert.

\subsection*{Schlüsselwörter}
Mensch-Computer-Interaktion, Reservieren, Assetmanagement, Snipe-IT, Dashboard,
Web-App, Framework

\cleardoublepage
\thispagestyle{plain}

\foreignlanguage{english}{%
  \pdfbookmark{Abstract}{abstract}
  \section*{Abstract}
  At the Institute for Multimedia and Interactive Systems (\ac{imis}), assets
  are used without registration or only with a verbal agreement. For the forward
  planning of upcoming projects, there are no reservation options or an overview
  of when the required assets will be available again. In addition, many
  employees and students of the University of Lübeck are in the dark which
  assets are located in the \ac{imis} laboratories.

The goal of this thesis is to develop an effective system that makes the
reservation and lending process at \ac{imis} more consistent, efficient, and
satisfying. The basis for the reservation tool is provided by the asset
management software Snipe-IT, which is already in use at \ac{imis}.

For the development of an asset reservation tool, research questions were
pursued as to which central difficulties the current planning and reservation of
assets for staff and students entail and which requirements a system should
bring along in order to address and reduce these. To this end, this thesis
develops systematic, standardized requirements for such a reservation tool based
on users, tasks, and context, and captures them in a formalized requirements
analysis. Parts of the concept are realized in the form of a technologically
mature prototype. The evaluation of the prototype could show that such a system
is perceived as support. Finally, the further development of the prototype and
follow-up studies to evaluate the application are proposed and discussed.

  \subsection*{Keywords}
  Human-Computer Interaction, Reservation, Asset Management, Snipe-IT,
  Dashboard, Web App, Framework

}