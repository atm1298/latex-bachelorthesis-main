\chapter{Dialogbeispiel}
\label{chapter-dialogbeispiel}

Der folgende Abschnitt präsentiert das in der vorliegenden Arbeit realisierte System anhand eines
beispielhaften Nutzungsszenarios. Das Szenario Start mit Mila, einer Erstsemester Studentin im Studiengang
Medieninformatik. Mila belegt im ersten Semester das Modul \textit{\ac{emi}}. Das Einführungsmodul
umfasst eine Gruppenarbeit, in der zum Thema VR/AR eine Idee entwickelt werden soll. Am Ende des
Semesters soll das Projekt bei den \textit{Media Momements} in der \ac{emi} Award
App\footnote{RAIMUNDS BA HEHE} ausgestellt werden. Milas Projektgruppe hat sich entschieden eine
AR-Anwendung für Erze und Metalle zu gestalten, bei der die Erze und Metalle mithilfe einer App
eingescannt werden können und die entsprechenden Eigenschaften angezeigt werden. Außerdem sollen die
Erze und Metallen in einer AR-Galerie gespeichert werden. Um das Projekt in der \ac{emi}
Award App präsentieren zu können, will die Gruppe ein Werbevideo für die App aufnehmen. In einem
Videoworkshop von Georg XX, fiktiver \ac{wimi} am \ac{imis}, erfahren Mila und ihre Gruppe, dass
unter anderem Videoequipment über die Web-App \textit{Snipe-IT Companion} an Studierenden verliehen
wird. 

Mila ruft die Ausleih-App unter der URL \textit{https://app.snipe-itcompanion.de/} auf und meldet
sich mit ihren \todo{LDAP-} Daten an. Daraufhin wird sie zu einem bisher leeren Dashboard weitergeleitet und
aufgefordert nach benötigten Material zu suchen. Da Mila sich nicht sicher ist, welche Kamera sie
braucht, schaut sie sich erstmal unter dem Menü in den Kategorien um und findet recht schnell die
Kategorie \enquote{Kameras}. Nachdem sie auf die Seite der Unterkategorien weitergeleitet wurde,
entscheidet Mila sich dafür eine \textit{GoPro} auszuleihen, weil sie so auch Erze und Metallen
Unterwasser aufnehmen können. Mila klickt also auf \enquote{Hinzufügen}, nun wird sie dazu
aufgefordert einen Ausleihzeitraum anzugeben. Milas Gruppe hat entschieden, dass sie am kommenden
Montag bis Mittwoch mit den Aufnahmen fertig sein wollen. Da Mila von 8:00 Uhr bis 10:00 Uhr eine
Vorlesung hat, gibt sie 10:30 Uhr als Abholzeit und als Rückgabezeit an. Nun klick sie auf
\enquote{Reservieren}. Mila überprüft nun, ob ihre Angaben stimmen und stellt fest, dass die
Rückgabezeit am Mittwoch doch nicht passt, daher ändert sie die Uhrzeit auf 12:30 Uhr. Abschließend
bestätigt Mila ihre Reservierung. Da sich die Gruppe nicht sicher ist, welches Mikrofon sie für ihr
Voiceover ausleihen sollen suchen sie zunächst in der Ausleih-App nach Mikrofon und sehen das Georg
XX für diese Zuständig ist. Daraufhin schreibt Mila Georg eine E-Mail, welches Mikrofon er für ein
Voiceover empfehlen würde. Georg gibt in seiner E-Mail zwei Vorschläge. Nachdem Mila Georgs Antwort
erhalten hat, gibt sie in der Suche den Ausleihzeitraum ein und sucht nach den Vorschlägen. Schnell
stellt sie fest, dass nur eines der Mikrofone in dem Ausleihzeitraum verfügbar ist und Reserviert
dieses. 

Georg ist als fiktiver \ac{wimi} am \ac{imis} für zwei Labore zuständig, in denen Material
ausgeliehen werden kann. Zum Feierabend checkt er jedesmal sein Verwaltungsdashboard auf dem Desktop
in der Ausleih-App. Er sieht, das eine neue Reservierung für Montag um 10:30 Uhr von Mila
eingegangen ist. 

Am Sonntag schaut Mila noch einmal in der Ausleih-App rein, um zu erfahren, wo eigentlich der
Abholort der Materialien ist und findet diesen auf der Dashboardansicht direkt.

Nach der Vorlesung am Montag macht sich Mila auf den Weg in das Gebäude 64 zum Abhoolort: LaborXY.
Georg wartet dort bereits auf Mila und erklärt ihr kurz, was sie bei der GoPro noch beachten sollte. 
Nachdem Mila weg ist, trägt Georg auf seinem Smartphone ein, dass das die Geräte abgeholt wurden.
Sobald Georg die Abholung bestätigt hat wird der Status auch in Snipe-IT geändert.\todo{LAURA
mitabreitende guckt was in Snipeit}

Milas Gruppe stellt am Dienstag fest, dass ihnen der Ausleihzeitraum nicht ausreicht und möchte
diesen daher um einen Tag verlängern. Dafür öffnet sie die Ausleih-App und sieht auf dem Dashboard
unter \textit{Laufende} ihre Reservierungen. Daraufhin ändert sie die Daten der beiden Materialien
auf Donnerstag um 9:00 Uhr. Donnerstag um 9:00 Uhr warte Mila bereits auf Georg, der die Materialien
entgegennimmt und in seinem Büro unter Rückgabe die Materialien bestätigt.

Im vierten Semester möchte Mila das Mikrofon für das Modul: \ac{ide} ausleihen, um wieder
ein Voiceover aufnehmen zu können. Sie findet das Mikrofon schnell unter \textit{zurückgegeben} und
leiht das Material erneut aus.