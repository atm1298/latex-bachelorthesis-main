\chapter{Konzeption}
\label{chapter-konzept}

In diesem Kapitel werden die zuvor erarbeiteten Erkenntnisse und Anforderungen genutzt, um konkrete
Funktionalitäten zu definieren. Anschließend wird auf die Systemarchitektur und Wahl der Frameworks
eingegangen. 
\section{Funktionalität}
\label{section:funktionalitaeten}
Im Folgenden werden die für das System relevanten Funktionalitäten aufgeführt. Jene beschreiben, wie
das System die erarbeiteten Anforderungen erfüllt. Zur besseren Lesbarkeit wurden die
Funktionalitäten nach Benutzergruppen aufgeteilt.

\subsection{Funktionalitäten für Ver- und Ausleihende}
Zu Beginn werden Funktionalitäten, welche für Verleihende und Ausleihende von Bedeutung sind,
näher erläutert (\ref{table:ft-b}).

\begin{table}[h]
    \centering
    \caption{Funktionalitäten für VA (Ver- und Ausleihende)}
    \begin{longtable}{lll}
        \arrayrulecolor{maincolor}\hline
        \sffamily\color{maincolor}ID & \sffamily\color{maincolor}Titel    &
        \sffamily\color{maincolor}Anforderungen \\
        \arrayrulecolor{maincolor}\hline
        Ft-VA-1                       & Authentifizierung über LDAP  & \anfref{F70} \anfref{F80} \\
        Ft-VA-2                       & Übersicht über ausleihbaren Assets  & \anfref{V20}
        \anfref{Z20} \anfref{F50} \anfref{K10} \anfref{F10} \anfref{F30} \\
        Ft-VA-3                       & Verfügbarkeit von Assets           & \anfref{V20}
        \anfref{Z20} \anfref{F50} \anfref{K10} \anfref{F10} \anfref{F30} \\
        Ft-VA-4                       & Zuständigkeitsbereich              & \anfref{F50} \\
        Ft-VA-5                       & Benachrichtigungen \& Erinnerungen & \anfref{F100}
        \anfref{F110} \anfref{F120}                                     \\
        Ft-VA-6                       & Material-Suche                     & \anfref{V20}
        \anfref{Z20} \anfref{K10} \anfref{F10} \anfref{F30} \\
        Ft-VA-7                       & Filtern und Sortieren              & \anfref{V30}
        \anfref{F30} \anfref{F70}                                        \\
        Ft-VA-8                       & Detailansicht                      & \anfref{V50}
        \anfref{Z30} \anfref{F40} \anfref{F50}                           \\
        \arrayrulecolor{maincolor}\hline
    \end{longtable}
    \label{table:ft-b}
\end{table}

{\sffamily\color{maincolor}{Ft-VA-1 | Authentifizierung über LDAP}}\\
Mithilfe der Verknüpfung zum LDAP-System der Universität zu Lübeck, können sich Nutzende mit einem
bereits existierenden IDM Account einloggen. Folglich muss kein neues Konto erstellt werden.
Außerdem kann überprüft werden, ob es sich bei der Anmeldung um Studierenden oder \ac{wimi} handelt,
um entsprechende Systemrechte zu vergeben. Des Weiteren verhindert die Nutzung von LDAP das
Eindringen unbefugter Personen. 


    {\sffamily\color{maincolor}{Ft-VA-2 | Übersicht über ausleihbare Assets }}\\
Die Übersicht, der am \ac{imis} vorhandenen Assets, wurde mittels Kategorien umgesetzt. Dafür gibt es
eine Übersicht, bei der alle Assets eingesehen werden können. Die einzelnen Kategorien beinhalten
Unterkategorien. In der Übersicht werden Informationen wie Name, Seriennummer, Marke und Status
eines Assets angezeigt.

    {\sffamily\color{maincolor}{Ft-VA-3 |  Verfügbarkeit von Assets }}\\
Der Status eines Asset \textit{(Verfügbar, Nicht verfügbar, Hinweis)} muss klar ersichtlich sein.
Dies geschieht mittels Color-Coding.  


{\sffamily\color{maincolor}{Ft-VA-4 | Zuständigkeitsbereich }}\\
Um für Ausleihende Kontaktinformationen anzeigen zu können \textit{(Ft-A-6)}, müssen die
Zuständigkeitsbereiche eingetragen werden können. Außerdem sollten alternative Ansprechpartner:innen
kenntlich gemacht werden, um Abholtermine aufgrund von Krankheit oder Homeoffice nicht verschieben
zu müssen. Für Rückfragen zu einem Asset sind Kontaktinformation zu Ansprechpartner:innen
(Verleihende) sowie E-Mail-Adressen hinterlegt.

{\sffamily\color{maincolor}{Ft-VA-5 | Benachrichtigungen \& Erinnerungen }}\\
Verleihende werden nach dem Ausleihen eines von ihnen verantwortlichen Assets benachrichtigt. Die
Benachrichtigung umfasst, wer das Asset wann reserviert hat und wann die Abholung für
das Asset stattfinden soll. Außerdem wird es eine direkte Weiterleitungsmöglichkeit geben, sollten
Verleihende verhindert sein. Nachdem die Reservierung eines Assets stattgefunden hat, erhalten
Ausleihende eine Zusammenfassung über die Ausleihdaten und einen Hinweis, wann die Abholung
stattfindet. Außerdem werden Kontaktinformation des Verleihenden angezeigt (Name, E-Mail und
Lagerort des Assets). Zusätzlich erhalten Verleihende und Ausleihende eine Erinnerung, sobald die
Abholung oder Rückgabe eines Assets ansteht.


    {\sffamily\color{maincolor}{Ft-VA-6 | Material-Suche }}\\
Die Material-Suche umfasst eine Einteilung der Assets nach Kategorie, sowie die gezielte Suche nach
der Verfügbarkeit der Assets (verfügbar, Nicht verfügbar und Hinweis). Das gezielte Suchen nach
Verfügbarkeit wird durch die Aufforderung, den Ausleihzeitraum anzugeben, ermöglicht. Daraufhin gibt
es die Möglichkeit gewünschte Materialien in einem Suchfeld einzugeben oder über die Kategorien nach
dem Asset zu suchen. Das Suchfeld unterstützt bereits während der Eingabe mithilfe von
Assetvorschlägen. Die Vorschläge umfassen Assets direkt oder Kategorien. 

    {\sffamily\color{maincolor}{Ft-VA-7 | Filtern und Sortieren }}\\
Um das Finden der Assets leichter zu gestalten, sollen Nutzende stets nach Kategorie, Nutzen und
Verfügbarkeit filtern können. Außerdem ist das Sortieren von A-Z oder Z-A möglich. Um bei der
Anzeige einen Überblick über die Menge der Assets behalten zu können, wird die Anzahl der Assets in
der ausgewählten Kategorie angezeigt.

{\sffamily\color{maincolor}{Ft-VA-8 | Detailansicht }}\\
In der Detailansicht werden die Assets und deren Eigenschaften dargestellt. Hierbei werden
Informationen wie Name, Seriennummer, Artikelbeschreibung, technische Details und
Kontaktinformation der Verleihenden dargestellt. Des Weiteren kann der Ausleihzeitraum eingestellt
und die Zeiträume der Verfügbarkeit eines Assets eingesehen werden. In Form eines Button wird sichtbar, dass das
Asset zur Ausleihe hinzugefügt werden kann. Das Hinzufügen wird durch eine Pop-Up-Mitteilung
bestätigt.


\subsection{Funktionalitäten für Verleihende}
Im Folgenden werden Funktionalitäten, welche für Verleihende von Bedeutung sind näher erläutert
(\ref{table:ft-v}).

\begin{table}[h]
    \centering
    \caption{Funktionalitäten für (V)erleihenden }
    \begin{longtable}{lll}
        \arrayrulecolor{maincolor}\hline
        \sffamily\color{maincolor}ID & \sffamily\color{maincolor}Titel &
        \sffamily\color{maincolor}Anforderungen \\
        \arrayrulecolor{maincolor}\hline
        Ft-V-1                       & Dashboard               & ??? \\
        Ft-V-2                       & Bearbeiten des Assetstatus      & \anfref{F150} \\
        Ft-B-4                       & Kalenderansicht   für Verleihende                 &
        \anfref{V50} \anfref{Z30} \anfref{F40} \anfref{F50}                           \\
        Ft-V-3                       & Pflege von Assets               & \anfref{F130} \\
        Ft-V-4                       & Pflege der Datenbank            & \anfref{F140} \\
        \arrayrulecolor{maincolor}\hline
    \end{longtable}
    \label{table:ft-v}
\end{table}

{\sffamily\color{maincolor}{Ft-V-1 | Dashboard }}\\
Die Dashboard-Ansicht umfasst eine Aufgabenliste des ausgewählten Tages. Die Aufgaben umfassen
Abholungs- oder Rückgabeterminen sowie Wartungserinnerungen. Durch einer Wochenliste kann zwischen
den Tagen gewechselt werden. Außerdem werden die aktuellsten Benachrichtigungen angezeigt.

{\sffamily\color{maincolor}{Ft-V-2 | Bearbeiten des Assetstatus }}\\
Assets, welche an Ausleihende übergeben wurde, müssen zunächst manuell in ihrem Status bestätigt
werden. Hierfür ist eine simple Button-Funktion vorgesehen, sodass der Status des Assets schnell aktuell
gehalten werden kann. Außerdem soll das System, nachdem eine Abholung oder Rückgabe stattgefunden
hat, eine Benachrichtigung an Verleihende senden, sollte der Status nicht bereits bestätigt worden
sein.

{\sffamily\color{maincolor}{Ft-B-4 | Kalenderansicht für Verleihende}}\\
Die Kalenderansicht für Verleihende beinhaltet eine Übersicht, über alle Assets, welche gerade
Verliehen sind. Mithilfe einer Monatsübersicht werden Termin zur Asset-Abholung, Rückgabe oder zur
Wartung angezeigt. In einer Wochenansicht auf dem Dashboard gibt es eine detailreichere Ansicht
eines jeweiligen Tages.


    {\sffamily\color{maincolor}{Ft-V-3 | Pflege von Assets   }}\\
    \todo[]{baue ich eigentlich nict mit ein}
Mit Hilfe der Zuständigkeitsbereich \textit{(Ft-V-2)} kann die Pflege der Assets besser kontrolliert
werden. Außerdem können durch Erinnerungen und Checkliste Wartungen weniger leicht vergessen werden
und besser aufgeteilt werden. 


    {\sffamily\color{maincolor}{Ft-V-4 | Pflege der Datenbank }}\\
    Diese Funktionalität wird bereits durch SnipeIT bereitgestellt. Jedoch sollten Anleitungen
erstellt werden, um die Pflege zu erleichtern. 

\subsection{Funktionalitäten für (A)usleihende}
Im Folgenden werden Funktionalitäten, welche für Ausleihende von Bedeutung sind, näher erläutert
(\ref{table:ft-A}).

\begin{table}[h]
    \centering
    \caption{Funktionalitäten für (A)usleihende }
    \begin{longtable}{lll}
        \arrayrulecolor{maincolor}\hline
        \sffamily\color{maincolor}ID & \sffamily\color{maincolor}Titel &
        \sffamily\color{maincolor}Anforderungen \\
        \arrayrulecolor{maincolor}\hline
        Ft-A-1                       & Startübersicht                  & \anfref{F60} \\
        Ft-A-2                       & Reservierungs-Checkout          & \anfref{F60} \anfref{F150}
        \\

        Ft-A-3                       & Rückgabe-Checkliste             & ??? \\
        Ft-B-4                       & Kalenderansicht für Ausleihende                  &
        \anfref{V50} \anfref{Z30} \anfref{F40} \anfref{F50}                           \\
        \arrayrulecolor{maincolor}\hline
    \end{longtable}
    \label{table:ft-A}
\end{table}


{\sffamily\color{maincolor}{Ft-A-1 | Startübersicht }}\\
Die Startseite soll Nutzenden helfen, einen Überblick zu erlangen. Für Erstnutzende sind Hinweise
für die Materal-Suche  in Form von Buttons gegeben. Für Nutzende, die bereits Assets ausgeliehen
haben, wird eine Übersicht über laufende, kommende und vergangene Reservierungen angezeigt. Wichtige
Informationen, wie der Zeitraum, werden direkt auf einen Blick ersichtlich.

    {\sffamily\color{maincolor}{Ft-A-2 | Reservierungs-Checkout }}\\
Mithilfe des Reservierungs-Checkouts können alle ausgewählten Assets überblickt werden. Außerdem
werden alle Ausleihdaten, wie Zeitraum der Ausleihe, Abholung und Rückgabe aufgeführt. Des Weiteren
gibt es die Möglichkeit, alle Ausleihdaten bearbeiten zu können, sollte ein Datum oder eine Uhrzeit
unpassend sein. Abschließend werden die Regeln und Sicherheitshinweise aufgeführt. Nach Bestätigung
einer Erklärung zur Reservierung, gilt die Reservierung des Assets als abgeschlossen.

    {\sffamily\color{maincolor}{Ft-A-3 | Rückgabe-Checkliste}}\\
Bevor die ausgeliehenen Assets an Verleihende zurückgeben werden, wird eine Checkliste für das
jeweilige Asset angezeigt. Dort werden einige Hinweise angezeigt, wie zum Beispiel SD-Karte geleert, Assets auf
Ursprungseinstellungen zurückgestellt, Akkus geladen. Diese Funktionalität soll insbesondere dafür
sorgen, dass nachfolgende Ausleihende die Assets direkt nutzen können.

{\sffamily\color{maincolor}{Ft-B-4 | Kalenderansicht für Ausleihende}}\\
\todo[]{Mehr arbeit.. eher raus nehmen?}
Die Kalenderansicht für Ausleihende beinhaltet eine Übersicht, über alle Assets, welche gerade
ausgeliehen sind. Mithilfe einer Monatsübersicht werden Termin zur Asset-Abholung oder Rückgabe
angezeigt. 


\section{Systemarchitektur}
Die Systemarchitektur gibt eine Übersicht über die technische Umsetzung des Systems und
bildet die Basis der Realisierung von Software-Systemen \cite{dumke_software-metriken_2000}. Diese besteht im
Wesentlichen aus den folgenden drei Komponenten (\ref{fig:uml}): dem Snipe-IT Server, dem
Reservierungsinterface und dem Front-End.

Aufbauend auf den Anforderungen und der am \ac{imis} bereits eingesetzten Asset Managementsoftware
\textit{Snipe-IT} werden im Folgenden die gewählten Frameworks erläutert. Um ein besseres
Verständnis für die Architektur des entwickelten Software-Systemes voraussetzen zu können, wird nach
dem C4-Modell zur Visualisierung von Softwarearchitektur, der Aufbau der Architektur mit den
Teilkomponenten detaillierter dargestellt \cite{brown2013software}. 

\subsection{Frameworkwahl}
Die Frameworkwahl nimmt, durch die unterschiedlichen Arbeitsweisen und Funktionen der Frameworks,
enormen Einfluss auf den Entwurf eines Systems und wird daher im Folgenden näher erläutert. 
 
Die Grundlage der Auswahl, der im Rahmen dieser Arbeit eingesetzten Frameworks, bilden die eingangs
beschriebenen Anforderungen (\ref{section:anforderung}). Dem System wird vorausgesetzt, dass es sich
um eine Web-Anwendung mit Fokus auf den Einsatz im mobilen Kontext (\anfref{R10}\anfref{R40}) handelt. Für
Nutzende ist es wichtig, dass das System dauerhaft erreichbar ist (\anfref{R50}). Aus funktionaler
Sicht müssen die Frameworks eine Unterstützung für progressive Web-Applikationen bieten. Folglich
ist auch eine Unterstützung für HTTPS notwendig (\anfref{Q50}). Außerdem sollte es einfache
Möglichkeiten zur Verknüpfung von LDAP bieten (\anfref{K10} \anfref{F90}).


\subsection{C4-Modell zur Visualisierung von Softwarearchitektur}
Das Modell unterteilt die Architektur in vier Abstraktionsebenen (Level): \textbf{Context},
\textbf{Container}, \textbf{Components} und \textbf{Code}. Wobei im folgenden ausschließlich
auf die ersten drei Ebenen eingegangen wird. Des Weiteren werden die Bestandteile des
Systems in vier Kategorien gegliedert:

\begin{itemize}
    \item \textbf{Person}: Stellt Nutzende eines Software-Systems dar.
    \item \textbf{Software-System}: Stellt die höchste Abstraktionsebene der Software dar.
    \item \textbf{Container}: Stellt einzeln, ausführbare Teilkomponenten des Software Systems dar. 
    \item \textbf{Component}: Stellt z.B. die Datenbank oder Awendung dar.
\end{itemize}

{\sffamily\color{maincolor}{Level 1: Context}} 

Das erste Level der Architekturvisualisierung nach dem C4-Modell stellt mithilfe eines
Systemkontextdiagramms die entwickelte Software dar. \ref{fig:level1} zeigt die zwei Komponenten
des im Rahmen der Arbeit entwickelten Systems im Zusammenhang mit der Asset Managementsoftware
\textit{Snipe-IT}.

\begin{figure}[h]
    \centering
    \includegraphics[scale=0.4]{Bilder/level1.pdf}
    \caption[Architektur Level 1]{Architektur Level 1 nach \citeA{brown2013software}}
    \label{fig:level1}
\end{figure}

Der Kontext umfasst zwei Software-Systeme: das Reservierungstool und die Asset Managementsoftware
Snipe-IT. Das Reservierungstool, bildet die Oberfläche für Ausleihende und ermöglicht es, die Assets
einzusehen, zu suchen und zu buchen. Außerdem laufen über die Web-Oberfläche alle administrativen
Aufgaben für Verleihende, wie das Aktualisieren eines Assetstatus. Hierzu nutzt die Web-App das
Reservierungsinterface, um auf die SnipeIT Funktionalitäten zuzugreifen.  
\todo[inline]{Referenzierungen hinzufügen}
Die Basis für das in dieser Arbeit umgesetzte System schafft die Asset Managementsoftware
\textit{Snipe-IT} \cite{noauthor_home_nodate}, welche bereits am \ac{imis} eingesetzt wird.
\textit{Snipe-IT} ist eine kostenlose, quelloffene IT-Asset-Verwaltungs-Plattform, welche das
Nachverfolgen von Software-Lizenzen, Hardware und Verbrauchsgegenständen ermöglicht. Genannte Assets
können über ein Dashboard hinzugefügt, verwaltet und gelöscht werden. Über Labels können Assets zur
Übersichtlichkeit in verschiedene Kategorien eingeordnet werden, während Tags ein Asset eindeutig
identifizieren (z. B. Seriennummer). Zudem ermöglicht das „Checkin/Checkout“-System die
Nachverfolgung aller Assets, falls diese zum Beispiel an Person ausgeliehen werden. Zu jedem
Zeitpunkt kann ein Asset maximal einer Person zugeordnet werden, wodurch das mehrfache gleichzeitige
Ausleihen eines Assets verhindert wird. Darüber hinaus beschreiben Status-Label den Zustands eines
Assets und ob dieses ausgeliehen werden kann. Alle Funktionalitäten können zudem über eine REST-API
programmatisch genutzt werden. Des Weiteren verfügt \textit{Snipe-It} über eine Schnittstelle,
welche die Integration von LDAP stark vereinfacht.

{\sffamily\color{maincolor}{Level 2: Container}}

Im zweiten Level werden die Container des Software-Systems gezeigt. Hierbei werden Verantwortlichkeiten und
Kommunikation zwischen den Bestandteilen des Software-Systems dargestellt (Brown, 2021). \ref{fig:level2}

\begin{figure}[h]
    \centering
    \includegraphics[scale=0.47]{Bilder/level2.pdf}
    \caption[Architektur Level 2]{Architektur Level 2 nach \citeA{brown2013software}}
    \label{fig:level2}
\end{figure}

Für Grundlage des Frontends wird Vue.js\footnote{\url{https://vuejs.org/}} verwendet. Vue.js ist
eine progressive JavaScript Framework. Bei der Nutzung von Vue
CLI\footnote{\url{https://cli.vuejs.org/}} kann eine PWA-Funktionalität mithilfe des
\inlinecode{@vue/cli- plugin-pwa}-Pakets schnell eingebunden werden. Zudem steigt das Framework in
der Zufriedenheit und Verwendung laut \citeA{noauthor_state} im Vergleich zu Angular über die Jahre
stetig an (\ref{fig:vue}). Außerdem zeigt Vue.js in den Leistungs-Benchmarks positive Merkmale
\todo[]{(Krause, 2022)}. Zusätzlich zu den bereits aufgeführten Eigenschaften wird Vue.js aufgrund der
begrenzten Implementierungszeit und bestehende Erfahrung gewählt.

Das Reservierungsinterface nutzt die von Snipe-IT bereitgestellten Daten. Die Hauptaufgabe der
Schnittstelle ist das Reservieren von Assets in die Zukunft. Da Snipe-IT selbst die zukünftige
Reservierung nicht unterstützt, werden diese Reservierungen stattdessen im Reservierungsinterface
gespeichert. Die gesamte Kommunikation zwischen Frontend und SnipeIT wird somit über das
Reservierungsinterface stattfinden, um auch alle zukünftigen Reservierungen zu berücksichtigen. Die
Datenbank und API von Snipe-IT bilden die Ausgangsposition des Systems. Folglich wird die Verwaltung
und Speicherung der vorhandenen Assets über Snipe-IT direkt abgewickelt. Über die bereitgestellte
API werden die gespeicherten Daten für alle weiteren Komponenten bereitgestellt. Durch die
eingeschränkten Funktionen der Kalender-Funktion und die fehlende Möglichkeit Termine in der Zukunft
auswählen zu können, wird das Reservierungsinterface benötigt.

Um unteranderem ein besseres Verständnis der Interaktion von einzelnen Komponenten voraussetzen zu können wurde
ein UML-Sequenzdiagramm erstellt (\ref{fig:uml}). Bei der vierten Komponente handelt es sich um das
LDAP-System der Universität zu Lübeck. Mithilfe der Integration können Nutzende sich mit ihrem bereits
bestehenden IDM Account anmelden. Zum einen kann somit sichergestellt werden, dass nur befugte
Personen Assets einsehen und ausleihen können. Zum anderen erleichtert es den Nutzenden den
Ausleihprozess, da kein neuer Account erstellt werden muss.

\begin{figure}[h]
    \centering
    \includegraphics[scale=0.45]{Bilder/uml.pdf}
    \caption[UML-Sequenzdiagramm]{UML-Sequenzdiagramm}
    \label{fig:uml}
\end{figure}

{\sffamily\color{maincolor}{Level 3: Components}}

Das dritte Level stellt die Container aus Level 2 genauer dar, um die elementaren strukturellen
Bestandteile und Wechselwirkungen zwischen diesen aufzuzeigen (Brown, 2021). Im Folgenden werden die
Container Web-Application und Web Server genauer betrachtet \ref{fig:level3}.

\begin{figure}[h]
    \centering
    \includegraphics[scale=0.9]{Bilder/level3.pdf}
    \caption[Architektur Level 3]{Architektur Level 3 nach \citeA{brown2013software}}
    \label{fig:level3}
\end{figure}

Für das Reservierungsinterface wird ein Serverframework und eine Speichermöglichkeit in Form einer
Datenbank benötigt. Aufgrund der ausgeprägten
Pluginauswahl\footnote{\url{https://www.fastify.io/ecosystem/}} und breiten
Nutzung\footnote{\url{https://www.fastify.io/organisations/}} wird
Fastify\footnote{\url{https://www.fastify.io/}} als Serverframework eingesetzt. Als relationale
Datenbank wird die quelloffene Software SQLite\footnote{\url{https://www.sqlite.org/index.html}}
eingesetzt. Um den Zugriff und die Verwaltung der Daten zu vereinfachen wird zudem die
ORM-Bibliothek Prisma verwendet.


