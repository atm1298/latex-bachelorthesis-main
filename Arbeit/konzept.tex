%!TEX root = thesis.tex

\chapter{Konzept}
\label{chapter-konzept}

In diesem Kapitel wird die eigentliche Erkenntnis dieser Arbeit beschrieben. Der Aufbau dieses
Kapitels hängt stark vom Thema der Arbeit ab. Die in dieser Vorlage vorgeschlagenen Kapitel sind
auch nur als Vorschlag und auf keinen Fall als verbindlich zu verstehen.

\section{Funktionalität}




\section{Frameworks} 
Ziel dieser Arbeit ist es, abgeleitet aus den eingangs beschriebenen Anforderungen, ein wirksames
Reservierungstool für Assets zu konzipieren, implementieren und evaluieren. Die Basis für ein
solches Tool schafft die Asset Managementsoftware \textit{SnipeIT} \cite{noauthor_home_nodate},
welche bereits am \ac{imis} eingesetzt wird.

Snipe-IT ist eine kostenlose, quelloffene IT-Asset-Verwaltungs-Plattform,
welche das Nachverfolgen von Software-Lizenzen, Hardware und
Verbrauchsgegenständen ermöglicht. Genannte Assets können über ein Dashboard
hinzugefügt, verwaltet und gelöscht werden. Über Labels können Assets zur
Übersichtlichkeit in verschiedene Kategorien eingeordnet werden, während
Tags ein Asset eindeutig identifizieren (z. B. Seriennummer). Zudem ermöglicht
das „Checkin/Checkout“-System die Nachverfolgung aller Assets, falls diese
z. B.  an Person ausgeliehen werden. Zu jedem Zeitpunkt kann ein Asset
maximal einer Person zugeordnet werden, wodurch das mehrfache gleichzeitige
Ausleihen eines Assets verhindert wird. Darüber hinaus beschreiben Status-Label
den Zustands eines Assets und ob dieses ausgeliehen werden kann. Alle
Funktionalitäten können zudem über eine REST-API programmatisch genutzt werden.


\section{Systemarchitektur}

\section{Interface-Design}

\subsection*{Low-Fidelity-Prototyp}

\subsection*{High-Fidelity-Prototyp}