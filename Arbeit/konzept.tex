%Wie eingangs er-wähnt, definieren die Anforderungen, was das System zu
%leisten hat, während die Funktionalitä-ten definieren, wie das System diese gewährleistet.
\chapter{Konzept}
\label{chapter-konzept}

In diesem Kapitel wird die eigentliche Erkenntnis dieser Arbeit beschrieben. Der Aufbau dieses
Kapitels hängt stark vom Thema der Arbeit ab. Die in dieser Vorlage vorgeschlagenen Kapitel sind
auch nur als Vorschlag und auf keinen Fall als verbindlich zu verstehen.

\section{Funktionalität}
% Funktionalitäten im Bereich der Vorbereitung
\subsection{Funktionalitäten im Bereich der Vorbereitung}
Um ein Asset ausleihen zu können, müssen bestimmte Vorbereitungen getroffen werden, diese werden im
Folgenden näher erläutert.

\begin{table}[h]
    \centering
    \caption{Funktionalitäten im Bereich der Vorbereitung}
    \begin{tabular}{lll}
            \arrayrulecolor{maincolor}\hline
            \sffamily\color{maincolor}ID & \sffamily\color{maincolor}Titel &
            \sffamily\color{maincolor}Anforderungen \\
            \arrayrulecolor{maincolor}\hline
            Ft-Vt-1 & Übersicht über ausleihbare Assets                         & ?? \\
            Ft-Vt-2 & Dialog für Vorschläge (Use Case)                          & ?? \\
            Ft-Vt-3 & Verfügbarkeit von Assets einsehen                         & ?? \\
            Ft-Vt-4 & Reservierungen von Assets                                 & ?? \\
            Ft-Vt-5 & Kalenderübersicht                                         & ?? \\
            Ft-Vt-6 & Sichtbarkeit von Ansprechpartner:innen                    & ?? \\
            \arrayrulecolor{maincolor}\hline
    \end{tabular}
    \label{table:ft-vt}
\end{table}

\label{subsubsection:Ag-Vt-1}
Um ein Asset ausleihen zu können, muss eine Anfrage an die verantwortlichen gesendet werden, dies
geschieht meist per E-Mail. Ausleihende fragen, aufgrund des mangelnden Wissen, nach einem direkten
Asset, welches über bspw. den Flurfunk an diese gelang. Wie bereits in der Problemanalyse geschildet
(\ref{section:probleme-allgemein}) gibt es keine Übersicht, über ausleihbare Assets. Dies zeigt die
Dringlichkeit des SnipeIT Companion für eine besser Vorbereitung. So können Ausleihende mit mehr
Wissen eine Anfrage zum Ausleihen stellen. 

Die zuvor erhaltene Anfrage aus \nameref{subsubsection:Ag-Vt-1} wird von Verleihenden verarbeitet.
Um den Ausleihprozess für Verleihende zu erleichtern, kann der vorangestellt Use Case mittels eines
Dialogs ermöglicht werde. So können aufbauend auf dem Dialog, direkt Vorschläge an Ausleihende gegeben
werden.

Um ein Assets ausleihen zu können ist es sowohl für Verleihende als auch Ausleihende von bedeutung,
ob Assets Verfügbar sind

Durch die Verfügbarkeit können wir Reservieren

Anknüpfend an Ag-Vt-3 und Ag-Vt-4 ist eine ... sinnvoll.

Nicht ersichtlich aktuell, Fragen im Voraus 

% Funktionalitäten der Ausgabe 
\subsection{Funktionalitäten im Bereich der Ausgabe}
Im nächsten Abschnitt werden alle Zentralen Funktionalitäten aufgeführt, welche für die Übergabe von
Verleihnden zu Ausleihnden umfasst.
\begin{table}[h]
    \centering
    \caption{Funktionalitäten im Bereich der Ausgabe}
    \begin{tabular}{lll}
            \arrayrulecolor{maincolor}\hline
            \sffamily\color{maincolor}ID & \sffamily\color{maincolor}Titel &
            \sffamily\color{maincolor}Anforderungen \\
            \arrayrulecolor{maincolor}\hline
            Ft-Au-1 & Zugang zu Assets                              & ?? \\
            Ft-Au-2 & Kommunikationsmöglichkeit bieten              & ?? \\
            Ft-Au-3 & Vermittlung/Nutzung von Assets erläutern      & ?? \\
            Ft-Au-4 & Ausgeliehene Assets dokumentieren             & ?? \\
            Ft-Au-5 & Rechtliche Rahmenbedingungen                  & ?? \\
            Ft-Au-6 & Sichtbarkeit von Ansprechpartner:innen        & ?? \\
            \arrayrulecolor{maincolor}\hline
    \end{tabular}
    \label{table:ft-Au}
\end{table}


Damit Ausleihen die gewünschten Assets aus den Laboren entnehmen können, müssen Verleihende die
Entsprechenden Räumlichkeiten aufschließen. Hierbei ist zu berücksichtigen, dass die
Forschungsgruppe untereinander auch nicht immer Zugriff haben (Verweis:Ansprechpartner Ag-Vt-6).


Ausleihene wissen vorher häufig nicht genau um was für ein Gerät es genau handelt, so kann es zu
kompatiblitätsfehlern kommen und das Asset ist für den Usprübglichen Gebauch nicht nutzbar. Daher
ist eine Übersicht mit Informationen wie: Seriennummer,... sowie der dazugehörigen Anleitung
sinnvoll.

?

Durch die uneinheitlichen Vorgänge, sind die Verantwortlichkeiten häufig nicht immer klar.
(Forschungsgruppe..) Daher... Da die Assets nicht versichert sind, läuft das Ausleihen auf
Vertrauenbasis. Sollten Krasser oder weitere Schäden entstehen vereinbaren Nutzenden somit, diese
direkt zu melden.

% Funktionalitäten der Rückgabe
\subsection{Funktionalitäten im Bereich der Rückgabe}
Die Nachfolgenden Funktionalitäten umfassen die Rückgabe der Ausgeliehenen Assets.
\begin{table}[h]
    \centering
    \caption{Funktionalitäten im Bereich der Rückgabe}
    \begin{tabular}{lll}
            \arrayrulecolor{maincolor}\hline
            \sffamily\color{maincolor}ID & \sffamily\color{maincolor}Titel &
            \sffamily\color{maincolor}Anforderungen \\
            \arrayrulecolor{maincolor}\hline
            Ft-Wt-1 & Erinnerungen erhalten                     & ?? \\
            Ft-Wt-2 & Assets korrekt zurückgebracht             & ?? \\
            \arrayrulecolor{maincolor}\hline
    \end{tabular}
    \label{table:Ft-Rg}
\end{table}

Kann ja vorkommen, dass mal was vergessen wird...

Akkus voll, SD karte leeren, Ursprüngliche Einstellungen,...

% Funktionalitäten der Wartung
\subsection{Funktionalitäten im Bereich der Wartung}
\label{subsec:ft-wartung}
Im Folgenden werden Funktionalitäten, welche für Verleihende auf Administariver Ebene von Bedeutung sind
näher erläutert. 
\begin{table}[h]
    \centering
    \caption{Funktionalitäten im Bereich der Wartung}
    \begin{tabular}{lll}
            \arrayrulecolor{maincolor}\hline
            \sffamily\color{maincolor}ID & \sffamily\color{maincolor}Titel &
            \sffamily\color{maincolor}Anforderungen \\
            \arrayrulecolor{maincolor}\hline
            Ft-Wt-1 & Pflege von Assets                  & ?? \\
            Ft-Wt-2 & Pflege der Datenbank/Übersicht/Liste            & ?? \\
            \arrayrulecolor{maincolor}\hline
    \end{tabular}
    \label{table:Ft-Wt}
\end{table}

Updates nach langer Nicht Nutzung, Akuus entladen,...

Bei Neuanschaffung, keinen Überblick über die Erneuerungen, ...


\section{Frameworks} 

\subsection{Relevante Anforderung an ein Framework}
\subsection{Wahl des genutzten Frameworks}
Ziel dieser Arbeit ist es, abgeleitet aus den eingangs beschriebenen Anforderungen, ein wirksames
Reservierungstool für Assets zu konzipieren, implementieren und evaluieren. Die Basis für ein
solches Tool schafft die Asset Managementsoftware \textit{SnipeIT} \cite{noauthor_home_nodate},
welche bereits am \ac{imis} eingesetzt wird.

Snipe-IT ist eine kostenlose, quelloffene IT-Asset-Verwaltungs-Plattform,
welche das Nachverfolgen von Software-Lizenzen, Hardware und
Verbrauchsgegenständen ermöglicht. Genannte Assets können über ein Dashboard
hinzugefügt, verwaltet und gelöscht werden. Über Labels können Assets zur
Übersichtlichkeit in verschiedene Kategorien eingeordnet werden, während
Tags ein Asset eindeutig identifizieren (z. B. Seriennummer). Zudem ermöglicht
das „Checkin/Checkout“-System die Nachverfolgung aller Assets, falls diese
z. B.  an Person ausgeliehen werden. Zu jedem Zeitpunkt kann ein Asset
maximal einer Person zugeordnet werden, wodurch das mehrfache gleichzeitige
Ausleihen eines Assets verhindert wird. Darüber hinaus beschreiben Status-Label
den Zustands eines Assets und ob dieses ausgeliehen werden kann. Alle
Funktionalitäten können zudem über eine REST-API programmatisch genutzt werden.


\section{Systemarchitektur}

\section{Interface-Design}

\subsection{Low-Fidelity-Prototyp}

\subsection{High-Fidelity-Prototyp}