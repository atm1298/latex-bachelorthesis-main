%!TEX root = thesis.tex

\chapter{Forschungsziele}

% Die Einleitung führt zum eigentlichen Thema dieser Arbeit hin. Dabei wird ein großer Bogen gespannt, in dem die Relevanz und der Kontext der untersuchten Thematik deutlich wird. Grundlegende Begriffe aus dem Titel und der Kurzfassung sollten aufgegriffen und definiert werden. Unterstützend können Zitate herangezogen werden, die der Arbeit einen Rahmen geben.

\section{Ziel der Arbeit}
Das Ziel dieser Arbeit ist es ein System zu entwickeln, welches den Reservierungs- und Entleihprozess einheitlicher, effizienter und zufriedenstellender lösen lässt. Die Anwendung soll es unter anderem ermöglichen, die auszuleihenden Assets in einer Liste abzubilden und diese zu durchsuchen.
Hierbei ist auch das Anzeigen einer Stückzahl der jeweils verfügbaren Assets, deren Bedienungsanleitung sowie der verantwortliche Mitarbeitende denkbar. Des Weiteren kann für die Vorausplanung
eine Kalenderfunktion sinnvoll sein.

Ziel dieser Arbeit ist es, abgeleitet aus den eingangs beschriebenen Anforderungen, ein wirksames
Reservierungstool für die Hardware zu konzipieren, implementieren und evaluieren. Die Basis für
ein solches Tool schafft die Asset Managementsoftware \textit{SnipeIT} \cite{noauthor_home_nodate}, welche bereits am \ac{imis} eingesetzt wird.
Es sollen die folgenden Forschungsfragen beantwortet werden:



\section{Forschungsfragen}
\begin{enumerate}
  \item[\sffamily\color{maincolor} {F1}] {Welche zentralen Schwierigkeiten bringt die aktuelle Planung und Reservierung von Assets für Mitarbeitende und Studierende mit sich?}
  \item[\sffamily\color{maincolor} {F2}] {Was sind Anforderungen an ein System, welches die in F1 gezeigten Schwierigkeiten adressiert und reduziert?}
  \item[\sffamily\color{maincolor} {F3}] {Inwieweit kann ein aus F2 resultierender Prototyp die in F1 identifizierten Schwierigkeit reduzieren oder komplett minimieren?}
\end{enumerate}


\section{Vorgehensweise}
Die Entwicklung des Systems orientiert sich am menschenzentrierten Gestaltungsprozess \cite{din_en_iso_9421-2102020-03_din_nodate}. Der Prozess teilt sich im Rahmen dieser Arbeit, in fünf aufeinanderfolgende
Phasen (Abbildung 1), wobei die Entwurfs- und Implementierungsphasen Spielraum für ein iteratives
Vorgehen lassen. Unter anderem werden in der Analyse die Aufgaben des Systems, die Benutzenden
und der Kontext nach dem Entwicklungsprozess für interaktive Medien \cite{herczeg_einfuhrung_2009} aufgeführt,
um ein gebrauchstaugliches Ergebnis erzielen zu können.
\begin{description}
  \item[\ref{chapter-basics}] beschreibt die für diese Arbeit benötigten Grundlagen. In diesem Kapitel werden \ldots, \ldots und \ldots eingeführt, da diese für die folgenden Kapitel dringend benötigt werden.
  \item[\ref{chapter-konzept}] stellt das eigentliche Konzept vor. Dabei handelt es sich um ein Konzept zur Verbesserung der Welt. Das Kapitel gliedert sich daher in einen globalen und einen lokalen Ansatz, wie die Welt zum Besseren beeinflusst werden kann.
  \item[\ref{chapter-evaluation}] beinhaltet eine Evaluation des Konzeptes aus dem vorherigen Kapitel. Anhand von Simulationen wird in diesem Kapitel untersucht, wie die Welt durch konkrete Maßnahmen deutlich verbessert werden kann.
\end{description}

