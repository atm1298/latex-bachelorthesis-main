%!TEX root = thesis.tex

\chapter{Einleitung}
Das Ausleihen von Assets jeglicher Art ist keine Neuheit \cite{soderholm2018borrowing}. An der
Zentralen Hochschulbibliothek Lübeck (ZHB) wird hierzu beispielsweise die Buchungsapp
\textit{Affluences}\footnote{\url{https://affluences.com/?lang=de}} verwendet. Zum Ausleihen von
Materialien müssen Terminabholungen online gebucht werden \cite{zhb_offnung_nodate}. Die Anfragen
können überprüft werden, wodurch das vorausschauende Planen der Materialien ermöglicht wird. 

Am Institut für Multimediale und Interaktive Systeme (\ac{imis}) werden Assets ohne Anmeldung oder
mit einer mündlichen Absprache verwendet. Aufseiten der Mitarbeitenden wird der Gebrauch der Assets
individuell geplant, sodass das frühzeitige Reservieren oder das geplante Ausleihen erschwert wird.
Für die vorausschauende Planung von anstehenden Projekten gibt es keine feste Reservierung oder
einen Überblick, wann die gewünschten Assets wieder verfügbar sind. Zudem ist vielen Mitarbeitenden
unklar, welche Assets sich in den Laboren des \ac{imis} befinden. Folglich kennen Mitarbeitende nur
selten die Möglichkeiten, mit denen sie ihre Forschungsprojekte oder die Lehre ergänzen könnten.

\todo{Neu: Kommender Absatz überarbeitet}
Aktuell werden Reservierungen sowie der Gebrauch von Equipment über unterschiedliche
Kommunikationswege wie E-Mail oder mündlicher Absprache bei zuständigen Mitarbeitenden angefragt.
Die zuständigen Mitarbeitenden prüfen die Anfrage und koordinieren potenzielle Kollisionen mit
bereits reservierten Zeiträumen oder Absprachen und bestätigen, ändern oder lehnen diese
(Reservierungs-)Anfrage ab. In einigen Fällen werden die gebuchten Zeiten auf Papier dokumentiert.
Aufseiten der Studierenden kann ein Reservierungstool für Projekte, wie zum Beispiel die des Moduls
\ac{emi} hilfreich sein. Studierende müssen während dieses Projekts eine multimediale Abgabe
produzieren, welche beispielsweise die Form eines Videos haben kann.\\ Aktuell ist vielen
Studierenden nicht bewusst, dass das \ac{imis} Equipment wie Kameras, Gimbals oder VR-Equipment für
Videos oder Projektaufgaben bereitstellt. Des Weiteren sind Hemmschwelle und Aufwand zum Ausleihen
der Hardware hoch, da uneindeutig ist, welche Mitarbeitenden für die jeweilige Hardware zuständig
sind.

\section{Ziel der Arbeit}
Das Ziel dieser Arbeit ist es ein wirksames System zu entwickeln, welches den Reservierungs- und
Ausleihprozess am \ac{imis} einheitlicher, effizienter und zufriedenstellender lösen lässt. Die
Anwendung soll es unter anderem ermöglichen, die auszuleihenden Assets in einer Liste abzubilden und
diese zu durchsuchen. In diesem Zusammenhang soll auch das Anzeigen einer Stückzahl der jeweils
verfügbaren Assets, deren Bedienungsanleitung sowie die verantwortlichen Mitarbeitenden umgesetzt
werden. Das Reservierungstool soll eine niedrigschwellige Möglichkeit bieten, um Mitarbeitenden die
Arbeit im Reservierungsprozess zu erleichtern. Außerdem soll es eine Übersicht über Assets
geben, sodass Mitarbeitende nicht mehr im Unklaren über die Hardware-Möglichkeiten sind und das
Material optimal genutzt werden kann.

Die Basis für ein solches Tool schafft die Asset-Managementsoftware \textit{Snipe-IT}
\cite{noauthor_home_nodate}, welche bereits am \ac{imis} eingesetzt wird. \textit{Snipe-IT} ist eine
kostenlose, quelloffene IT-Asset-Verwaltungs-Plattform, welche das Nachverfolgen von
Software-Lizenzen, Hardware und Verbrauchsgegenständen ermöglicht. Genannte Assets können über ein
Dashboard hinzugefügt, verwaltet und gelöscht werden. Über Labels können Assets zur
Übersichtlichkeit in verschiedene Kategorien eingeordnet werden, während Tags ein Asset eindeutig
identifizieren (z. B. Seriennummer). Zudem ermöglicht das „Checkin/Checkout“-System die
Nachverfolgung aller Assets, falls diese zum Beispiel an eine Person ausgeliehen werden. Zu jedem
Zeitpunkt kann ein Asset maximal einer Person zugeordnet werden, wodurch das mehrfache gleichzeitige
Ausleihen eines Assets verhindert wird. Da \textit{Snipe-IT} selbst die zukünftige Reservierung
nicht unterstützt, umfasst das Ziel der Arbeit einen \textit{Snipe-IT Companion}, welcher das
Ausleihen in die Zukunft möglich macht.


\section{Forschungsfragen}
\label{sec:Forschungsfragen}
Im Sinne der eingangs beschriebenen Ziele soll im Rahmen der vorliegenden Arbeit untersucht werden,
wie ein Reservierungstool gestaltet werden kann, um Mitarbeitende und Studierende darin zu
unterstützen, Assets einzusehen und ausleihen zu können.

Um den aktuellen Stand und die Probleme des aktuellen Ausleihprozess nachvollziehen zu können,
müssen diese zunächst ermittelt und klassifiziert werden. Die erste Forschungsfrage beschäftigt sich
daher mit der Analyse der Schwierigkeit des Ausleihprozesses.
\begin{enumerate}
  \item[\sffamily\color{maincolor} {F1 |}] {Welche zentralen Schwierigkeiten bringt die aktuelle Planung und Reservierung von Assets für Mitarbeitende und Studierende mit sich?}
\end{enumerate}
Um die zentralen Schwierigkeiten und Probleme lösen zu können, müssen Anforderungen an das System
ermittelt werden. Anschließend müssen diese nach Relevanz sortiert werden, um möglichst viele
Schwierigkeiten lösen zu können. Die zweite Forschungsfrage beschäftigt sich folglich mit den
Anforderungen, welche ein solches Reservierungstool umfassen sollte, um die Schwierigkeiten zu minimieren.
\begin{enumerate}
  \item[\sffamily\color{maincolor} {F2 |}] {Was sind Anforderungen an ein System, welches die in F1 gezeigten Schwierigkeiten adressiert und reduziert?}
\end{enumerate}
Abschließend soll eruiert werden, ob die erste Iteration des Systems die ermittelten Schwierigkeiten
mit den erarbeiteten Funktionalitäten lösen kann. Die letzte Forschungsfrage beschäftigt sich
demzufolge mit der Evaluation des erarbeiteten Systems und den möglichen Stärken und Schwächen.
\begin{enumerate}
  \item[\sffamily\color{maincolor} {F3 |}] {Inwieweit kann ein aus F2 resultierender Prototyp die in F1 identifizierten Schwierigkeiten reduzieren?}
\end{enumerate}

\section{Vorgehensweise}
Die Entwicklung des Systems orientiert sich am menschenzentrierten Gestaltungsprozess
\cite{DINISO9241}. Der Prozess teilt sich im Rahmen dieser Arbeit in
fünf aufeinanderfolgende Phasen (\ref{fig:vorgehen}), wobei die Entwurfs- und Implementierungsphasen
Spielraum für ein iteratives Vorgehen lassen. Unter anderem werden in der Analyse die Aufgaben des
Systems, die Nutzenden und der Kontext nach dem Entwicklungsprozess für interaktive Medien
aufgeführt, um ein gebrauchstaugliches Ergebnis erzielen zu können \cite{HerczegMDI2009}.

\begin{figure}[h]
  \centering
  \includegraphics[scale=0.156]{Bilder/Screenshot 2022-10-17 at 18-39-40 Vorgehensmodell.pptx.png}
  \label{fig:vorgehen}
  \caption[Vorgehensmodell]{Vorgehensmodell}
\end{figure}

In der ersten Phase werden die Anforderungen an das zu entwickelnde System erarbeitet und
verstanden. Die Erkenntnisse dieser Phase sind in \ref{chapter-analyse} zu finden.
\ref{chapter-analyse} soll unter anderem die Forschungsfragen F1 und F2 beantworten. Durch
semi-strukturierte Interviews mit Stakeholdern sollen die zentralen Schwierigkeiten und Aufgaben des
Ausleihprozesses festgestellt werden. Folglich werden die Notwendigkeit der angedachten
Anforderungen mithilfe der Interviews überprüft und ergänzt. Außerdem wurde eine Recherchiere
vergleichbarer Systeme durchgeführt, um die Aussagekraft der ermittelten Interviewergebnisse zu
überprüfen.

\todo{Neu: Kommender Absatz überarbeitet}
In der Spezifikationsphase werden die Anforderungen an das System weiter spezialisiert
(\ref{chapter-konzept}). Daher werden Funktionalitäten entsprechend den Anforderungen entwickelt und
in einer priorisierten Feature-Liste festgehalten. Anschließend wird die Systemarchitektur,
aufgeteilt in Frontend, Reservierungsinterface (Backend) und dem bestehenden Snipe-IT Server,
mithilfe des C4-Models erarbeitet. Aufbauend darauf werden passende Frameworks zur Entwicklung
ausgewählt. Darüber hinaus wird das Interface-Design erarbeitet (\ref{chapter-design}). Hierbei wird
durch Usability Tests ein iteratives Vorgehen ermöglicht.
 

Die Implementierungsphase umfasst die eigentliche Umsetzung des Reservierungstools
(\ref{chapter-implementierung}). Hierbei werden die in der Konzeptionsphase festgelegten Frameworks
genutzt. \ref{chapter-dialogbeispiel} präsentiert das realisierte System anhand von
Dialogbeispielen.

In der abschließenden Phase wird das realisierte System mithilfe von Interviews und Umfragen
evaluiert (\ref{chapter-evaluation}). Die Ergebnisse der Phase beantworten Forschungsfrage F3 und
geben Aufschluss über die Wirksamkeit des entwickelten Systems. Im Anschluss gibt
\ref{chapter-fazit} einige Perspektiven über offene Punkte und die mögliche Weiterentwicklung des
Systems.