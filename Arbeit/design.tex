%Wie eingangs er-wähnt, definieren die Anforderungen, was das System zu
%leisten hat, während die Funktionalitä-ten definieren, wie das System diese gewährleistet.
\chapter{Interface-Design}
\label{chapter-design}
Im Folgenden wurde sich an dem Designprozess Scenario Based Design für die Entwicklung interaktiver
Systeme orientiert. Mithilfe der in \refname{section:iststand} geschilderten Schwierigkeiten konnten
einzelne Szenarien erarbeitet und in Formen einer Vision ausgehend von der aktuellen Situation
formuliert werden \refname{section:funktionale}.

Während des Interface-Designs wurden aus Szenarien Mockups und High-Fidelity-Prototypen entwickelt.
Die Prototypen wurden in Form einer Usability Evaluation direkt getestet: Für die Tests wurden
Usability Spezifikationen aus Szenarien abgeleitet.

Die Szenarien beschreiben Aufgaben die als Vorlage für die Test gestaltung dienen können. Die
Ergebnisse aus den Usability Evaluationen fließen in die weitere Entwicklung mit ein.

Außerdem wurden die Usability Heuristiken nach \citeA{experience_10_nodate} während des gesamten
Prozesses berücksichtigt, um Gebrauchstauglichkeit voraussetzen zu können.

Eine formative Evaluation des Interface-Designs ist wichtig, um früh die Gebrauchstauglichkeit,
die Einfachheit der Bedienung und die Akzeptanz der ästhetischen Gestaltung zu etablieren.



\section{Mockups}

\section{High-Fidelity-Interface-Design}
\subsection{Designsprache}

\subsection{Evaluation der ersten Entwurfs-Iteration}


\begin{table}[h]
    \centering
    \caption{Teilnehmende der Evaluation}
    \begin{tabular}{lll}
            \arrayrulecolor{maincolor}\hline
            \sffamily\color{maincolor}ID & \sffamily\color{maincolor}Alter &
            \sffamily\color{maincolor}Rolle \\
            \arrayrulecolor{maincolor}\hline
            E1                           & 19 - 25 J.                      &
            Medieninformatikerinnen, Hilfswissenschaftlerin                        \\
            E2                           & 19 - 25 J.                      & Roboterinnen \\
            E3                           & 19 - 25 J.                      & ??         \\
            E4                          & 19 - 25 J.                      & ?? \\
            \arrayrulecolor{maincolor}\hline
    \end{tabular}
    \label{table:e}
\end{table}

\todo[inline]{Kriteriensuchen, mehrere Suchen, Wording ändern ->  Eindeutigkeit}

\todo[inline]{Assistent -> Wording: Auswahlhilfe, Suchhilfe, Kriteriensuche, Kiterienhilfe, Auwahl nach Kriterien, Ausleihhilfe
Material statt Assets}

\todo[inline]{Sauber, Clean und nicht überladen}

\todo[inline]{Bei Asset-Suche: Strich bei Kalender: Drawer
Suche und “Kategorien durchstöbern” }

