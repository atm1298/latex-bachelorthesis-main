%!TEX root = thesis.tex

\chapter{Evaluation}
\label{chapter-evaluation}
Im Rahmen der Forschungsfrage F3 soll herausgearbeitet werden, inwiefern die Beeinflussung des XXXX
durch das in der vorliegenden Arbeit entwickelte Reservierungssystem, innerhalb einer Studie
untersucht werden kann. Da es sich um ein universitätsinternes Tool handelt, wurde sich bei den
Versuchspersonen, im Rahmen eines Proof of Concepts (N=5) mit Mitarbeitenden des \ac{imis} und
Studierenden im Bereich der Medieninformatik zusammengesetzt, um das konzipierte Studiendesign
zu evaluieren. Die Evaluation wird zusätzlich dafür genutzt, das entwickelte Reservierungssystem zur
Untersuchung von Forschungsfrage F2 hinsichtlich Funktionen, Gestaltung und
Nachvollziehbarkeit bewerten zu lassen.


\section{Vorgehen und Methodik}
- Wording mit Abfragen


Evaluationsaufgaben definieren, die die Teilnehmer:innen Schritt für Schritt durchführen sollten.
Dabei sollten diese Think-Aloud anwenden und immer sagen, wenn sie mit einer Aufgabe fertig sind.
Dabei Notizen machen (z. B. wenn ein Problem auftritt, was unklar war, was als verständlich oder gut
angemerkt wurde etc.).


\section{Verleihende}

\section{Ausleihende}


\section{Diskussion}