%!TEX root = thesis.tex

\chapter{Evaluation}
\label{chapter-evaluation}
Im Rahmen der Forschungsfrage F3 soll herausgearbeitet werden, inwiefern die
Gebrauchstauglichkeit und Nützlichkeit durch das in der vorliegenden Arbeit
entwickelte Reservierungssystem gewährleistet werden kann. Da es sich um ein
universitätsinternes Tool handelt, wurde sich bei den Versuchspersonen, im
Rahmen des Prototypen (N=6) mit Mitarbeitenden des \ac{imis} und Studierenden im
Bereich der Medieninformatik zusammengesetzt, um das konzipierte Studiendesign
zu evaluieren. Die Evaluation wird zusätzlich dafür genutzt, das entwickelte
Reservierungssystem zur Untersuchung von Forschungsfrage F2 hinsichtlich
Funktionen, Gestaltung und Nachvollziehbarkeit bewerten zu lassen. Das Kapitel
geht zunächst auf das Vorgehen ein. Daraufhin werden die Versuchsgruppen, welche
sich nach den in \ref{section:benutzer} erarbeiteten Nutzenden gliedert und
Ergebnisse betrachtet.


\section{Vorgehen und Methodik}
Zu Beginn der Studienplanung wurden Evaluationsaufgaben definiert, welche die
Versuchspersonen Schritt für Schritt durchführen sollten. Dabei sollten die
Teilnehmenden die Think-Aloud-Methode anwenden, dass heißt bei der Nutzung des
System ständig laut denken - das heißt, ihre Gedanken zu verbalisieren, während
sie sich durch die Benutzeroberfläche bewegen und stets mitteilen, wenn sie mit
einer Aufgabe fertig sind \cite{nielsen_usability_1994}. Der Vorteil der Methode
ist, dass durch das Bobachten der Nutzenden nicht nur Probleme auffallen,
sondern diese auch begründet werden. Zudem können wünsche oder  Erwartungen
besser nachvollzogen werden \cite{nielsen_think}.

Um die Gebrauchstauglichkeit und Nützlichkeit der Web-App abschließend
feststellen zu können, wurde ein Online-Fragebogen entworfen \todo{Anhang}. Zu
Beginn des Fragebogens wurden Teilnehmende nach den demografische Daten gefragt.
Diese dienen der besseren Klassifizierung der Daten. Daraufhin wurde die
Technikaffinität mithilfe der \ac{ati}-Skala erfragt. Im dritten Abschnitt
wurden Fragen zu den Funktion der Anwendung gestellt. Hierbei sollten zum einen
bereits vorhandene Funktionen Wichtigkeit zugeordnet werden. Zum anderen sollten
noch nicht vorhandene Funktionen angegeben werden. Dies dient unter anderem dem
Ausblick und abwägen der zukünfitgen Weiterentwicklung des Systems. Des Weiteren
wurde in diesem Teil auf das Wording der Anwendung eingengangen, da diese in der
Zwischenevaluation des \nameref{chapter-design}s  vermehrt zu
Unverständlichkeiten geführt hat. Abschließen wurde mit Hilfe des UEQ-Fragebogen
die Usability getestet. Schließlich wurden Proband:innen befragt, wie sie das
System in seiner Gesamtheit bewerten würden und ob sie sich dieses System für
den regelmäßigen Gebrauch vorstellen könnten.




\section{ATI und UEQ Ergebnisse}
Da sich die Anwendung ledglich in Bereich der Verwaltung unterscheide werden im
folgenden die Technikaffinität sowie Usability der Anwendung beider
Versuchsgruppen dargestellt.

\begin{table}[h]
  \centering
  \caption{Werte der \ac{ati}-Skala}
  \begin{tabular}{lccc}
    \arrayrulecolor{maincolor}\hline
    \sffamily\color{maincolor}Benutzergruppe &
    \sffamily\color{maincolor}Mittelwert $(M)$
                                             & \sffamily\color{maincolor}Standardabweichung $(SD)$ &
    \sffamily\color{maincolor}Teilnehmende $(N)$
    \\
    \arrayrulecolor{maincolor}\hline
    Verleihende                              & 5,00
                                             & 0,58                                                & 3
    \\
    Ausleihende                              & 5,13
                                             & 0,48                                                & 6
    \\
    \arrayrulecolor{maincolor}\hline
  \end{tabular}
  \label{table:atipartzwei}
\end{table}



\section{Verleihende}
\begin{table}[h]
  \centering
  \caption{Teilnehmende der Interviews, Verleihende}
  \begin{tabular}{lll}
    \arrayrulecolor{maincolor}\hline
    \sffamily\color{maincolor}ID & \sffamily\color{maincolor}Alter &
    \sffamily\color{maincolor}Zuständigkeitsbereich
    \\
    \arrayrulecolor{maincolor}\hline
    E-V1                         & 25 - 35 J.                      & Keine
    direkte Zuständigkeit, Zugänge zu verschiedenen Laboren
    \\
    E-V2                         & 25 - 35 J.                      &
    Multimedialabor                                                           \\
    E-V3                         & 25 - 35 J.                      & VR-Labor \\
    \arrayrulecolor{maincolor}\hline
  \end{tabular}
  \label{table:vzwei}
\end{table}

\section{Ausleihende}

\begin{table}[h]
  \centering
  \caption{Teilnehmende der Interviews, Ausleihende \\
    (die mit * gekennzeichneten Versuchspersonen waren Mitarbeitende)}
  \begin{tabular}{lll}
    \arrayrulecolor{maincolor}\hline
    \sffamily\color{maincolor}ID & \sffamily\color{maincolor}Alter &
    \sffamily\color{maincolor}Rolle
    \\
    \arrayrulecolor{maincolor}\hline
    E-A1                         & 19 - 25 J.                      &
    Bachelorstudentin, Hilfswissenschaftlerin
    \\
    E-A2                         & 19 - 25 J.                      &
    Bachelorstudent                                                  \\
    E-A3                         & 19 - 25 J.                      &
    Masterstudent, Hilfswissenschaftler
    \\
    \arrayrulecolor{maincolor}\hline
  \end{tabular}
  \label{table:azwei}
\end{table}


-> E-A1 hat ja gesagt würde es nutzen wollen

\blockquote{App war top...}


\section{Diskussion}