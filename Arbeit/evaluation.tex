%!TEX root = thesis.tex

\chapter{Evaluation}
\label{chapter-evaluation}
Das folgende Kapitel evaluiert das im Rahmen dieser der Arbeit realisierte System. Zunächst werden
die Ziele sowie auf das Vorgehen erläutert. Daraufhin werden die Versuchsgruppen, welche sich nach
den in \nameref{section:benutzer} erarbeiteten Gruppe gegliedert und die Ergebnisse betrachtet.
Abschließende werden die Ergebnisse entsprechend diskutiert zusammengefasst.

\section{Ziele}
Im Rahmen der Forschungsfrage F3 soll herausgearbeitet werden, inwiefern die Gebrauchstauglichkeit
und Nützlichkeit durch das in der vorliegenden Arbeit entwickelte Reservierungssystem gewährleistet
werden kann. Die Evaluation wird zusätzlich dafür genutzt, das entwickelte Reservierungssystem zu
Untersuchung von Forschungsfrage F2 hinsichtlich Funktionen, Gestaltung und Nachvollziehbarkeit
bewerten zu lassen.


\section{Vorgehen und Methodik}
Da es sich um ein universitätsinternes Tool handelt, wurde sich bei den Versuchspersonen, im Rahmen
des Prototypen (N=6) mit Mitarbeitenden des \ac{imis} und Studierenden im Bereich der
Medieninformatik zusammengesetzt, um das konzipierte System zu evaluieren. Zu Beginn der
Studienplanung wurden Evaluationsaufgaben definiert, welche die Versuchspersonen Schritt für Schritt
durchführen sollten (\ref{appendix:Evaluation}). Dabei sollten die Teilnehmenden die
Think-Aloud-Methode anwenden, das heißt bei der Nutzung des System sollten Versuchspersonen ständig
laut denken und somit ihre Gedanken verbalisieren, während sie sich durch die Benutzeroberfläche
bewegen \cite{nielsen_usability_1994}. Der Vorteil der Methode ist, dass durch das Bobachten der
Nutzenden nicht nur Probleme auffallen, sondern diese auch begründet werden. Zudem können wünsche
oder Erwartungen besser nachvollzogen werden \cite{nielsen_think}.

Um die Gebrauchstauglichkeit und Nützlichkeit der Web-App abschließend feststellen zu können, wurde
ein Online-Fragebogen entworfen \todo{Anhang: Umfrage}. Zu Beginn des Fragebogens wurden
Teilnehmende nach den demografische Daten gefragt. Diese dienen der besseren Klassifizierung der
Daten. Daraufhin wurde die Technikaffinität mithilfe der \ac{ati}-Skala erfragt. Im dritten
Abschnitt wurden Fragen zu den Funktion der Anwendung gestellt. Hierbei sollten zum einen bereits
vorhandene Funktionen Wichtigkeit zugeordnet werden. Zum anderen sollten noch nicht vorhandene
Funktionen angegeben werden. Dies dient unter anderem dem Ausblick und abwägen der zukünfitgen
Weiterentwicklung des Systems. Des Weiteren wurde in diesem Teil auf das Wording der Anwendung
eingengangen, da diese in der Zwischenevaluation des \nameref{chapter-design}s  vermehrt zu
Unverständlichkeiten geführt hat. Abschließen wurde mit Hilfe des UEQ-Fragebogen die Usability
getestet. Schließlich wurden Proband:innen befragt, wie sie das System in seiner Gesamtheit bewerten
würden und ob sie sich dieses System für den regelmäßigen Gebrauch vorstellen könnten.



\section{ATI und UEQ Ergebnisse}
Da sich die Anwendung ledglich in Bereich der Verwaltung unterscheidet werden im
Folgenden die Technikaffinität sowie Usability der Anwendung beider
Versuchsgruppen dargestellt.

\begin{table}[h]
  \centering
  \caption{Werte der \ac{ati}-Skala}
  \begin{tabular}{lccc}
    \arrayrulecolor{maincolor}\hline
    \sffamily\color{maincolor}Benutzergruppe &
    \sffamily\color{maincolor}Mittelwert $(M)$
                                             & \sffamily\color{maincolor}Standardabweichung $(SD)$ &
    \sffamily\color{maincolor}Teilnehmende $(N)$
    \\
    \arrayrulecolor{maincolor}\hline
    Verleihende                              & 5,00
                                             & 0,58                                                & 3
    \\
    Ausleihende                              & 5,13
                                             & 0,48                                                & 6
    \\
    \arrayrulecolor{maincolor}\hline
  \end{tabular}
  \label{table:atipartzwei}
\end{table}



\section{Verleihende}
Der kommende Abschnitt umfasst die Evaluationsergebnisse der Aufgaben, unter der Anwendung der
Think-Aloud-Methode, aufseiten der Verleehenden. \ref{table:vzwei} zeigt die Rolle und das Alter der
Versuchspersonen. Die IDs der Versuchspersonen werden als Verweise in dem folgenden Abschnitt
verwendet.


\begin{table}[h]
  \centering
  \caption{Teilnehmende der Interviews, Verleihende}
  \begin{tabular}{lll}
    \arrayrulecolor{maincolor}\hline
    \sffamily\color{maincolor}ID & \sffamily\color{maincolor}Alter &
    \sffamily\color{maincolor}Zuständigkeitsbereich
    \\
    \arrayrulecolor{maincolor}\hline
    E-V1                         & 25 - 35 J.                      & Keine
    direkte Zuständigkeit, Zugänge zu verschiedenen Laboren
    \\
    E-V2                         & 25 - 35 J.                      &
    Multimedialabor                                                           \\
    E-V3                         & 25 - 35 J.                      & VR-Labor \\
    \arrayrulecolor{maincolor}\hline
  \end{tabular}
  \label{table:vzwei}
\end{table}



\section{Ausleihende}

Der kommende Abschnitt umfasst die Evaluationsergebnisse der Aufgaben, unter der Anwendung der
Think-Aloud-Methode, aufseiten der Ausleihenden. \ref{table:azwei} zeigt die Rolle und das Alter der
Versuchspersonen. Die IDs der Versuchspersonen werden als Verweise in dem folgenden Abschnitt
verwendet.

\begin{table}[h]
  \centering
  \caption{Teilnehmende der Evaluation, Ausleihende}
  \begin{tabular}{lll}
    \arrayrulecolor{maincolor}\hline
    \sffamily\color{maincolor}ID & \sffamily\color{maincolor}Alter &
    \sffamily\color{maincolor}Rolle
    \\
    \arrayrulecolor{maincolor}\hline
    E-A1                         & 19 - 25 J.                      &
    Bachelorstudentin, Hilfswissenschaftlerin
    \\
    E-A2                         & 19 - 25 J.                      &
    Bachelorstudent                                                  \\
    E-A3                         & 19 - 25 J.                      &
    Masterstudent, Hilfswissenschaftler
    \\
    E-A4                         & 19 - 25 J.                      &
    Masterstudent, Hilfswissenschaftler
    \\
    \arrayrulecolor{maincolor}\hline
  \end{tabular}
  \label{table:azwei}
\end{table}

Das Dashboard wurde von den Nutzenden stets als Übersichtlich und hilfreich betietelt. In der
Beschreibung der Tabs würde ein \textit{deine} hilfreich sein, um zu verdeutlichen, dass es sich um
die eigens getätigten Reservierungen handelt und nicht um alle Reservierungen (E-A1). Da das
Dashboard bei keinen Reservierungen viel freien Platz lässt, wäre eine direkte Übersicht über
Kategorien denkbar sinnvoll (E-A3). 

Die Kategorien wurden von allen Nutzenden als einfach und wichtig bezeichent. Positiv wurde angemerkt, dass
diese ein schnelles nach gucken ermöglichen, um zu erfahren, was gibt es zum Ausleihen gibt (E-A3). Ohne
Kateogrien sei die Anwendung nur halb so übersichtlich (V-E4).

\begin{longtable}{p{0.85\textwidth}} 
  \arrayrulecolor{maincolor}\hline
  \enquote{\textit{[\dots], die Kategorien das hätte ich mir nur ein bisschen übersichtlicher gewünscht,
  [\dots] und vielleicht auf dem Dashboard schon eine Art Übersicht.}} \\
  \arrayrulecolor{maincolor}\hline
\end{longtable}

Ein weiterer Punkt war, dass beim aufgeklappen einer Unterkateogien, zunächt wieder auf
\textit{zurück} geklickt werden, um die restlichen Kategorien sehen zu können (V-E3). 

Allen Versuchespersonen ist das Suchen über die Sucheleiste sowie die Navigationsleiste leicht
gefallen. Ansprechpartner:innen und der Abholort eines Assets wurden auf den ersten Blick entsteckt.
Wobei eine Raumnummer bei dem Abholort als Fehlend dekleriert wurde. 

\begin{longtable}{p{0.85\textwidth}} 
  \arrayrulecolor{maincolor}\hline
  \enquote{\textit{[\dots], wenn hier die Kontaktdaten zu den Verantwortlichen sind [\dots], dass man
    draufklicken könnte und direkt kontaktieren könnte über die App [\dots]}} \\
  \arrayrulecolor{maincolor}\hline
\end{longtable}

Suche über die Navigationsleiste -> Zeitraum eingeben und dann Asset suchen nicht so 100 eindeutig
(AUFWENDIG)

Der Reservierungsprozess wurde durch die Kalenderkomponente zu Teilen erschwärt. Als irritirend
wurde angemerkt, dass das Datum des aktuellen Tages nicht hervorgehoben wurde (E-A1,2). Nicht
auswählbare Tage werden bisher lediglich ausgegraut, was die Unterscheidung zwischen generell nicht
auswählbare Tage (Wochenende) und bereits reservierten Zeiträumen unübersichlich wirken lässt. Die
Uhreitenansicht wurde bis dato auf Englisch angezeigt, wobei Versuchspersonen anmerkten, dass diese
auf Deutsch für die einheitlichkeit sinnvoller wären (E-A1). Sobald der Zeitraum ausgewählt wurde,
erhielten Nutzenden eine Reseriverungsübersicht und eine Zusammenfassungübersicht. Die Bestätigung
von zwei Reseriverungsübersichten war für die Versuchsperson zu teilen unklar. Bereits bei der
ersten Übersicht interpretierten Nutzenden die Reservierung als abgeschlossen und kehrten zum
Dashboard zurück, wo die Reservierung jedoch nicht angezeigt wurde. Die wurde unter anderem auf die
Betitelung des Butten \textit{weiter} zurückgeführt. Um den Reseriverungsprozess sichtbarer
gestalten zu können, wurde Vorschlag eine Vortschriftanzeige genannt (E-A3). Ein weiterer Punkte,
weswegen das Abschließen der Reservierung als (nicht)entgültig interpretiert werden kann, ist das
Fehlen eines \textit{Zurück zum Dashboard}-Button auf der Reservierungszusammenfassung (E-A1,3).

Um erneut Informationen zum rservierten Material einsehen zu können erschien es als umständlich,
dass Nutzenden erneut nach dem Material suchen müssen. Alle Versuchspersonen haben intuitiv auf
das im Dashboard angezeigte Listenelement geklickt  und eine Verlinkung erwartet (E-V1 bis E-V4). 

 Beim Bearbeiten des Zeitraums stellte sich das Kalender-Popup, auf der mobile Version, als
unhandlich heraus, da sich automatisch die Tastatur öffnet und die Komponente sich bei danebentippen
direkt wieder schloss. Außerdem fehlte eine Bestätigung der Änderung oder ein \textit{Änderungen
speichern}-Button (E-A1 bis 3). Ebenfalls wurde von allen Versuchsperson die im Kalender
ausgewählten Tage per ziehen versucht zuverändert. Eine Versuchsperson hat zunächst auf das
Listenelement geklickt, in der Erwatung, dass sich eine Seite zum Bearbeiten des Materials öffnet
(E-A4).

Das Löschene einer Reservierung hat bei allen Versuchspersonen gut funktoniert und wurde als
intuitiv bezeichnet. Eine Person wies darauf hin, das vor dem Löschene eine Warnung sinnvoll ist,
falls Nutzende ausversehen auf Löschen klicken (V-E1). 

Die Assetstatus \textit{Fest verbaut} und \textit{Am Standort nutzbar} führten bei den
Versuchspersonen zu iritation. Versuchspersonen fehlte zu den Betitelung eine Erkläurng.  Wording oder legende.

\begin{longtable}{p{0.85\textwidth}} 
  \arrayrulecolor{maincolor}\hline
  \enquote{\textit{Was bedeutet das für die Reservierung?}} \\
  \enquote{\textit{[\dots] das heißt von \enquote{fest verbaut} frage ich mich, ob ich das jetzt trotzdem
  ausleihen kann oder ob \enquote{fest verbaut} heißt, dass ich dass nur vor Ort nutzen kann[\dots]}}
  \\
  \arrayrulecolor{maincolor}\hline
\end{longtable}

Generell wurde die Oberfläche als übersichtlich und \textit{clean} beschrieben (E-A1 bis E-A4). Der Wunsch die
Anwendung im Universitätalltag für Projekte nutzen zu können wurde entsprechend geäußert (E-A1,3,4).

\begin{longtable}{p{0.85\textwidth}} 
  \arrayrulecolor{maincolor}\hline
  \enquote{\textit{Ja, wäre cool, wenn wir das am IMIS wirklich nutzen könnten}}\\
  \enquote{\textit{[\dots] es ist insgesamt sehr übersichtlich und ich würde sagen, dass ich mich auf jeden
  Fall gut zurechtfinde [\dots]}} \\
  \arrayrulecolor{maincolor}\hline
\end{longtable}

\section{Diskussion}