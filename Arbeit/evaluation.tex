%!TEX root = thesis.tex

\chapter{Evaluation}
\label{chapter-evaluation}
Im Rahmen der Forschungsfrage F3 soll herausgearbeitet werden, inwiefern die Gebrauchstauglichkeit
und Nützlichkeit durch das in der vorliegenden Arbeit entwickelte Reservierungssystem gewährleistet
werden kann. Da es sich um ein universitätsinternes Tool handelt, wurde sich
bei den Versuchspersonen, im Rahmen eines Proof of Concepts (N=5) mit Mitarbeitenden des \ac{imis}
und Studierenden im Bereich der Medieninformatik zusammengesetzt, um das konzipierte Studiendesign
zu evaluieren. Die Evaluation wird zusätzlich dafür genutzt, das entwickelte Reservierungssystem zur
Untersuchung von Forschungsfrage F2 hinsichtlich Funktionen, Gestaltung und Nachvollziehbarkeit
bewerten zu lassen. Das Kapitel geht zunächst auf das Vorgehen und daraufhin auf die Benutzergruppen
und Ergebnisse ein.


\section{Vorgehen und Methodik}
- Wording mit Abfragen


Zu Beginn der Studienplanung wurden Evaluationsaufgaben definiert, die die Versuchspersonen Schritt
für Schritt durchführen sollten. Dabei sollten diese Think-Aloud anwenden und stets mitteilen, wenn sie
mit einer Aufgabe fertig sind.

Um die Gebrauchstauglichkeit und Nützlichkeit der Web-App abschließend feststellen zu können, wurde ein
Online-Fragebogen entworfen \todo{Anhang}. Zu Beginn des Fragebogens mussten Proband:innen die Rolle
(Verleihende, Ausleihende) angeben.

Dieser enthielt Fragen zu demografische Daten, UEQ und
den Funktionen der Anwendung (z. B. Ordnen Sie die im System bereits vorhandenen Funktionen nach
Wichtigkeit an.). Abschließend hatten Versuchspersonen in Freitexten weitere Anmerkungen zu geben.

In beiden Fällen wurde zuerst die Nützlichkeit der eingesetzten Funktionalitäten bewertet. Dies geschah mit einer
4-Punkt Likert-Skala, welche mit nicht hilfreich, wenig hilfreich, hilfreich und sehr hilfreich beschriftet
wurde.

Schließlich wurden Proband:innen befragt, wie sie das System in seiner Gesamtheit bewerten würden
und ob sie sich dieses System für den regelmäßigen Gebrauch vorstellen könnten.


\section{Verleihende}
\begin{table}[h]
  \centering
  \caption{Teilnehmende der Interviews, Verleihende}
  \begin{tabular}{lll}
    \arrayrulecolor{maincolor}\hline
    \sffamily\color{maincolor}ID & \sffamily\color{maincolor}Alter &
    \sffamily\color{maincolor}Zuständigkeitsbereich                                  \\
    \arrayrulecolor{maincolor}\hline
    V1                           & 25 - 35 J.                      & Keine direkte
    Zuständigkeit, Zugänge zu verschiedenen Laboren                                  \\
    V2                           & 25 - 35 J.                      & Multimedialabor
    \\
    V3                           & 25 - 35 J.                      & VR-Labor
    \\
    V4                           & 40 - 59 J.                      & Administratives \\
    \arrayrulecolor{maincolor}\hline
  \end{tabular}
  \label{table:vzwei}
\end{table}

\begin{table}[h]
  \centering
  \caption{Werte der \ac{ati}-Skala}
  \begin{tabular}{lccc}
    \arrayrulecolor{maincolor}\hline
    \sffamily\color{maincolor}Benutzergruppe & \sffamily\color{maincolor}Mittelwert
    $(M)$                                    & \sffamily\color{maincolor}Standardabweichung $(SD)$ &
    \sffamily\color{maincolor}Teilnehmende $(N)$                                                          \\
    \arrayrulecolor{maincolor}\hline
    Verleihende                              & 5,00                                                & 0,58
                                             & 3                                                          \\
    Ausleihende                              & 5,13                                                & 0,48
                                             & 6                                                          \\
    \arrayrulecolor{maincolor}\hline
  \end{tabular}
  \label{table:atipartzwei}
\end{table}

\section{Ausleihende}

\begin{table}[h]
  \centering
  \caption{Teilnehmende der Interviews, Ausleihende \\
    (die mit * gekennzeichneten Versuchspersonen waren Mitarbeitende)}
  \begin{tabular}{lll}
    \arrayrulecolor{maincolor}\hline
    \sffamily\color{maincolor}ID & \sffamily\color{maincolor}Alter &
    \sffamily\color{maincolor}Rolle                                                     \\
    \arrayrulecolor{maincolor}\hline
    A1                           & 19 - 25 J.                      & Bachelorstudentin,
    Hilfswissenschaftlerin                                                              \\
    A2                           & 19 - 25 J.                      & Bachelorstudent
    \\
    A3                           & 19 - 25 J.                      & Masterstudent,
    Hilfswissenschaftler                                                                \\
    A4*                          & 19 - 25 J.                      & Bachelorstudentin
    \\
    A5*                          & 19 - 25 J.                      & Bachelorstudentin
    \\
    \arrayrulecolor{maincolor}\hline
  \end{tabular}
  \label{table:azwei}
\end{table}


\begin{table}[h]
  \centering
  \caption{Werte der \ac{ati}-Skala}
  \begin{tabular}{lccc}
    \arrayrulecolor{maincolor}\hline
    \sffamily\color{maincolor}Benutzergruppe & \sffamily\color{maincolor}Mittelwert
    $(M)$                                    & \sffamily\color{maincolor}Standardabweichung $(SD)$ &
    \sffamily\color{maincolor}Teilnehmende $(N)$                                                          \\
    \arrayrulecolor{maincolor}\hline
    Verleihende                              & 5,00                                                & 0,58
                                             & 3                                                          \\
    Ausleihende                              & 5,13                                                & 0,48
                                             & 6                                                          \\
    \arrayrulecolor{maincolor}\hline
  \end{tabular}
  \label{table:atidrie}
\end{table}


\blockquote{App war top...}


\section{Diskussion}