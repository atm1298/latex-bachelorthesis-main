

\chapter{Evaluation}
\label{chapter-evaluation}
Das folgende Kapitel stellt die Evaluation des im Rahmen dieser Arbeit realisierten Systems vor.
Zunächst werden die Ziele der Evaluation (\ref{sec:zieleev}) und das Vorgehen (\ref{sec:vorgeen})
erläutert. Hierbei werden die Versuchsgruppen, welche sich aus der \nameref{section:Nutzenden}
ergeben, näher betrachtet. Abschließend werden die Ergebnisse umfassend präsentiert (\ref{sec:prese}), zusammengefasst
und entsprechend diskutiert (\ref{sec:disc}).

\section{Ziele}
\label{sec:zieleev}
Im Rahmen der Forschungsfrage F3 soll herausgearbeitet werden, inwiefern die
Gebrauchs-tauglichkeit und Nützlichkeit durch das entwickelte Reservierungssystem
gewährleistet werden kann. Die Evaluation wird zusätzlich zur Bewertung
hinsichtlich der Funktion, Gestaltung und Nachvollziebarkeit genutzt, um
Forschungsfrage F2 zu beantworten.


\section{Vorgehen und Methodik}
\label{sec:vorgeen}
Da es sich um ein universitätsinternes Tool handelt, wurde sich zur Evaluation im Rahmen einer
Laborstudie (N=8) mit Mitarbeitenden des \ac{imis} und Studierenden im Bereich der Medieninformatik
zusammengesetzt \cite{jandl_paper_2015}. Zu Beginn der Studienplanung wurden Evaluationsaufgaben
definiert, welche die Versuchspersonen Schritt für Schritt durchführen sollten
(\ref{appendix:Evaluation}). Dabei sollten die Teilnehmenden die Think-Aloud-Methode anwenden.
Hierbei werden Versuchspersonen gebeten, ihre Gedanken während der Nutzung zu verbalisieren
\cite{nielsen_usability_1994}. Als Vorteil gegenüber anderen Methoden werden Probleme somit nicht
nur erfasst, sondern auch begründet \cite{nielsen_usability_1994}.

Um die Gebrauchstauglichkeit und Nützlichkeit der Web-App abschließend feststellen zu können, wurde
ein Online-Fragebogen entworfen (\ref{appendix:digitaleMedien}). Zu Beginn des Fragebogens wurden
die demografischen Daten der Teilnehmenden erfasst. Diese dienen der besseren Klassifizierung der
Daten. Daraufhin wurde die Technikaffinität mithilfe der \ac{ati}-Skala erfragt. Im dritten
Abschnitt wurden Fragen zu den Funktionen der Anwendung gestellt. Hierbei sollten die wahrgenommene
Nützlichkeit der Funktionen angegeben und gewünschte Funktionen genannt werden. Dies dient unter
anderem dem Ausblick und Abwägen der zukünfitgen Weiterentwicklung des Systems. Des Weiteren wurde
in diesem Teil auf die Verwendung verschiedener Begrifflichkeiten innerhalb der Anwendung
eingegangen, da diese in der Zwischenevaluation des \nameref{chapter-design}s vermehrt zu
Unverständlichkeiten geführt haben.

Zusätzlich wurde mithilfe des \ac{ueq} die Gebrauchstauglichkeit erfasst
\cite{laugwitz2009subjektive}. Dieser teilt die Usability des Systems in seine pragmatische und
hedonische Qualität auf. Hierbei beschreibt die pragmatische Qualität die wahrgenommene Fähigkeit
eines Systems, Nutzende in ihrer Aufgabenerledigung effektiv und effizient zu unterstützen
\cite{hassenzahl_thing_2004}. Die hedonische Qualität bezieht sich hingegen auf die emotionalen und
ästhetischen Bestandteile eines Systems. So wird beispielsweise die Erfüllung menschlicher
Bedürfnisse nach Neugier und sozialem Vergleich erfasst \cite{hassenzahl_thing_2004}.

Abschließend wurden Proband:innen nach einer Gesamtbewertung des Systems befragt und ob sie sich die
regelmäßige Nutzung vorstellen könnten.



\section{ATI und UEQ Ergebnisse}
\label{sec:prese}
Da sich die Anwendung lediglich im Bereich der Verwaltung unterscheidet, werden im
Folgenden die Technikaffinität und die bewertete Usability der Anwendung beider
Versuchsgruppen dargestellt.

Zur Vollständigkeit sollte die Gebrauchstauglichkeit der Anwendung mithilfe des
\ac{ueq} betrachtet werden. Werte zwischen -0.8 und 0.8 stehen für eine neutrale
Bewertung der entsprechenden Skala \cite{schrepp2018user}. Somit stehen Werte
größer als 0.8 für eine positive Bewertung und Werte kleiner als -0.8 für eine
negative Bewertung. Der Bereich der Skalen liegt zwischen -3 (sehr schlecht) und
+3 (sehr gut). Für die Interpretation der Ergebnisse sollte beachtet werden,
dass kein Vergleichsystem vorliegt. Daher wurde sich an dem \ac{ueq} Benchmark
nach \citeA{schrepp2013user} orientiert. 
\begin{table}[h]
        \centering
        \caption{Werte der kurzen \ac{ueq}-Skala}
        \begin{tabular}{lc}
                \arrayrulecolor{maincolor}\hline
                Pragmatische Qualität & 1.75 \\
                Hedonsiche Qualität   & 0.75 \\
                Gesamt                & 1.25 \\
                \arrayrulecolor{maincolor}\hline
        \end{tabular}
        \label{table:ueq}
\end{table}

Des Weiteren wurde mithilfe des \ac{ati} das technische Interesse und Verständnis der Teilnehmenden
festgestellt \cite{attig_assessing_2017}. In beiden Gruppen konnten wie zu Beginn in der
\nameref{section:Nutzenden} lediglich geringe Unterschiede innerhalb der soziodemografischen Daten
festgestellt werden.

Die erfasste Technikaffinität der Proband:innen wird in \ref{table:atipartzwei} dargestellt. Durch
die Hinzunahme zweier Vergleichsstichproben aus \citeA{franke_personal_2019} (M=4.14, N=300 und
M=4.23, N=65), lässt sich schlussfolgern, dass die Nutzendengruppen eine vergleichsweise hohe
Technikaffinität aufweisen.

\begin{table}[h]
        \centering
        \caption{Evaluations-Ergebniswerte der \ac{ati}-Skala}
        \begin{tabular}{lccc}
                \arrayrulecolor{maincolor}\hline
                \sffamily\color{maincolor}Nutzendengruppe           &
                \sffamily\color{maincolor}Mittelwert $(M)$          &
                \sffamily\color{maincolor}Standardabweichung $(SD)$ &
                \sffamily\color{maincolor}Teilnehmende $(N)$                          \\
                \arrayrulecolor{maincolor}\hline
                Verleihende                                         & 5.11 & 0.48 & 3 \\
                Ausleihende                                         & 5.18 & 0.49 & 8 \\
                \arrayrulecolor{maincolor}\hline
        \end{tabular}
        \label{table:atipartzwei}
\end{table}

\section{Verleihende}
\label{sec:ev-ver}
Dieser Abschnitt umfasst die Evaluationsergebnisse, welche spezifisch die Aufgaben der Verleihenden
betreffen. Hierbei wird der Fokus auf den Verwaltungsteil des Systems beschränkt. \ref{table:vzwei}
zeigt die Rollen der Versuchspersonen. Die IDs der Versuchspersonen werden in dem folgenden
Abschnitt als Verweise verwendet.

\begin{table}[h]
        \centering
        \caption{Teilnehmende der Evaluation: Verleihende}
        \begin{tabular}{lll}
                \arrayrulecolor{maincolor}\hline
                \sffamily\color{maincolor}ID &
                \sffamily\color{maincolor}Zuständigkeitsbereich \\
                \arrayrulecolor{maincolor}\hline
                EV1                          & Multimedialabor  \\
                EV2                          & VR-Labor         \\
                EV3                          & Koordination               \\
                \arrayrulecolor{maincolor}\hline
        \end{tabular}
        \label{table:vzwei}
\end{table}

Für Verleihende ist es von hoher Bedeutung überprüfen zu können, ob die richtige Person das
entsprechende Asset abholt. Diese Information wurde nach dem Klicken auf das Listenelement erwartet
(EV1, EV3). Des Weiteren wird die Menge der möglichen Aktionen (\enquote{abgeholt} oder
\enquote{zurückgegeben}) bei einer hohen Anzahl an Listenelementen unübersichtlich. Stattdessen
wurde eine Bearbeitungsansicht beim Klicken des Listenelements erwartet und/oder vorgeschlagen
(EV1-3). Außerdem wurden die \enquote{nicht abgeholt} und \enquote{nicht zurückgegeben}-Buttons als
irritierend empfunden, wenn die Abholung oder Rückgabe des Assets noch nicht anstand. Diese wurden
erst erwartet, wenn die Reservierung bereits überfällig ist (EV1, EV2). Zudem fehlt eine Rückmeldung
bei getätigten Aktionen und die Möglichkeit Aktionen rückgängig zu machen, falls diese versehentlich
ausgelöst wurden (EV1, EV3).

\begin{longtable}{p{0.85\textwidth}} \arrayrulecolor{maincolor}\hline
        \enquote{\textit{\enquote{Abgeholt}, dass muss man bestätigen können. Wenn ich
        da ausversehen raufklicke [\dots] wichtig ist eine undo-Oberfläche [\dots]
        keine Ahnung wie ich das jetzt zurückhole [\dots]}} \\
        \arrayrulecolor{maincolor}\hline
\end{longtable}

Des Weiteren wurde das Anzeigen von überflüssigen Informationen und das Verbergen von relevanten
Informationen im Listenelement kritisiert. Für die Abholung und Rückgabe eines Assets ist lediglich
das Datum der jeweiligen Aktion relevant, jedoch wurden stets beide Daten angegeben (EV1-3). Zudem
löste die fehlende Beschriftung der \enquote{(nicht) abgeholt} und \enquote{(nicht)
        zurückgegeben}-Buttons Unsicherheit aus (EV1, EV3). Durch das Verschieben der Buttons an den unteren
Rand des Listenelements und das Einbinden eines Hinweistextes könnte diese Problematik gelöst
werden (EV1). Außerdem sei eine Sortierung der Reservierungen nach ihrem Datum hilfreich, um die
aktuellsten und somit relevantesten Reservierungen zuerst anzuzeigen (EV2, EV3). Zuletzt wurde das Fehlen
der Assetbilder bemängelt (EV3).

Des Weiteren wurde sich das Ablehnen von Reservierungen von Verleihenden gewünscht. Dies stelle
eine zentrale Funktion dar, welche in der jetzigen Umsetzung des Systems nicht eingebunden ist und
separat per E-Mail geschehen müsste (EV1, EV3).

\section{Ausleihende}
Im Folgenden werden die Ergebnisse der Aufgaben, welche von Ausleihenden ausgeführt wurden,
präsentiert. \ref{table:azwei} zeigt die Rolle und das Alter der Versuchspersonen. Die IDs der
Versuchspersonen werden in dem folgenden Abschnitt als Verweise verwendet.

\begin{table}[h]
        \centering
        \caption{Teilnehmende der Evaluation: Ausleihende}
        \begin{tabular}{lll}
                \arrayrulecolor{maincolor}\hline
                \sffamily\color{maincolor}ID & \sffamily\color{maincolor}Alter &
                \sffamily\color{maincolor}Rolle                                  \\
                \arrayrulecolor{maincolor}\hline
                EA1                          & 19 - 25 J.                      &
                Masterstudent:in, Hilfswissenschaftlerin:in                      \\
                EA2                          & 19 - 25 J.                      &
                Bachelorstudent:in                                               \\
                EA3                          & 19 - 25 J.                      &
                Bachelorstudent:in, Hilfswissenschaftler:in                      \\
                EA4                          & 19 - 25 J.                      &
                Bachelorstudent:in, Hilfswissenschaftler:in                      \\
                EA5                          & 19 - 25 J.                      &
                Masterstudent:in                                                 \\
                \arrayrulecolor{maincolor}\hline
        \end{tabular}
        \label{table:azwei}
\end{table}

\subsection{Einlogmaske}
Auf dem Einlogbildschirm wurde der Name der Anwendung erwartet, da die
Zuordnung des Tools sonst schwerfallen könnte. Außerdem ist die Wortwahl und der
Platz des \enquote{\textit{Mit IDM Accout einloggen}}-Button problematisch, da die
Eingabefelder und der Button als zwei unterschiedliche Einlog-Möglichkeiten verstanden
wurden und nicht als Bestätigungs-Button für das Formular (EV1, EA4).

\subsection{Dashboard}
Das Dashboard wurde von Nutzenden stets als übersichtlich und hilfreich
betitelt, wobei die Namensgebung \textit{Dashboard} als einziges englisches
Wort negativ aufgefallen ist (EV1). In der Beschreibung der Tabs würde ein
\enquote{\textit{deine}} förderlich sein, um zu verdeutlichen, dass es sich um die \textit{eigens}
getätigten Reservierungen handelt und nicht um alle Reservierungen (EA1). Da
das Dashboard ohne getätigte Reservierungen viel freien Platz lässt, wäre eine
direkte Übersicht über Kategorien erwünscht (EA3).

\subsection{Kategorien}
Die Kategorien wurden überwiegend als einfach und wichtig bezeichnet. Positiv wurde angemerkt, dass
diese ein schnelles Durchsuchen ermöglichen, um zu erfahren, was für Materialien ausgeliehen werden
können (EA3). Ohne Kategorien sei die Anwendung deutlich weniger übersichtlich (EV4).

\begin{longtable}{p{0.85\textwidth}} \arrayrulecolor{maincolor}\hline
        \enquote{\textit{[\dots], die Kategorien das hätte ich mir nur ein bisschen
                        übersichtlicher gewünscht, [\dots] und vielleicht auf dem Dashboard schon eine
        Art Übersicht.}} \\
        \arrayrulecolor{maincolor}\hline
\end{longtable}

Eine weitere Beobachtung, in Hinsicht auf die Kategorien, betrifft das Aufgeklappen der
Unterkategorien. Nachdem diese geöffnet wurden, müssen Nutzende erst auf den \enquote{Zurück}-Button
klicken, um Oberkategorien erneut einsehen zu können (EV3, EV5). Zudem wurde die Navigationsleiste
beim Auswählen einer Unterkategorie in der mobilen Ansicht nicht automatisch geschlossen (EV1-5).
Schließlich wurde sich eine Anzeige der Stückzahl an Assets gewünscht, welche sich hinter der
entsprechenden (Unter-)Kategorie befinden(EV1, EA5).

\subsection{Suche}
Allen Versuchspersonen ist das Suchen über die Suchleiste sowie die Navigationsleiste leicht
gefallen. Ansprechpartner:innen und der Abholort eines Assets wurden auf den ersten Blick entdeckt.
Jedoch sollte der Abholort um eine Raumnummer ergänzt werden (EA1, EA5).
\begin{longtable}{p{0.85\textwidth}} \arrayrulecolor{maincolor}\hline
        \enquote{\textit{[\dots], wenn hier die Kontaktdaten zu den Verantwortlichen
                        sind [\dots], dass man draufklicken könnte und direkt kontaktieren könnte
        über die App [\dots]}} \\
        \arrayrulecolor{maincolor}\hline
\end{longtable}

Das Suchen über die Navigationsleiste wurde als modern und ansprechend befunden. Das Suchen im
Hamburger-Menü über einen bestimmten Zeitraum erschien sehr praktisch (EA3, EA5). Ergänzend könnte
in der Navigationsleiste ein Filter-Icon eingebaut werden, welches ebenfalls das Suchen über einen
gewünschten Zeitraum ermöglicht (E-A5).

Um die Suchergebnisse für Nutzende übersichtlicher zu gestalten, sollten einige Meta-Daten zur
Suche angezeigt werden. Hierzu wurde auf ähnliche Suchmaschinen in Online-Shops verwiesen, welche
beispielsweise die Anzahl der durchsuchten Objekte, die Ergebnissanzahl und Vorschläge für ähnliche
oder korrigierte Suchbegriffe angeben (EV1, EA5).

\subsection{Reservieren und Kalender}
Der Reservierungsprozess wurde durch die Kalenderkomponente zu Teilen erschwert. Als irritierend
galt die fehlende Hervorhebung des aktuellen Tages (E-A1,2). Nicht auswählbare Tage werden bisher
lediglich ausgegraut, was die Unterscheidung zwischen generell nicht auswählbaren Tagen (Wochenende)
und bereits reservierten Zeiträumen unübersichlich wirken lässt (EV1, EA1). Auch die Bedeutung
\enquote{Zurücksetzen} und \enquote{Heute}-Buttons wurde nicht direkt deutlich. Zudem könnte das
versehentliche Nutzen der Buttons zu Frustration führen (EV1). Zuletzt wurde angemerkt, dass die
Kalender- und Uhrzeitansicht als einzige Oberfläche in Englisch beschriftet ist und verschiedene
Datumsformate nutzt. Zur Einheitlichkeit sollte nur ein Datumsformat genutzt werden und die
Oberfläche standardmäßig in Deutsch verfügbar sein (EV1, EV3, EA1).

Sobald der Zeitraum ausgewählt wurde, erhielten Nutzende eine Reservierungsübersicht und
Zusammenfassung. Die Bestätigung beider Reservierungsübersichten war für die Versuchsperson unklar.
Bereits bei der ersten Übersicht interpretierten Nutzende die Reservierung als abgeschlossen. Einige
Nutzende kehrten direkt zum Dashboard zurück, wo die Reservierung jedoch nicht angezeigt wurde.
Dieses Missverständnis wurde unter anderem auf die Betitelung des Button \enquote{weiter}
zurückgeführt. Um den Reservierungsprozess sichtbarer zu gestalten, wurde das Einblenden einer
Fortschrittsanzeige vorgeschlagen (EV1, EA3). Das Fehlen eines \enquote*{Zurück zum Dashboard}-Buttons
auf der Reservierungszusammenfassung könnte ebenfalls zur falschen Interpretation des
Abschlusszustands beitragen (EA1, EA3). Des Weiteren wurde die Animation der Kalenderkomponente beim
Schließen dieser als Bestätigung interpretiert (EV1, EA3).

Für die Detailansicht eines Assets wurde sich eine konsistente Übersicht gewünscht (EV1, EV2).
Hierzu zählt auch, dass die Kontaktinformationen zu Teilen triviale Beschriftungen beinhalten und
relevante Informationen fehlen. Beispielsweise wurde die Beschriftung des Namens als überflüssig
bezeichnet und sich eine Verlinkung mit Icons zu der E-Mail-Adresse und bisher fehlenden
Telefonnummer gewünscht (EV1, EA3).

Um erneut Informationen zum reservierten Material einsehen zu können, erschien es als umständlich,
dass Nutzende erneut nach dem Material suchen müssen. Alle Versuchspersonen haben intuitiv auf das
im Dashboard angezeigte Listenelement geklickt und eine Verlinkung erwartet (EV1-EV5, EA1, EA2).
Eine Versuchsperson hat zunächst auf das Listenelement geklickt, in der Erwartung, dass sich eine
Seite zum Bearbeiten des Materials öffnet (EA4). Beim Bearbeiten des Zeitraums stellte sich das
Kalender-Popup auf der mobile Version als unhandlich heraus. Beim Öffnen der Komponente wurde
automatisch die Tastatur eingeblendet, welche die Komponente überwiegend verdeckte. Zudem schloss
sich die Kalenderkomponente, wenn der äußere Bereich versehentlich angetippt wurde. Außerdem
fehlte eine Bestätigung der Änderung oder ein \enquote{Änderungen speichern}-Button (EA1-EA5, EV1,
EV2). Ebenfalls versuchten alle Versuchspersonen, die im Kalender ausgewählten Tage per Ziehen
zu verändern.

Das Löschen einer Reservierung hat bei allen Versuchspersonen gut funktoniert und wurde als
\enquote{intuitiv} bezeichnet. Zwei Versuchspersonen wiesen darauf hin, dass vor dem Löschen eine
Warnung angezeigt werden sollte, falls Nutzende versehentlich auf \enquote{Löschen} klicken (VE1,
EA1).


\subsection{Status der Assets}
Die Assetstatus \enquote{Fest verbaut} und \enquote{Am Standort nutzbar} führten bei den
Versuchspersonen zu Irritationen. Versuchspersonen fehlte zu der Betitelung eine Erklärung, da die
Implikationen für die Ausleihbarkeit unklar blieben (EA1-EA5). Die teils fehlende und inkonsistente
Farbcodierung des Status erschwerte die Interpretation zusätzlich. Während alle Status in der
Kachelansicht orange angezeigt wurden, fehlte die Farbcodierung in der Detailansicht (EV1, EV3, EA1,
EA2, EA5).

\begin{longtable}{p{0.85\textwidth}} \arrayrulecolor{maincolor}\hline
        \enquote{\textit{Was bedeutet das (\enquote{fest verbaut}) für die Reservierung?}} \\
        \enquote{\textit{[\dots] das heißt von \enquote{fest verbaut} frage ich mich,
                        ob ich das jetzt trotzdem ausleihen kann oder ob \enquote{fest verbaut} heißt,
        dass ich dass nur vor Ort nutzen kann[\dots]}}                                     \\
        \enquote{\textit{[\dots] das (der Assetstatus) könnnte vielleicht ein bisschen
        salienter sein.}}                                                                  \\
        \arrayrulecolor{maincolor}\hline
\end{longtable}



\section{Fragebogen}
Der Aufbau des Fragebogens und eine ausführliche Auflistung der Ergebnisse befindet sich in
(\ref{appendix:digitaleMedien}). Versuchspersonen mussten die Funktionen in einer
4-Punkte-Likert-Skala bewerten. Anschließend sollten die gleichen Funktionen nach ihrer empfundenen
Wichtigkeit in einem Ranking angeordnet werden.

Zunächst konnte aus den Antworten entnommen werden, dass keine bisher implementierte
Funktion unerwünscht sei. Des Weiteren ließ sich folgern, dass das Reservieren und die Suche die
zentralen Funktionen der Anwendung darstellen. Außerdem wurde der Assetstatus als eine wichtige
Funktion empfunden.

Auf die Frage, welche Funktionalitäten Versuchspersonen fehlten, ergab sich folgendes Ranking:
\begin{enumerate}
        \item Nach Zweck suchen (M \(\approx 2.13\), SD \(\approx 0.99\))
        \item Set-Vorschläge (M \( = 2.5\), SD \(\approx 1.07\))
        \item Filter-Funktion (M \( \approx 2.63\), SD \(\approx 1.19\))
        \item Erfahrungsberichte und Kommentare (M \( = 2.75\), SD \(\approx 1.39\))
\end{enumerate}

Zu erkennen ist, dass die Standardabweichung (SD) bei zunehmender Platzierung mit ansteigt. Somit
sollten die Funktionalitäten \enquote{Nach Zweck suchen} und \enquote{Set-Vorschläge}  in
zukünftigen Weiterentwicklungen priosiert werden.  Zudem wurde sich von Verleihenden das Ablehnen von
Reservierungen als Funktion gewünscht (vgl. \ref{sec:ev-ver}), während aufseiten der Ausleihenden
eine Filterfunktion in der Suche genannt wurde.

\section{Oberfläche}
Generell wurde die Oberfläche als übersichtlich und \enquote{clean} beschrieben
(EA1-EA4). Der Wunsch, die Anwendung im Universitätsalltag für Projekte
nutzen zu können, wurde entsprechend geäußert (EA1, EA3, EA4).

\begin{longtable}{p{0.85\textwidth}} \arrayrulecolor{maincolor}\hline
        \enquote{\textit{Ja, wäre cool, wenn wir das am IMIS wirklich nutzen
        könnten}}                                                      \\
        \enquote{\textit{[\dots] es ist insgesamt sehr übersichtlich und ich würde
        sagen, dass ich mich auf jeden Fall gut zurechtfinde [\dots]}} \\
        \enquote{\textit{Allgemein die Idee, dass es wie ein Online-Shop ist [\dots] ,
                        weil viele Leute einfach alles neu kaufen, statt es auszuleihen und wenn du
                        denen einen Online-Shop an die Hand gibts, können die einfach [\dots] danach
        suchen.}}                                                      \\
        \arrayrulecolor{maincolor}\hline
\end{longtable}

\section{Diskussion}
\label{sec:disc}
Die eingangs formulierte Forschungsfrage F3 konnte lediglich bedingt beantwortet werden, da das
positive Ergebnis der durchgeführten Studie nicht verglichen werden konnte. Eine weitere,
ausführlichere Studie mit einem Vergleichsystem wird somit empfohlen. Trotz des fehlenden Vergleichs
konnte die Arbeit ein qualitativ hochwertigen Prototypen hervorbringen, welches sich als
übersichtlich und unterstützend herausstellt. Für Forschungsfrage 2 konnten weitere Anforderungen an
den Prototypen herausgearbeitet werden. Die dargestellten Verbesserungsvorschläge sollten in das
Reservierungstool eingearbeitet werden, bevor die Erhebung einer weiteren Studie durchgeführt wird.
Zudem sollte eine weitere Analyse zu den verwendeten Begrifflichkeiten vorgenommen werden, um letzte
Verwirrungen zu beseitigen. Des Weiteren sollte eine Feldstudie in Betracht gezogen werden, um die
technische Performanz des Prototypen im realen Einsatz zu testen.

Insgesamt führten die anonyme Umfrage und Evaluation per Think-Aloud-Methode zu übereinstimmenden
Ergebnissen. Sowohl die Ergebnisse der Verleihenden als auch der Ausleihenden zeigen, dass der
Prototyp als Unterstützung wahrgenommen wurde. Aufseiten der Ausleihenden wurden alle
Funktionalitäten überwiegend als hilfreich eingestuft. Auch aus Sicht der Verleihenden war die
Nutzung des Systems überwiegend positiv eingestuft worden.
