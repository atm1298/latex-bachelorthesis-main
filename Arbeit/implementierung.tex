
\chapter{Implementierung}
\label{chapter-implementierung}
Das folgende Kapitel beschreibt die Implementierung des Backends, Reservierungsinterfaces sowie des
Fontends. Zunächst wird ... beschreiben. Daraufhin wird ... beschreiben. Abschließend wird auf die
Inbetriebnahme des Systems eingegangen.


\section{Implementierung des Backends}
Bild: Verzeichnisstruktur des Backends

\subsection{Implementierung der Kernfunktionalität}
Dieser Abschnitt präsentiert die Implementierung der Kernfunktionalität des Zwischenbackends,
welche aus den Anforderungen aus \ref{section:anforderung} bestimmt wurden. Bei der Funktionalität
handelt es sich um das Reservieren in die Zukunft, sowie das Speichern dieser Vorgänge und die
damit einhergehende Bestätigung für die Aktualisierung in Snipe-IT.


\subsection{Sackgassen}
LDAP hat Probleme aufgeworfen, weil:

\begin{itemize}
  \item API nur per Token nutzbar
  \item Token nur manuell im Dashboard generierbar
  \item
\end{itemize}
Statuseingabe/Ausleihen schwer und komisch -> rumarbeiten

\section{Implementierung des Frontends}

Bei der Implementierung wurde sich an den best practices der
Vue.js-Dokumentation orientiert \todo{(Vue.js, 2021a)}. Zunächst wurde für die
in \ref{subsection:system} Components eine eigene View erstellt.


\section{Nutzung des Systems}
Um das System Nutzen und weiterentwicklen zu können muss...
API KEy generieren lassen

\section{Fazit der Implementierung}