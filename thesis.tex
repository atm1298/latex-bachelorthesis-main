\documentclass{scrbook}

\input{Style/style}
\input{macros}

% Set title and author used in the PDF meta data
\hypersetup{
  pdftitle={SnipeIT Companion: Entwicklung eines Reservierungstools für Assets},
  pdfauthor={Anna-Tabea Manske}
}

% Depending on which of the following two color schemes you import your thesis will be in color or grayscale. I recommend to generate a colored version as a PDF and a grayscale version for printing.

%!TEX root = thesis.tex

% define color of example university
\xdefinecolor{exampleuniversity}{RGB}{0,62,75}


\colorlet{maincolor}{exampleuniversity}
%\colorlet{maincolor}{mycolor}

\colorlet{stringcolor}{green!60!black}
\colorlet{commentcolor}{maincolor!30}
\colorlet{keywordcolor}{maincolor!80!black}

\newcommand{\imagesuffix}{-color}
%\input{schema-gray}

\newcommand{\duedate}{XX. Oktober 2022}

\begin{document}
\frontmatter
%!TEX root = thesis.tex

\begin{titlepage}
  \centering
  \thispagestyle{empty}

  \vskip1cm

  \includegraphics[scale=0.8]{Bilder/imis-thesis-logo.pdf}

  \vskip2cm

  \LARGE

  \textbf{\sffamily\color{maincolor}Snipe-IT Companion: Entwicklung eines Reservierungstools für Assets}

  {\sffamily Snipe-IT Companion: Development of a Reservation Tool for Assets}

  \normalfont\normalsize

  \vskip2em

  \textbf{\sffamily\color{maincolor}Bachelorarbeit}

  im Rahmen des Studiengangs \\
  \textbf{\sffamily\color{maincolor}Medieninformatik} \\
  der Universität zu Lübeck

  \vskip1em

  vorgelegt von \\
  \textbf{\sffamily\color{maincolor}Anna-Tabea Manske}

  \vskip1em

  ausgegeben und betreut von \\
  \textbf{\sffamily\color{maincolor} Univ.-Prof. Dr. rer. nat. Hans-Christian Jetter}

  \vskip1em

  mit Unterstützung von\\
  Jan-Henrik Schröder, M.Sc.

  \vskip1em


  \vfill

  Lübeck, den \duedate
\end{titlepage}



%!TEX root = thesis.tex

\cleardoublepage
\thispagestyle{plain}

\pdfbookmark{Kurzfassung}{kurzfassung}
\section*{Kurzfassung}

Am \ac{imis} werden Assets ohne Anmeldung oder nur mit einer mündlichen
Absprache verwendet. Für die vorausschauende Planung von anstehenden Projekten
gibt es keine Reservierungsmöglichkeiten oder einen Überblick, wann die
benötigten Assets wieder verfügbar sind. Zudem ist vielen Mitarbeitenden und
Studierenden der Universität zu Lübeck unklar, welche Assets sich in den Laboren
des \ac{imis} befinden.

Das Ziel der vorliegenden Arbeit ist die Entwicklung eines wirksamen Systems,
welches den Reservierungs- und Ausleihprozess am \ac{imis} einheitlicher,
effizienter und zufriedenstellender gestaltet. Die Basis für das
Reservierungstool schafft die Asset-Managementsoftware \textit{Snipe-IT}, welche
bereits am \ac{imis} eingesetzt wird.

Für die Entwicklung eines Asset-Reservierungstools wurden Forschungsfragen
nachgegangen, welche zentralen Schwierigkeiten die aktuelle Planung und
Reservierung von Assets für Mitarbeitende und Studierende mit sich bringen und
welche Anforderungen ein System mitbringen sollte, um diese zu adressieren und
reduzieren. Dazu werden in der vorliegenden Arbeit systematisch, standardisierte
Anforderungen an ein solches Reservierungstool auf Basis von Nutzenden, Aufgaben
und Kontext entwickelt und in einer formalisierten Anforderungsanalyse erfasst.
Teile des Konzeptes werden in Form eines technologisch reifen Prototyps
realisiert. Die Evaluation des Prototypens konnte aufzeigen, dass ein solches
System als Unterstützung wahrgenommen wird. Abschließend wird die
Weiterentwicklung des Prototyps und Folgestudien zur Evaluation der Anwendung
vorgeschlagen und diskutiert.

\subsection*{Schlüsselwörter}
Mensch-Computer-Interaktion, Reservieren, Assetmanagement, Snipe-IT, Dashboard,
Web-App, Framework

\cleardoublepage
\thispagestyle{plain}

\foreignlanguage{english}{%
  \pdfbookmark{Abstract}{abstract}
  \section*{Abstract}


  \subsection*{Keywords}
  Human-Computer Interaction, Reservation, Asset Management, Snipe-IT,
  Dashboard, Web App, Framework

}

\cleardoublepage
\phantomsection
\pdfbookmark{\contentsname}{tableofcontents}
\markboth{\contentsname}{}
\tableofcontents

\mainmatter
%!TEX root = thesis.tex

\chapter{Einleitung}

Das Ausleihen von Assets jeglicher Art ist keine Neuheit \todo{Söderholm}. An der Zentralen
Hochschulbibliothek Lübeck (ZHB) wird die Buchungsapp Affluences verwendet. Zum Ausleihen von
Materialien, müssen Terminabholungen online gebucht werden \todo{(ZHB, o. J.)}. Die Anfragen können
überprüft werden, wodurch das vorausschauende Planen der Materialien ermöglicht wird. Am \ac{imis}
werden Assets ohne Anmeldung oder mit einer mündlichen Absprache verwendet. Aufseiten der
Mitarbeitenden wird der Gebrauch der Assets individuell geplant, sodass das frühzeitige Reservieren
oder das geplante Ausleihen erschwert wird. Für die vorausschauende Planung von anstehenden
Projekten gibt es keinerlei feste Reservierung oder einen Überblick, wann die gewünschten Assets
wieder verfügbar sind. Zudem ist vielen Mitarbeitenden unklar, welche Assets sich in den Laboren des
\ac{imis} befinden. Folglich kennen Mitarbeitende nur selten die Möglichkeiten, mit denen sie ihre
Forschungsprojekte oder die Lehre ergänzen könnten.

Aktuell werden Reservierungen sowie der Gebrauch von Equipment über unterschiedliche
Kommunikationswege wie E-Mail oder direkte Absprache bei zuständigen Mitarbeitenden angefragt. Die
zuständigen Mitarbeitenden prüfen die Anfrage und koordinieren potenzielle Kollisionen mit bereits
reservierten Zeiträumen oder Absprachen und bestätigen, ändern oder lehnen die
(Reservierungs-)Anfrage ab. In einigen Fällen werden die gebuchten Zeiten auf Papier dokumentiert.
Aufseiten der Studierenden kann ein solches System für Gruppenarbeiten, wie zum Beispiel die des Moduls
„Einführung in die Medieninformatik“ hilfreich sein. Studierende müssen während dieses Projekts eine
multimediale Abgabe produzieren, welche beispielsweise die Form eines Videos haben kann. Aktuell ist
vielen Studierenden nicht bewusst, dass das \ac{imis} Equipment wie Kameras, Greenscreen oder Gimbal für
Videos bereitstellt. Des Weiteren sind Hemmschwelle und Aufwand zum Ausleihen der Hardware hoch, da
uneindeutig ist, welche Mitarbeitenden für die jeweilige Hardware zuständig sind.


\section{Ziel der Arbeit}
Das Ziel dieser Arbeit ist es ein wirksames System zu entwickeln, welches den Reservierungs- und
Entleihprozess am \ac{imis} einheitlicher, effizienter und zufriedenstellender lösen lässt. Die
Anwendung soll es unter anderem ermöglichen, die auszuleihenden Assets in einer Liste abzubilden und
diese zu durchsuchen. In diesem Zusammenhang soll auch das Anzeigen einer Stückzahl der jeweils
verfügbaren Assets, deren Bedienungsanleitung sowie die verantwortlichen Mitarbeitenden umgesetzt
werden. Das Reservierungstool soll eine niedrigschwellige Möglichkeit bieten, um Mitarbeitenden die
Arbeit im Reservierungsprozess zu erleichtern. Außerdem soll es eine Übersicht über Utensilien
geben, sodass Mitarbeitende nicht mehr im Unklaren über die Hardware-Möglichkeiten sind und das
Material optimal ausgelastet werden kann.

Die Basis für ein solches Tool schafft die Asset Managementsoftware \textit{Snipe-IT}
\cite{noauthor_home_nodate}, welche bereits am \ac{imis} eingesetzt wird. \textit{Snipe-IT} ist eine
kostenlose, quelloffene IT-Asset-Verwaltungs-Plattform, welche das Nachverfolgen von
Software-Lizenzen, Hardware und Verbrauchsgegenständen ermöglicht. Genannte Assets können über ein
Dashboard hinzugefügt, verwaltet und gelöscht werden. Über Labels können Assets zur
Übersichtlichkeit in verschiedene Kategorien eingeordnet werden, während Tags ein Asset eindeutig
identifizieren (z. B. Seriennummer). Zudem ermöglicht das „Checkin/Checkout“-System die
Nachverfolgung aller Assets, falls diese zum Beispiel an Person ausgeliehen werden. Zu jedem
Zeitpunkt kann ein Asset maximal einer Person zugeordnet werden, wodurch das mehrfache gleichzeitige
Ausleihen eines Assets verhindert wird. Da \textit{Snipe-IT} selbst die zukünftige Reservierung
nicht unterstützt, umfasst das Ziel der Arbeit ein \textit{Snipe-IT Companion}, welcher das
Ausleihen in die Zukunft möglich macht.


\section{Forschungsfragen}
Im Sinne der eingangs beschriebenen Ziele soll im Rahmen der vorliegenden Arbeit untersucht werden,
wie ein Reservierungstool gestaltet werden kann, um Mitarbeitenden und Studierenden darin zu
unterstützen, Assets einsehen und ausleihen zu können.

Um den aktuellen Stand und die Probleme des aktuellen Ausleihprozess nachvollziehen zu können,
müssen diese zunächst ermittelt und klassifiziert werden. Die erste Forschungsfrage beschäftigt sich
daher mit der Analyse der Schwierigkeit des Ausleihprozesses.
\begin{enumerate}
  \item[\sffamily\color{maincolor} {F1 |}] {Welche zentralen Schwierigkeiten bringt die aktuelle Planung und Reservierung von Assets für Mitarbeitende und Studierende mit sich?}
\end{enumerate}

Um die zentralen Schwierigkeiten und Probleme lösen zu können, müssen Anforderungen an das System
ermittelt werden. Anschließend müssen diese nach Relevanz sortiert werden, um möglichst viele
Schwierigkeiten lösen zu können. Die zweite Forschungsfrage beschäftigt sich folglich mit den
Anforderungen, welche ein solches Reservierungstool umfassen sollte, um die Schwierigkeiten zu minimieren.

\begin{enumerate}
  \item[\sffamily\color{maincolor} {F2 |}] {Was sind Anforderungen an ein System, welches die in F1 gezeigten Schwierigkeiten adressiert und reduziert?}
\end{enumerate}

Abschließend soll eruiert werden, ob die erste Iteration des Systems die ermittelten Schwierigkeiten
mit den erarbeiteten Funktionalitäten lösen kann. Die letzte Forschungsfrage beschäftigt sich
demzufolge mit der Evaluation des erarbeiteten Systems und den möglichen stärken und schwächen.

\begin{enumerate}
  \item[\sffamily\color{maincolor} {F3 |}] {Inwieweit kann ein aus F2 resultierender Prototyp die in F1 identifizierten Schwierigkeit reduzieren?}
\end{enumerate}

\section{Vorgehensweise}
Die Entwicklung des Systems orientiert sich am menschenzentrierten Gestaltungsprozess
\cite{din_en_iso_9421-2102020-03_din_nodate}. Der Prozess teilt sich im Rahmen dieser Arbeit in
fünf aufeinanderfolgende Phasen {\ref{fig:schablone}} wobei die Entwurfs- und Implementierungsphasen
Spielraum für ein iteratives Vorgehen lassen. Unter anderem werden in der Analyse die Aufgaben des
Systems, die Benutzenden und der Kontext nach dem Entwicklungsprozess für interaktive Medien
\cite{herczeg_einfuhrung_2009} aufgeführt, um ein gebrauchstaugliches Ergebnis erzielen zu können.

\begin{figure}[h]
  \centering
  \includegraphics[scale=0.6]{Bilder/Vorgehensmodell.pptx.pdf}
  \label{fig:schablone}
  \caption[Vorgehensmodell]{Vorgehensmodell}
\end{figure}

In der ersten Phase wurden die Anforderungen an das zu entwickelnde System erarbeitet und
verstanden. Die Erkenntnisse dieser Phase sind in Kapitel \ref{chapter-analyse} zu finden.
\ref{chapter-analyse} beantwortet unter anderem auch die Forschungsfragen F1 und F2. Durch die
Interviews mit den Stackholdern konnten die zentralen Schwierigkeiten des Ausleihprozesses
festgestellt werden. Foglich wurden die Notwendigkeiten der angedachten Anforderungen mithilfe der
Interviews überprüft und ergänzt. 

In der Spezifikationsphase werden die Anforderung an das System weiter spezialisiert
(\ref{chapter-konzept}). Daher wurden Funktionalitäten entsprechend den Anforderungen entwickelt und
in einer priorisierten Featureliste festgehalten. Anschließend wurde die Systemarchitektur, für das
Back-end, aufbauend auf den Anforderungen und damit einhergehenden Frameworks ermittelt. Darüber
hinaus wurde für das Front-end in der Entwurfsphase das Interface-Design erarbeitet
(\ref{chapter-design}). Hierbei wurde durch Usability Tests ein iteratives Vorgehen ermöglicht.

Die Implementierungsphasen umfasst die eigentliche Umsetzung des Reservierungstools
(\ref{chapter-implementierung}). Hierbei wurden die in der Konzeptionsphase festgelegten Frameworks
genutzt. \ref{chapter-dialogbeispiel} präsentiert das realisierte System anhand von
Dialogbeispielen.

In der abschließenden Phase wurde das realisierte System mithilfe von Interviews und Umfragen
evaluiert (\ref{chapter-evaluation}). Die Ergebnisse der Phase beantworten Forschungsfrage F3 und
geben Aufschluss über die Wirksamkeit des entwickelten Systems. Im Anschluss gibt
\ref{chapter-fazit} einige Perspektiven über offene Punkte und die mögliche Weiterentwicklung des
Systems. 
%!TEX root = thesis.tex

\chapter{Analyse}
\label{chapter-analyse}

Dieses Kapitel beschreibt alle für die Arbeit notwendigen Grundlagen.

\section{Datenquellen}
\label{section:daten}
Für die Analyse wurden Stakeholder-Interviews durchgeführt. Die Interviews fanden im Mai 2022
auf dem Universitätsgelände oder online statt. Die Befragten wurden durch ein qualitatives,
semi-strukturiertes Interview geführt. Im Vorfeld wurde dafür ein Interviewleitfaden entwickelt
(Anhang Verlinken). Es wurde eine Unterteilung in Verleihende und Ausleihende von Assets vorgenommen
(genaue Definition der Benutzergruppen in Abschnitt
\ref{section:benutzer}). Bei den Teilnehmenden handelt es sich um Mitarbeitende und Studierende der
Universität zu Lübeck, welche am \ac{imis} tätig sind. In \ref{table:e} sind die Rollen der befragten
Verleihenden aufgeführt. Die Verleihenden der Assets können gleichzeitig auch die Rolle der
Ausleihende einnehmen. Die Rollen der befragten Ausleihende sind in \ref{table:b} dargestellt. Die IDs
der Tabellen werden als Verweise in den folgenden Abschnitten verwendet.


\begin{table}[h]
    \centering
    \begin{zebratabular}{ll}
        \headerrow ID & Rolle \\
            V1 & Professor\\
            V2 & Wissenschaftlicher Mitarbeiter\\
            V3 & Wissenschaftlicher Mitarbeiter \\
            V4 & Sekretariat und administratives Personal\\
    \end{zebratabular}  
    \caption{Teilnehmende der Interviews, Verleihende}
    \label{table:e}
\end{table}

\begin{table}[h]
    \centering
    \begin{zebratabular}{ll}
        \headerrow ID & Rolle \\
            A1  & Bachelorstudent und Hilfswissenschaftler\\
            A2 & Bachelorstudent\\
            A3  & Masterstudent und Hilfswissenschaftlerin\\
    \end{zebratabular}
    \caption{Teilnehmende der Interviews, Ausleihende}
    \label{table:b}
    \hfill
    
\end{table}

\section{Benutzeranalyse}
\label{section:benutzer}

\section{Aufgabenanalyse}
\label{section:aufgaben}

\section{Kontextanalyse}
\label{section:kontext}

\section{Analyse des aktuellen Stands}
\label{section:iststand}

Snipe-IT ist eine kostenlose, quelloffene IT-Asset-Verwaltungs-Plattform,
welche das Nachverfolgen von Software-Lizenzen, Hardware und
Verbrauchsgegenständen ermöglicht. Genannte Assets können über ein Dashboard
hinzugefügt, verwaltet und gelöscht werden. Über Labels können Assets zur
Übersichtlichkeit in verschiedene Kategorien eingeordnet werden, während
Tags ein Asset eindeutig identifizieren (z. B. Seriennummer). Zudem ermöglicht
das „Checkin/Checkout“-System die Nachverfolgung aller Assets, falls diese
z. B.  an Person ausgeliehen werden. Zu jedem Zeitpunkt kann ein Asset
maximal einer Person zugeordnet werden, wodurch das mehrfache gleichzeitige
Ausleihen eines Assets verhindert wird. Darüber hinaus beschreiben Status-Label
den Zustands eines Assets und ob dieses ausgeliehen werden kann. Alle
Funktionalitäten können zudem über eine REST-API programmatisch genutzt werden.


\section{Formalisierte Anforderungen}
\label{section:anforderung}

Im Folgenden werden systematisch formalisierte Anforderungen präsentiert, welche die Ergebnisse der Analysen abschließend zusammenfassen.
Es werden zunächst die Visionen und Ziele (\ref{section:visionziel}) definiert, des Weiteren werden
die Rahmenbedingungen (\ref{section:rahmen}) und der Kontext des Systems
(\ref{section:kontextueberblick}) dargestellt. Darauf aufbauend wird eine funktionale Anforderung
erstellt (\ref{section:funktionale}). Abschließen werden die Qualitätsanforderungen formuliert
(\ref{section:qualität}).


\subsection*{Vision und Ziele}
\label{section:visionziel}
Zunächst sollten die Visionen und Ziele des Systems konkretisiert werden, an denen sich die
Anforderungen auf Zielgerichtetheit überprüfen lassen \cite{balzert2009}. Diese setzen sich aus der
Analyse der Benutzenden sowie Aufgaben und des Kontextes zusammen. Im ersten Schritt werden die
Visionen für die Zukunft realitätsnah festgelegt.



\begin{center}
        \renewcommand{\arraystretch}{1.5}
        \begin{tabular}{p{0.1\textwidth}p{0.8\textwidth}}
                \hline
                \textbf{/V10/} & Verleihende des Status sind in der Lage,                                          \\
                \textbf{/V20/} & Ausleihende sind besser dazu in der Lage,                                         \\
                \hline
        \end{tabular}
\end{center}

Basierend auf diese Visionen lassen sich die Ziele formulieren, welche die Visionen
operationalisieren. Diese folgen dabei den standardisierten Regeln zur Formulierung von Zielen
\cite{pohl_requirements_2008}.


\begin{center}
        \renewcommand{\arraystretch}{1.5}
        \begin{tabular}{p{0.1\textwidth}p{0.8\textwidth}}
                \hline
                \textbf{/Z10/} & Ausleihende eines Assets erhalten zielgerichtete und aktuelle
                Informationen zum Verbleib der.                                                                          \\
                \textbf{/Z20/} & XX sind jederzeit in der Lage, ihre Informationen zu ändern.                                       \\
                \textbf{/Z40/} & XX sind überall in der Lage, ihre Informationen zu ändern.                                         \\
                \textbf{/Z50/} & Der Status ist durch XX leicht und unkompliziert veränderbar.                                      \\
                \hline
        \end{tabular}
\end{center}

\subsection*{Rahmenbedingungen}
\label{section:rahmen}
Die Randbedingungen legen organisatorische und technische Restriktionen für das System oder den
Entwicklungsprozess fest \cite{balzert2009}. Die Bedingungen wurden aus dem Lastenheft und der
Benutzer- und Kontextanalyse abgeleitet.

\begin{center}
        \renewcommand{\arraystretch}{1.5}
        \begin{tabular}{p{0.1\textwidth}p{0.8\textwidth}}
                \hline
                \textbf{/R10/} & Das System ist eine Web-Anwendung.                                                        \\
                \textbf{/R20/} & Die Zielgruppe sind Mitarbeitende des IMIS und Studierende.                               \\
                \textbf{/R30/} & Die Zielgruppe teilt sich in zwei Nutzergruppen: die Verleihenden und
                Ausleihende von Assets. Die Definitionen der Nutzergruppen sind in Kapitel (\secref{section:benutzer})
                zu finden.                                                                                                 \\
                \textbf{/R40/} & Das System wird von Verleihenden sowohl im mobilen als auch im Arbeitskontext genutzt (). \\
                \textbf{/R50/} & Die eingesetzte Software auf der Zielmaschine ist clientseitig ein
                Webbrowser. Die marktführenden Webbrowser müssen unterstützt werden: Chrome, Firefox,
                Safari \cite{noauthor_browser_nodate}.                                                                     \\
                \hline
        \end{tabular}
\end{center}

\subsection*{Kontext und Überblick}
\label{section:kontextueberblick}
Ein System ist in einer technischen Umgebung eingebettet \cite{balzert2009}. Es wurde im Folgenden Bezug auf das aktuelle Vorgehen und die Schnittstellen des System genommen.

\begin{center}
        \renewcommand{\arraystretch}{1.5}
        \begin{tabular}{p{0.1\textwidth}p{0.8\textwidth}}
                \hline
                \textbf{/K10/} & Das aktuelle Vorgehen umfasst Klebezettel an den Türen der Mitarbeitenden.                                                                  \\
                \textbf{/K20/} & Es existiert eine Schnittstelle zum Back-End des Labormanagementsystems.                                                                    \\
                \textbf{/K30/} & Es existiert ein digitaler Kalender, auf den die Mitarbeitenden Zugriff haben, in dem geblockte Zeiten einzelner Personen angezeigt werden. \\
                \hline
        \end{tabular}
\end{center}

\subsection*{Funktionale Anforderungen}
\label{section:funktionale}
Im Folgenden werden die Kernfunktionalitäten des Systems aufgeführt \cite{balzert2009}.
Diese ergeben sich aus der Aufgabenanalyse (\ref{section:aufgaben}). Um die Anforderungen mit einer eindeutigen Semantik zu formulieren, wurde eine Anforderungsschablone (\ref{fig:schablone}) verwendet, um natürlichsprachliche Anforderungen zu definieren \cite{balzert2009}.

\begin{figure}[h]
        \centering
        \includegraphics[scale=0.45]{Bilder/anforderungsschablone.pdf}
        \label{fig:schablone}
        \caption[Anforderungsschablone]{Anforderungsschablone \cite{balzert2009}}
\end{figure}

\begin{center}
        \renewcommand{\arraystretch}{1.5}
        \begin{tabular}{p{0.1\textwidth}p{0.8\textwidth}}
                \hline
                \textbf{/F10/}  & Das System \textit{muss} XX die Möglichkeit bieten, Informationen jederzeit einzutragen (A).                             \\
                \textbf{/F20/}  & Das System \textit{muss} XX die Möglichkeit bieten, Status wiederzuverwenden.                                            \\
                \textbf{/F30/}  & Das System \textit{muss} XX die Möglichkeit bieten, ein Profil zu erstellen (A).                                         \\
                \textbf{/F40/}  & Das System \textit{muss} XXn relevante Informationen anzeigen, welche vom Erstellenden hinterlassen werden sollen (A).   \\
                \textbf{/F50/}  & Das System \textit{muss} XX die Möglichkeit geben, den alleinigen Zugriff auf die eigenen Daten zu haben.                \\
                \textbf{/F60/}  & Das System \textit{soll} XX die Möglichkeit bieten, individuelle und personalisierte Inhalte zu visualisieren (A).       \\
                \textbf{/F70/}  & Das System \textit{soll} daran erinnern, die Informationen zu aktualisieren (A).                                                   \\
                \textbf{/F80/}  & Das System \textit{soll} XX die Möglichkeit bieten, sich eine Vorschau ihres aktuell angezeigten Türschilds anzuschauen. \\
                \textbf{/F90/}  & Das System \textit{sollte in Zukunft} XX.                                     \\
                \textbf{/F100/} & Das System \textit{sollte in Zukunft} XX die Möglichkeit bieten, Informationen zu hinterlassen (A).                     \\
                \textbf{/F110/} & Das System \textit{sollte in Zukunft} den angezeigten Status automatisch aus den Daten des digitalen Kalenders ermitteln können.   \\
                \hline
        \end{tabular}
\end{center}


\subsection*{Qualitätsanforderungen}
\label{section:qualität}
Im letzten Schritt werden die nicht-funktionalen Anforderungen festgelegt, welche die qualitativen oder quantitativen Eigenschaften eines Systems darstellen \cite{balzert2009}. Auch hier wird, falls möglich, die Anforderungsschablone aus \ref{fig:schablone} verwendet.

\begin{center}
        \renewcommand{\arraystretch}{1.5}
        \begin{tabular}{p{0.1\textwidth}p{0.8\textwidth}}
                \hline
                \textbf{/Q10/} & Das System \textit{muss} den Grundsätzen der DIN EN ISO
                9241-110:2019-09 (Ergonomie der Mensch-System-Interaktion - Teil 110:
                Interaktionsprinzipien) folgen (\textit{DIN EN ISO 9241-110}, 2019).                                                                                                                \\
                \textbf{/Q20/} & Das System \textit{muss} die definierten Nutzungsklassen aus Kapitel
                section:benutzer (Sonderrolle) unterscheiden und die dazugehörigen Zugriffsrechte
                sicherstellen.                                                                                                                                                                      \\
                \textbf{/Q30/} & Das System \textit{soll} modular strukturiert sein, damit Inhalte und Funktionalitäten effizient eingebunden werden können und das System einfach erweiterbar ist. \\
                \textbf{/Q40/} & Das System \textit{soll} beim Zugriff über das Internet eine gesicherte Übertragung (bspw. \ac{HTTPS}) ermöglichen.                                                     \\
                \textbf{/Q50/} & Das System \textit{soll} alle Benutzerinteraktionen in unter fünf Sekunden ausführen.                                                                              \\
                \hline
        \end{tabular}
\end{center}

%Wie eingangs er-wähnt, definieren die Anforderungen, was das System zu
%leisten hat, während die Funktionalitä-ten definieren, wie das System diese gewährleistet.
\chapter{Konzeption}
\label{chapter-konzept}

In diesem Kapitel wird die eigentliche Erkenntnis dieser Arbeit beschrieben. Der Aufbau dieses
Kapitels hängt stark vom Thema der Arbeit ab. Die in dieser Vorlage vorgeschlagenen Kapitel sind
auch nur als Vorschlag und auf keinen Fall als verbindlich zu verstehen.

\section{Funktionalität}
% Funktionalitäten für Verleihende und Ausleihende
\subsection{Funktionalitäten für Verleihende und Ausleihende}
Im Folgenden werden Funktionalitäten, welche für Verleihende und Ausleihenden von Bedeutung sind
näher erläutert (\ref{table:ft-va}).

\begin{table}[h]
    \centering
    \caption{Funktionalitäten für Verleihende und Ausleihende}
    \begin{longtable}{lll}
        \arrayrulecolor{maincolor}\hline
        \sffamily\color{maincolor}ID & \sffamily\color{maincolor}Titel        &
        \sffamily\color{maincolor}Anforderungen                                                                           \\
        \arrayrulecolor{maincolor}\hline
        Ft-VA-1                      & Übersicht über ausleihbare Assets      & \anfref{V20}
        \anfref{Z20} \anfref{F50} \anfref{K10} \anfref{F10} \anfref{F30}                                                  \\
        Ft-VA-2                      & Benachrichtigungen \& Erinnerungen     & \anfref{F100} \anfref{F110} \anfref{F120} \\
        Ft-VA-3                      & Sichtbarkeit von Ansprechpartner:innen & \anfref{F50}                              \\
        Ft-VA-4                      & Authentifizierung per IDM Account      & \anfref{F70} \anfref{F80}                 \\
        \arrayrulecolor{maincolor}\hline
    \end{longtable}
    \label{table:ft-va}
\end{table}

{\sffamily\color{maincolor}{Ft-VA-1 | Übersicht über ausleihbare Assets }}\\
Die Übersicht, der am IMIS vorhandenen Assets, wurde mittels
Kategorien umgesetzt. Dazu gibt es eine Übersicht, bei der alle Assets
eingesehen werden können. Die einzelnen Kategorien beeinhalten Unterkategorien.
In der Übersicht werden Informationen, wie Name, Seriennummer, Marke und Status
eines Assets angezeigt. Wie bereits in der Problemanalyse geschildet
(\ref{section:probleme-allgemein}) gibt es keine Übersicht, über ausleihbare
Assets. Dies zeigt die Dringlichkeit des SnipeIT Companion für eine besser
Vorbereitung.

    {\sffamily\color{maincolor}{Ft-VA-2 | Benachrichtigungen \& Erinnerungen   }}\\
Verleihende werden nach dem Abschluss eines von ihnen verantwortlichen Assets
benachrichtigt. Die Benachrichtigung umfasst Informationen, wer das Asset, wann
rserviert hat, und wann die Abholung für das Assets stattfinden soll. Außerdem
wird es eine direkte weiterleitungs möglichkeit geben, sollten Verleihenden nich
können. Deweiterne Erhalten Ausleihende Erinnerungen, sobald die Abholung und
Rückgabe eines Assets ansteht.

    {\sffamily\color{maincolor}{Ft-VA-3 | Sichtbarkeit von Ansprechpartner:innen}}\\
Für Rückfragen zu einem Asset sind Kontaktinformation zu Ansprechpartner:innen
(Verleihende) sowie deren E-Mail-Adresse direkt ersichtlich.

Nachdem der Checkout stattgefunden hat, erhalten Verleihende eine
Zusammenfassung über die Ausleihdaten und eine  Hinweis, wann die Abholung
stattfindet. Außerdem werden Kontaktinformation des Verleihenden angezeigt
(Name, E-Mail und Lageort des Assets).


    {\sffamily\color{maincolor}{Ft-VA-4 | Authentifizierung per IDM Account}}\\
Mit hilfe einer Verknüpfung über den LDAP Account der Universität zu Lübeck,
können sich Nutzende mit ihrem bereits existierenden IDM Account einloggen, so
muss kein Neues Konto erstellt werden und die Zuständigkeit, sowie sicherheit,
wer sich einloggt ist gegeben.

\subsection{Funktionalitäten für Verleihende}
Im Folgenden werden Funktionalitäten, welche für Verleihende von Bedeutung sind
näher erläutert (\ref{table:ft-v}).

\begin{table}[h]
    \centering
    \caption{Funktionalitäten für (V)erleihenden}
    \begin{longtable}{lll}
        \arrayrulecolor{maincolor}\hline
        \sffamily\color{maincolor}ID & \sffamily\color{maincolor}Titel      &
        \sffamily\color{maincolor}Anforderungen                                             \\
        \arrayrulecolor{maincolor}\hline
        Ft-V-1                       & Kalenderübersicht                    & \anfref{V20}
        \anfref{Z20} \anfref{F50} \anfref{K10} \anfref{F10} \anfref{F30}                    \\
        Ft-V-2                       & Zuständigkeitsbereich                & \anfref{F50}  \\
        Ft-V-3                       & Verfügbarkeit von Assets             & \anfref{F70}  \\
        Ft-V-4                       & Bearbeiten des Assetstatus           & \anfref{F150} \\
        Ft-V-5                       & Pflege von Assets                    & \anfref{F130} \\
        Ft-V-6                       & Pflege der Datenbank/Übersicht/Liste & \anfref{F140} \\
        \arrayrulecolor{maincolor}\hline
    \end{longtable}
    \label{table:ft-v}
\end{table}

{\sffamily\color{maincolor}{Ft-V-1 | Kalenderübersicht }}\\


{\sffamily\color{maincolor}{Ft-V-2 | Zuständigkeitsbereich }}\\


{\sffamily\color{maincolor}{Ft-V-3 | Verfügbarkeit von Assets  }}\\


{\sffamily\color{maincolor}{Ft-V-4 | Bearbeiten des Assetstatu }}\\


{\sffamily\color{maincolor}{Ft-V-5 | Pflege von Assets   }}\\


{\sffamily\color{maincolor}{Ft-V-6 | Pflege der Datenbank/Übersicht/Liste }}\\


\subsection{Funktionalitäten für Ausleihende}
Im Folgenden werden Funktionalitäten, welche für Ausleihenden von Bedeutung sind
näher erläutert (\ref{table:ft-A}).


\begin{table}[h]
    \centering
    \caption{Funktionalitäten für (A)usleihenden}
    \begin{longtable}{lll}
        \arrayrulecolor{maincolor}\hline
        \sffamily\color{maincolor}ID & \sffamily\color{maincolor}Titel &
        \sffamily\color{maincolor}Anforderungen                                                                              \\
        \arrayrulecolor{maincolor}\hline
        Ft-A-1                       & Material-Suche                  & \anfref{V20}
        \anfref{Z20} \anfref{K10} \anfref{F10} \anfref{F30}                                                                  \\
        Ft-A-2                       & Filtern und Sortieren           & \anfref{V30} \anfref{F30} \anfref{F70}              \\
        Ft-A-3                       & Ausleihhilfe/Nutzensuche        & \anfref{V30}\anfref{F70}                            \\
        Ft-A-4                       & Startseite                      & \anfref{F60}                                        \\
        Ft-A-5                       & Reservierungs-Checkout          & \anfref{F60} \anfref{F150}                          \\
        Ft-A-6                       & Detailansicht                   & \anfref{V50} \anfref{Z30} \anfref{F40} \anfref{F50} \\
        Ft-A-7                       & Rückgabe-Checkliste             & ???                                                 \\
        \arrayrulecolor{maincolor}\hline
    \end{longtable}
    \label{table:ft-A}
\end{table}

{\sffamily\color{maincolor}{Ft-A-1 | Material-Suche }}\\
Die Material-Suche dient zum einen einer simplen aber unterstütztende
Kategorieneinteilung, zum anderen für das gezielte Suchen nach Verfügbarkeiten
(verfügbar, Nicht verfügbar und Hinweis). Das gezielte Suchen nach Verfügbarkeit
wird durch die AUfforderung den Ausleihzeitraum auszuwählen ermöglicht.
Daraufhin gibt es die Möglichkeit gewünschtest Material in einem Suchfeld
einzugeben oder über die Kategorien nach dem Asset zu suchen. Das Suchenfeld
gibt bereits beim eintippen Vorschläge. Die Vorschläge können das Assets direkt,
die Kategorie oder ... sein. Bei mehreren Assets gibt die Suche mithilfe einer
Aufzählung alle Elemente an, die bereits afür die Reservierung ausgewählt
wurden.

    {\sffamily\color{maincolor}{Ft-A-2 | Filtern und Sortieren }}\\
Um das Finden für Assets leichter zu gestalten sollen Nutzende sets nach
Kategorie, Nutzen und Verfügbarkeit filtern können. Außerdem ist das sortieren
von A-Z oder Z-A, sowie nach den Verfügbarkeiten möglich.

{\sffamily\color{maincolor}{Ft-A-3 | Ausleihhilfe/Nutzensuche  }}\\
Die Ausleihhilfe ist eine vereinfachte Form eines Assistenten. Die Suche nach
Nutzen, wie Videodreh mit Ton und draußen, soll dem Nutzenden Vorschläge geben
können. Dies soll insbesondere Ausleihende unterscützen, die in den bereichen
noch nicht viel Expertise mitbringen können. Außerdem soll es Verleihenden den
Ausleihprozess erleichtern, weil das Beratunggespräch (Aufgabe Ref) wegfallen
kann.

    {\sffamily\color{maincolor}{Ft-A-4 | Startseite }}\\
Die Startseite soll nutzenden helfen, eine  Überblick zu erlangen. Für
erstnutzende, sind hinweise für die Materal-Suche und Ausleihhilfe in Form von
Button gegebn. Für Nutzende, die bereits etwas ausgeliehen haben, wird eine
Übersicht über Ausleihe und vergangenes Material angezeigt. wichtige
Informationen, wie der Zeitraum werden direkt auf einen Blick ersichtlich.

    {\sffamily\color{maincolor}{Ft-A-5 | Reservierungs-Checkout }}\\
Mit Hilfe des Reservierungs-Checkouts können alle ausgewählten Assets im
Überblick eingesehen werdenn. Außerdem werden alle Ausleihdaten, wie Zeitraum
der Ausleihe, Abholung und Rückgabe aufgeführt. Desweiteren sollte es die
Möglichkeit, alle Ausleihdaten bearbeiten zu können, sollte ein Datum oder eine
Uhrzeit nicht passen. Abschließend werden die Regeln und SIcherheitshinwweise
aufgeführt. Mithilfe einer AGB Bla erklärung, gilt die Materialausleihe als
vertraglich abgeschlossen.

    {\sffamily\color{maincolor}{Ft-A-6 | Detailsansicht }}\\
In der Detailansicht werden die Assets und deren Eigenschaften dargestellt.
Hierbei werden Informationen wie: Name, Seriennummer, Artikelbeschreibung,
Technischedetaisl, Kontaktinformation der Verleihenden, ... dargestellt.
Außerdem wird hier die Möglichkeit gegebn, Den Ausleihzeitraum einzustellen und
einzusehen, wann ein Asset verfügbar ist. In Form eines Button wird sichtbar,
dass das Asset zur Ausleihe hinzugefügt werden kann.

    {\sffamily\color{maincolor}{Ft-A-7 | Rückgabe-Checkliste}}\\
Bevor die ausgeliehenen Assets an Verlehende zurückgeben werden, wird eine
Checkliste für das jeweilige Asset angezeigt, bei der Hinweise stehen wie:
SD-Karte geleert, Assets auf Ursprungseinstellungen zurückgestellt, Akkus
geladen. Diese Funktionalität soll insbesondere dafür sorgen, dass....

%Frameworks
\section{Frameworks}
\label{section:frameworks}
Die Frameworkwahl nimmt, durch die unterschiedlichen Arbeitsweisen und Funktionen
der Frameworks, enormen Einfluss auf den Entwurf eines Systems und wird daher im
folgenden näher erläutert. Zunächst wird auf die Anforderungen der vorliegenden
Arbeit und die damit einhergehenden relevanten Anforderungen an die Frameworks
eingeggangen. Resultierend daraus, werden die nutzenden Frameworks definiert.

\subsection{Relevante Anforderung an ein Framework}
Die Grundlage der Auswahl, der im Rahmen dieser Arbeit eingesetzten Frameworks,
bilden die eingangs beschriebenen Anforderungen (\ref{section:anforderung}). Dem
System wird vorausgesetzt, dass es sich um eine Web-Anwendung mit Fokus auf den
Einsatz im mobilen Kontext (\anfref{R10}\anfref{R40}). Für Nutzende ist es
wichtig, dass das System dauerhaft erreichbar ist (\anfref{R50}). Aus
funktionaler Sicht müssen die Frameworks eine Unterstützung für progressive
Web-Applikationen bieten. Folglich ist auch eine Unterstützung für HTTPS
notwendig (\anfref{Q50}). Außerdem sollte es einfache Möglichkeiten zur
Verknüpfung von LDAP bieten (\anfref{K10} \anfref{F90}).

\subsection{Wahl der genutzten Frameworks}
Aufbauend auf den Anforderungen und der am \ac{imis} bereits eingesetzen Asset
Managementsoftware \textit{Snipe-IT} werden im folgen die gewählten Frameworks
erläutert.

\subsubsection{Asset Managementsoftware Snipe-IT}
Die Basis für das in dieser Arbeit umgesetzte System schafft die Asset
Managementsoftware \textit{Snipe-IT} \cite{noauthor_home_nodate}, welche bereits
am \ac{imis} eingesetzt wird. \textit{Snipe-IT} ist eine kostenlose, quelloffene
IT-Asset-Verwaltungs-Plattform, welche das Nachverfolgen von Software-Lizenzen,
Hardware und Verbrauchsgegenständen ermöglicht. Genannte Assets können über ein
Dashboard hinzugefügt, verwaltet und gelöscht werden. Über Labels können Assets
zur Übersichtlichkeit in verschiedene Kategorien eingeordnet werden, während
Tags ein Asset eindeutig identifizieren (z. B. Seriennummer). Zudem ermöglicht
das „Checkin/Checkout“-System die Nachverfolgung aller Assets, falls diese zum
Beispiel an Person ausgeliehen werden. Zu jedem Zeitpunkt kann ein Asset maximal
einer Person zugeordnet werden, wodurch das mehrfache gleichzeitige Ausleihen
eines Assets verhindert wird. Darüber hinaus beschreiben Status-Label den
Zustands eines Assets und ob dieses ausgeliehen werden kann. Alle
Funktionalitäten können zudem über eine REST-API programmatisch genutzt werden.

Des Weiteren verfügt \textit{Snipe-It} über eine Schnittstelle, welche die
Integration von LDAP stark vereinfacht.
\subsubsection{Schnittstellenbackend}
- Noch nicht klar

\subsubsection{Vue.js}
Achtung von Raimund geklaut: Die Grundlage des Frontends bildet der progressive
JavaScript Frame- work Vue.js. Bei der Nutzung von Vue CLI kann die
PWA-Funktionalität mithilfe des @vue/cli- plugin-pwa Pakets schnell eingebunden
werden. Zudem ist Vue.js mit 33.6kB ein kompaktes Framework und kleiner als die
zwei größten Konkurrenten Angular und React. Ebenfalls schnei- det Vue.js in
Leistungs-Benchmarks besser ab (Krause, 2022). Weiterhin ist Vue.js mit 192.814
GitHub-Sternen3 das beliebteste Framework.

- Quellen von: Stack overflow und State of js 2022 wie populer,...

Zusätzlich zu den bereits aufgeführten Eigenschaften wird Vue.js aufgrund
der begrenzten Implementierungszeit und bestehende Erfahrung gewählt.

\section{Systemarchitektur}
Mit Hilfe der bereits festgelegten Frameworks (\ref{section:frameworks}) wurde
die Systemarchitektur entwickelt. Die Systemarchitektur gibt eine Übersicht über
die technische Umsetztung des Systems, welches im Wesentlichen aus den folgenden
drei Komponenten besteht: dem Snipe-IT Server, dem Schnittstellenbackend und dem
Front-End, welches mit Vue.js realisiert wurde (\ref{fig:uml}).

\begin{figure}[h]
    \centering
    \includegraphics[scale=0.45]{Bilder/uml.pdf}
    \label{fig:uml}
    \caption[UML-Sequenzdiagramm]{UML-Sequenzdiagramm}
\end{figure}

\subsection{Snipe-IT Server}

- Für das Verwalten der Assets selbst, also Was ist vorhanden, und alle
Eigenschaften dazu

- Bietet API, welche vom Schnittstellenbackend verwendet wird, jedoch fehlt eine
relevante Funktion, daher das Schnittstellenbackend

--> Anderer weg erläutern?

- Oder Snipe IT erweitern?

\subsection{Schnittstellenbackend}
- Reservierung in zukunft möglich mit SnipeIT

- Logik muss selbst eingebaut werden

- Haupaufgabe: Reservierung, den rest übernimmt es auch als Schnittstelle

\subsection{Web-App}
- Zur Darstellung, zum suchen, buchen und so der Assets, umgesetzt mit vue.js






%Wie eingangs er-wähnt, definieren die Anforderungen, was das System zu
%leisten hat, während die Funktionalitä-ten definieren, wie das System diese gewährleistet.
\chapter{Interface-Design}
\label{chapter-design}
Im Folgenden wurde sich an dem Designprozess Scenario Based Design für die Entwicklung interaktiver
Systeme orientiert. Mithilfe der in \refname{section:iststand} geschilderten Schwierigkeiten konnten
einzelne Szenarien erarbeitet und in Formen einer Vision ausgehend von der aktuellen Situation
formuliert werden \refname{section:funktionale}.

Während des Interface-Designs wurden aus Szenarien Mockups und High-Fidelity-Prototypen entwickelt.
Die Prototypen wurden in Form einer Usability Evaluation direkt getestet: Für die Tests wurden
Usability Spezifikationen aus Szenarien abgeleitet.

Die Szenarien beschreiben Aufgaben die als Vorlage für die Test gestaltung dienen können. Die
Ergebnisse aus den Usability Evaluationen fließen in die weitere Entwicklung mit ein.

Außerdem wurden die Usability Heuristiken nach \citeA{experience_10_nodate} während des gesamten
Prozesses berücksichtigt, um Gebrauchstauglichkeit voraussetzen zu können.

Eine formative Evaluation des Interface-Designs ist wichtig, um früh die Gebrauchstauglichkeit,
die Einfachheit der Bedienung und die Akzeptanz der ästhetischen Gestaltung zu etablieren.


-> Desings immer eingebaut -> Formulierung und begründung
\section{Mockups}
Die Skizzen wurden per Hand skizziert. Dabei wurde sich stets auf die 
\section{High-Fidelity-Interface-Design}
\subsection{Designsprache}


\subsection{Evaluation des Design-Entwurfs}


\begin{table}[h]
    \centering
    \caption{Teilnehmende der Evaluation}
    \begin{tabular}{lll}
            \arrayrulecolor{maincolor}\hline
            \sffamily\color{maincolor}ID & \sffamily\color{maincolor}Alter &
            \sffamily\color{maincolor}Rolle \\
            \arrayrulecolor{maincolor}\hline
            E1                           & 19 - 25 J.                      &
            Medieninformatikerin                     \\
            E2                           & 19 - 25 J.                      & Roboterinnen \\
            E3                           & 19 - 25 J.                      & Medieninformatiker, Hilfswissenschaftler        \\
            E4                          & 19 - 25 J.                      & Medieninformatiker \\
            E5                           & 19 - 25 J.                      &
            Medieninformatikerinnen, Hilfswissenschaftlerin \\
            E6                           & 25 - 30 J.                      & Wissenschaftlicher Mitarbeiter        \\
            \arrayrulecolor{maincolor}\hline
    \end{tabular}
    \label{table:e}
\end{table}

\todo[inline]{Kriteriensuchen, mehrere Suchen, Wording ändern ->  Eindeutigkeit}

\todo[inline]{Assistent -> Wording: Auswahlhilfe, Suchhilfe, Kriteriensuche, Kiterienhilfe, Auwahl nach Kriterien, Ausleihhilfe
Material statt Assets}

\todo[inline]{Sauber, Clean und nicht überladen}

\todo[inline]{Bei Asset-Suche: Strich bei Kalender: Drawer
Suche und “Kategorien durchstöbern” }

Warum Ausleihzeitraum separat zu Abholung und Rückgabe -> Impliziert das bereits

-> Um repetetive Evaluationsergebnisse zu verhinden, wurden die
Evaluationsergebnisse kontinuierlich eingearbeitet.

\chapter{Implementierung}
\label{chapter-implementierung}
Das folgende Kapitel beschreibt die Implementierung des Reservierungsinterfaces
sowie des Frontends. Zunächst wird die Implementierung des
Reservierungsinterfaces und die damit einhergehenden technischen Aspekte
beschrieben. Dabei wird Aufschluss über die Struktur gegeben und die
Kernfunktionalitäten sowie Sackgassen\todo{ugs.? herausforderungen? oder kursiv}
in der Realisierung näher erläutert. Daraufhin wird die Umsetzung des
Frontends erläutert. Abschließend wird auf die Inbetriebnahme des Systems
eingegangen.


\section{Implementierung des Reservierungsinterfaces}
Dieser Abschnitt erläutert den technischen Aufbau des Rervierungsinterfaces und
geht auf relevante Aspekte in der Realisierung der Kernfunktionalitäten
(\ref{section:funktionale}) ein. Des Weiteren werden unerwartete
Herausforderungen thematisiert, welche im Rahmen der Arbeit nicht bewältigt
werden konnten.


\subsection{Aufbau des Reservierungsinterface}
Das Reservierungsinterface teilt sich in drei wesentliche Bestandteile:
der \textit{Fastify-HTTP-Server}, die \textit{SQLite Datenbank} und das \textit{ORM
  Prisma} (vgl. \ref{fig:level3}). Diese Komponenten spiegeln sich auch in der Verzeichnisstruktur aus \ref{fig:db} wider.

\begin{figure}[h]
  \centering
  \includegraphics[scale=0.7]{Bilder/Db.jpg}
  \caption[Verzeichnisstruktur des Reservierungsinterfaces]{Verzeichnisstruktur des Reservierungsinterfaces}
  \label{fig:db}
\end{figure}

Der HTTP-Server findet sich in der \textit{server.ts} wieder und stellt dort die
API des Reservierungsinterfaces bereit. Die entwickelte API lässt sich in drei
Bereiche teilen: \textit{Assets}, \textit{Kategorien} und
\textit{Reservierungen} (\ref{table:impl-backend-routes}). Für die drei
Inhaltstypen werden die Routen aus \ref{table:impl-backend-routes}
bereitgestellt, welche die von \citeA{fielding_hypertext_2014} beschriebene
Semantik für HTTP-Methoden beachten. Folglich werden bei Verwendung der
\textit{GET}-Methode ausschließlich Daten zurückerhalten. Hingegen muss bei
einer Anfrage mit zu übermittelnden Daten die \textit{POST}-Methode verwendet
werden, um einen neuen Eintrag im System zu erschaffen. Beispielsweise wird eine
\textit{GET}-Anfrage an \textit{/assets/:id} abgeschickt, um die Informationen
eines Assets zu erhalten. Um den Status in Snipe-IT auf \textit{herausgegeben}
zu aktualisieren, wird eine \textit{POST}-Anfrage an
\textit{/reservation/receive} gesendet, sobald Verleihende eine abgeholte
Reservierung bestätigen.

\begin{table}[h]
  \centering
  \caption{API des Reservierungsinterfaces}
  \begin{tabular}{lll}
    \arrayrulecolor{maincolor}\hline
    \sffamily\color{maincolor}Methode & \sffamily\color{maincolor}Route &
    \sffamily\color{maincolor}Funktion
    \\
    \arrayrulecolor{maincolor}\hline
    GET                               & \textit{/assets}                &
    Erhalte alle Assets                                                   \\
    GET                               & \textit{/assets/:id}            &
    Erhalte ein Asset mit der entsprechende ID                            \\
    GET                               & \textit{/categories}            &
    Erhalte alle Kategorien                                               \\
    GET                               & \textit{/reservation}           &
    Erhalte Reservierungen                                                \\
    POST                              & \textit{/reservation}           &
    Erstellen Reservierung                                                \\
    POST                              & \textit{/reservation/receive}   &
    Erstellen Reservierung                                                \\
    POST                              & \textit{/reservation/return}    &
    Erstellen Reservierung                                                \\
    POST                              & \textit{/reservation/id}        &
    Erstellen Erstellen Reservierung                                      \\
    DELETE                            & \textit{/reservation/delete}    &
    Löschen Reservierungen                                                \\
    PATCH                             & \textit{/reservation/patch}     &
    Verändern Reservierung                                                \\
    \arrayrulecolor{maincolor}\hline
  \end{tabular}
  \label{table:impl-backend-routes}
\end{table}

Die mit SQLite bereitgestellte Datenbank speichert die Reservierungen sowie die
Nutzenden. Hierfür wurde das in \ref{fig:orm} dargestellte Datenbankschema
erarbeitet. Zur Umsetzung dieses Schemas wurde das ORM Prisma genutzt, welches
drei zentrale Aufgaben vereinfacht.

\begin{enumerate}[1]
  \item Schema: Die in einer Datenbank zu erstellenden Anforderungen werden
        durch Prisma in vielerlei Hinsicht abgenommen. Das zu erstellende Schema
        benötigt keine SQL-??. Um die Datenbanken (User und Reservation) erstellen zu
        können werden keinerlei Befehle benötigt, dies geschieht ebenfalls über
        Prisma. Folglich ist das Schema entsprechend vereinfacht (\ref{fig:prisma}).
  \item .
  \item .
\end{enumerate}

\begin{lstlisting}[style=pseudo,gobble=2,caption={Pseudocode: Kategorien anzeigen},label=fig:prisma]
    // This is your Prisma schema file, learn more about it in the docs: https://pris.ly/d/prisma-schema

        generator client {
          provider = "prisma-client-js"
        }

        datasource db {
          provider = "sqlite"
          url      = env("DATABASE_URL")
        }

        model User {
          id           Int           @id @default(autoincrement())
          email        String        @unique
          name         String?
          reservations Reservation[]
        }

        model Reservation {
          id        Int      @id @default(autoincrement())
          assetId   Int
          dateStart DateTime
          dateEnd   DateTime
          User      User     @relation(fields: [userId], references: [id])
          userId    Int
          received  Boolean  @default(false)
          returned  Boolean  @default(false)
        }
\end{lstlisting}


Ergänzt wird diese durch das ORM Prisma.
stellt die mit \textit{prisma} entwickelte Datenbank als Object Relational
Mapper (ORM) mit folgendem Schema dar. \textit{User} können
\textit{Reservierungen} aufgeben.

\begin{figure}[h]
  \centering
  \includegraphics[scale=0.2]{Bilder/drawSQL-export-2022-10-09_15 56.png}
  \caption[Datenstruktur der Reservierungen in Verbindung mit Nutzenden]{Datenstruktur der Reservierungen in Verbindung mit Nutzenden}
  \label{fig:orm}
\end{figure}


\subsection{Implementierung der Kernfunktionalität} \todo{vielleicht Probleme in
  einem Kapitel Zweiter Ordnung am Ende des Implementierungskapitels zusammen,
  das wäre insgesamt stimmiger? } Dieser Abschnitt präsentiert die
Implementierung der Kernfunktionalität des Zwischenbackends, welche aus den
Anforderungen bestimmt wurden (\ref{section:anforderung}). Bei der
Funktionalität handelt es sich um das Reservieren in die Zukunft, sowie das
Speichern dieser Vorgänge und die damit einhergehende Bestätigung für die
Aktualisierung in Snipe-IT. Konkret wird auf das LDAP-System und das
Reservieren von Assts eingegangen.

Um auf die Assets zugreifen zu können, wurde mit der Snipe-IT JSON REST API
gearbeitet. Die Snipe-IT API umfasst viele Ressourcen. Relevante Ressourcen für
die Umsetzung der Arbeit waren \textit{/hardware, /statuslabels, /users,
  /categories}. Mithilfe dieser konnten die Assets der Beispieldatenbank verwendet
werden. \ref{fig:snipe} zeigt die Beziehungen der API, beispielsweise ist ein
\textit{/statuslabel} immer an einen \textit{/user} gebunden.

\begin{figure}[h]
  \centering
  \includegraphics[scale=0.2]{Bilder/drawSQL-export-2022-10-12_17 27.png}
  \caption[UML Snipe-IT API]{UML Snipe-IT API}
  \label{fig:snipe}
\end{figure}

Um mit der API arbeiten zu können muss ein \textit{API
  key}\footnote{\url{https://snipe-it.readme.io/reference/generating-API-tokens}}
generiert werden. Da persönliche Zugriffstoken verwendet werden, spiegeln die
Berechtigungen des API-Tokens die Berechtigungen des Nutzenden wider. Das
bedeutet, dass die Token lediglich manuell im Dashboard generierbar sind. Was
zur Folge hat, dass das geplante LDAP-System der Universität zu Lübeck nicht
ohne Umstände eingebunden und entsprechend genutzt werden kann. Snipe-IT gibt
unter anderem die Möglichkeit, das LDAP
Formular\footnote{\url{https://snipe-it.readme.io/docs/ldap-sync-login}}
simpel einzubinden, trotz dessen besteht das Problem der Authentifizierung
sowie Generierung der Tokens, da die Daten im Reservierungsinterface bis dato
nicht übertragen werden können. Dies führt zu einem Datenschutzproblem.


Um unter anderem ein besseres Verständnis der Interaktion von einzelnen
Komponenten voraussetzen zu können, wurde ein UML-Sequenzdiagramm erstellt
(\ref{fig:uml}). Bei der vierten Komponente handelt es sich um das LDAP-System
der Universität zu Lübeck. Mithilfe der Integration können Nutzende sich mit
ihrem bereits bestehenden IDM Account anmelden. Zum einen kann somit
sichergestellt werden, dass nur befugte Personen Assets einsehen und ausleihen
können. Zum anderen erleichtert es den Nutzenden den Ausleihprozess, da kein
neuer Account erstellt werden muss.

\begin{figure}[h]
  \centering
  \includegraphics[scale=0.45]{Bilder/uml.pdf}
  \caption[UML-Sequenzdiagramm]{UML-Sequenzdiagramm}
  \label{fig:uml}
\end{figure}

Die für das Frontend benötigten Ressourcen wurden über den angelegten Routen
erstellt. Zur Unterstützung wurde \textit{Fastify} verwendet. \textit{Fastify}
ist ein Webframework für \textit{Node.js}. Das Framework verfügt über vielerlei
Plugins und ist daher gut erweiterbar, außerdem erleichter es DINGE.

Der eigentliche Kern der Anwendung befasst sich mit dem Reservieren der Assets.
Hierfür müssen die Daten (Assetname, Datum, Uhrzeit, Ort, \dots) gespeichert
werden. Bevor ein Status eines Assets in Snipe-IT von \textit{verfügbar} zu
\textit{herausgegeben} geändert wird, werden die Daten für eine Reservierung im
Reservierungsinterfaces zwischengespeichert. Für diesen Schritt wurde SQLite
verwendet. Zusätzlich wurde für eine leichtere Verwaltung der angelegten
Datenbank Prisma genutzt. Die von Snipe-IT zur Verfügung gestellten
Möglichkeiten der Statuseingabe/Ausleihen schwer und komisch -> rumarbeiten.

\todo[inline]{Gibt es sonst etwas Wichtiges, was man hier erwähnen sollte? Code Beispiele?}

\section{Implementierung des Frontends} \todo{Verweis auf die Dokumentationen,
  Hier unbedingt Rückbezug auf Konzeption nehmen} Das kommende Unterkapitel
beschreibt die Client-seitige Realisierung der Arbeit. Zunächst wird der
Aufbau betrachtet (\ref{fig:vue}), daraufhin wird die Nutzung von vue und XXX
erläutert. Des Weiteren wird auf die Struktur der Komponenten eingegangen
(\ref{fig:komponenten}). Abschließend wird das native App-Erlebnis
thematisiert.

\begin{figure}[h]
  \centering
  \includegraphics[scale=0.7]{Bilder/Db.jpg}
  \caption[Verzeichnisstruktur des Reservierungsinterfaces]{Verzeichnisstruktur des Reservierungsinterfaces}
  \label{fig:vue}
\end{figure}

Für den Aufbau des Projektes wurde aus den in \ref{chapter-konzept}
festgestellten Anforderungen \textit{vue.js} verwendet. Bei der Implementierung
wurde sich an den best practices der Vue.js-Dokumentation orientiert
\todo{(Vue.js, 2021a)}. Für sich wiederholende Elemente wurden eigene Views
erstellt. Dadurch ergibt sich eine hierarchisch geschachtelte Client-Anwendung
der Vue-Komponenten. \ref{fig:Komponenten} stellt die Komponenten-Struktur
vereinfacht dar. Um konkretere Vorschläge in der Entwicklungsumgebung zu
ermöglichen und vorzeitige Fehler zu minimieren, wurde ergänzt zu
\textit{JavaScript} \textit{TypeScript} verwendet.

\begin{figure}[h]
  \centering
  \includegraphics[scale=0.5]{Bilder/struktur.pdf}
  \caption[Vereinfachte Struktur der Komponenten]{Vereinfachte Struktur der Komponenten}
  \label{fig:Komponenten}
\end{figure}

Um das Styling zu vereinfachen wurde \textit{Tailwindcss} genutzt. Ergänzend
dazu wurde \textit{Preline} für UI verwendet. -> Nicht empfehlenswert
(Sackgassen). Für Animationen und Transitionen wurde \textit{HeadlessUi}
genutzt.

Der Kalender stellt einen wichtigen Bestandteil für das Reservieren von Assets
dar. Die Kalenderkomponente wurde durch \textit{v-calender} ergänzt. Durch die
Möglichkeiten der Komponente ...??. Außerdem bietet diese viele
Ausbaumöglichkeiten für folgende Funktionalitäten.

Die im Rahmen dieser Arbeit verwendete Beispieldatenbank beinhaltete eine
Vielzahl an Kategorien und Unterkategorien, welche wie folgt, eingetragen wurde:
\enquote{Kategorie - Unterkategorie}. Da das Anzeigen von den Kategorien
unübersichtlich ist, sollten zunächst nur die Kategorien und dann die
Unterkategorien angezeigt werden. \ref{fig:categoriecode} stellt den
Entwickelten Algorithmus als Pseudocode dar. Der \enquote*{-} wurde dazu als
Trennsymbol verwendet. Daraufhin wurden der Vorderteil als Kategorie verwendet
und der Hintere als Unterkategorie. Um die Kategorien und Unterkategorien nicht
doppelt anzeigen zulassen, wurde eine Abfragen gemacht, bei welcher diese
abgeglichen wurde bis keine Gleichheit mehr bestand und die nächste Kategorie
angezeigt werden kann, bis alle Kategorien und Unterkategorien verglichen
wurden.

\begin{lstlisting}[style=pseudo,gobble=2,caption={Pseudocode: Kategorien anzeigen},label=fig:categoriecode]
  // Trennung am Strich for $- \gets Kategorie$ to $Unterkapitel$ do ICH BIN
  DUMM
\end{lstlisting}


\section{Nutzung des Systems}
Der folgende Abschnitt führt die nötigen Schritte auf, um das System in Betrieb
nehmen zu können. Zuerst wird die Installation erklärt, gefolgt von der
Konfiguration und Ausführung des Systems.

\subsection{Installation}
Für die Nutzung des Systems wird eine Installation von \textit{Node.js} sowie
der Paket-Manager \textit{npm} benötigt. Anschließend, kann das System auf \dots

\subsection{Konfiguration}
.env Datei angelegen

Code einfügen

\subsection{Ausführung von Snipe-IT}
Um das System nutzen und weiterentwickeln zu können muss... API KEy generieren
lassen.

\subsection{Ausführung der Web-App}
\todo[inline]{Zum Ausführen oder Bereitstellen der Web-App muss diese vorher
  noch „gebaut“ werden. Hierbei handelt es sich um einen Prozess, in dem der
  Quellcode auf ein Minimum reduziert wird. Externe Pakete werden eingebunden
  und ungenutzter Code, soweit möglich, entfernt. Um diesen Prozess anzustoßen,
  muss im Stammverzeichnis der Web-App der Befehl yarn␣build ausgeführt werden.
  Nachdem der Bau-Prozess erfolgreich abgeschlossen ist, sollte ein neues ist
  Verzeichnis generiert wurden sein. Dieses Verzeichnis kann nun zum statischen
  Hosting verwendet werden.}


\section{Fazit der Implementierung}

Sackgassen YEAH

\todo[inline]{Probleme schildern, dann wirkt es nicht so komisch,  mitten in deiner Arbeit auf "Sackgassen" einzugehen, sondern dann hier einfach darauf eingehen, dass Probleme aufgetreten sind, welche mit diesen und jenen Workarounds gelöst wurden}
\chapter{Dialogbeispiele}
\label{chapter-dialogbeispiel}
Der folgende Abschnitt präsentiert das in der vorliegenden Arbeit realisierte
System anhand eines beispielhaften Nutzungsszenarios. Das Szenario startet mit
Mila, einer Erstsemester Studentin im Studiengang Medieninformatik. Mila belegt
im ersten Semester das Modul \textit{\ac{emi}}. Das Einführungsmodul umfasst
eine Gruppenarbeit, in der eine Idee  zum Thema \textit{VR/AR} entwickelt werden
soll. Am Ende des Semesters soll das Projekt bei den \textit{Media Moments} in
der \textit{\ac{emi} Award App}\todo{RAIMUNDS BA HEHE} ausgestellt werden.
Milas Projektgruppe hat sich entschieden eine AR-Anwendung für Erze und Metalle
zu gestalten, bei der die Erze und Metalle mithilfe einer App eingescannt werden
können und ihre entsprechenden Eigenschaften angezeigt werden. Um das Projekt in
der \textit{\ac{emi} Award App} präsentieren zu können, will die Gruppe ein
Werbevideo für die App aufnehmen.

In einem Videoworkshop von Georg Fink, fiktiver \ac{wimi} am \ac{imis}, erfahren
Mila und ihre Gruppenmitglieder:innen, dass über die Ausleih-App
\textit{Snipe-IT Companion} unter anderem Videoequipment an Studierende
verliehen wird.

Mila ruft die Ausleih-App unter der URL \textit{https://snipe-it-companion.de/} auf und meldet sich
mit ihren Daten an. Daraufhin wird sie zu einem bisher leeren \textit{Dashboard} weitergeleitet und
aufgefordert nach benötigtem Material zu suchen. Da Mila sich nicht sicher ist, welche Kamera sie
benötigt, schaut sich Mila unter dem Menü, in den Kategorien um und findet schnell die Kategorie
\enquote{Kameras} (\ref{fig:login}). Nachdem sie auf die Seite der Unterkategorien weitergeleitet
wurde, entscheidet sich Mila dafür, eine \textit{GoPro} auszuleihen, weil sie so auch Erze und
Metalle unter Wasser aufnehmen können. Dazu klickt Mila auf \enquote{Hinzufügen} (\ref{fig:suchen}).
\begin{figure}[p]
    \centering
    \includegraphics[scale=0.17]{Bilder/Dialgobeispiel/Login.png}\hspace{1em}
    \includegraphics[scale=0.17]{Bilder/Dialgobeispiel/Suiche2.png} \hspace{1em}
    \includegraphics[scale=0.17]{Bilder/Dialgobeispiel/Kategorien.png}
    \caption[Dialogbeispiel 1]{Kategorieergebniss (l), Detailansicht: GoPro (m, r)}\label{fig:login}
\end{figure}
\begin{figure}[p]
    \centering
    \includegraphics[scale=0.17]{Bilder/Dialgobeispiel/Suche.png}\hspace{1em}
    \includegraphics[scale=0.17]{Bilder/Dialgobeispiel/Details 1.png} \hspace{1em}
    \includegraphics[scale=0.17]{Bilder/Dialgobeispiel/Details 2.png}
    \caption[Dialogbeispiel 1]{Kategorieergebniss (l), Detailansicht: GoPro (m, r)}\label{fig:suchen}
\end{figure}

\newpage
Nun wird sie dazu aufgefordert, einen Ausleihzeitraum anzugeben. Milas Gruppe
hat entschieden, dass sie von kommenden Montag bis Mittwoch das Material
benötigten. Da Mila von 8:00 Uhr bis 10:00 Uhr eine Vorlesung hat, gibt sie
10:30 Uhr als Abholzeit und Rückgabezeit an. Um die Reservierung abschließen zu
können, klickt Mila auf \enquote{Reservieren} (\ref{fig:datum}). Mila überprüft in der
Zusammenfassung, ob ihre Angaben stimmen und stellt fest, dass die Rückgabezeit
am Mittwoch doch nicht mehr in ihren Kalender passt. Daher ändert sie die
Uhrzeit auf 12:30 Uhr und bestätigt abschließend ihre Reservierung (\ref{fig:geandert}).

\begin{figure}[p]
    \centering
    \includegraphics[scale=0.17]{Bilder/Dialgobeispiel/Datum eingeben.png}\hspace{1em}
    \includegraphics[scale=0.17]{Bilder/Dialgobeispiel/mit datum .png}\hspace{1em}
    \includegraphics[scale=0.17]{Bilder/Dialgobeispiel/Falsche Uhrzeit.png}
    \caption[Dialogbeispiel 1]{Zeitraum eingabe (l), Detailansicht: GoPro mit Datum (m), Übersicht (r)}\label{fig:datum}
\end{figure}
\begin{figure}[p]
    \centering
    \includegraphics[scale=0.17]{Bilder/Dialgobeispiel/datumänderung.png}\hspace{1em}
    \includegraphics[scale=0.17]{Bilder/Dialgobeispiel/Bestätigung.png} \hspace{1em}
    \includegraphics[scale=0.17]{Bilder/Dialgobeispiel/Zsuammenfassung.png}
    \caption[Dialogbeispiel 2]{zeitraumsänderung (l), Detailansicht: GoPro mit Datum (r)}\label{fig:geandert}
\end{figure}

Da sich die Gruppe nicht sicher ist, welches Mikrofon für die Aufnahme eines
Voiceover sinnvoll ist, suchen sie zunächst in der Ausleih-App nach Mikrofon und
sehen, dass Georg Fink für diese zuständig ist. Daraufhin verfasst Mila eine
E-Mail an Herrn Fink, in der sie um eine Mikrofon-Empfehlung für Voiceovers
bittet. Nachdem dieser mit zwei Vorschlägen geantwortet hat, sucht Mila nach
jenen, unter der Angabe ihres Ausleihzeitraums.  Direkt stellt Mila fest, dass
nur eines der Mikrofone in dem Ausleihzeitraum ausleihbar ist und reserviert
dieses.

Georg ist als fiktiver \ac{wimi} am \ac{imis} für zwei Labore zuständig, in
denen Material ausgeliehen werden kann. Zum Feierabend überprüft er sein
Verwaltungsdashboard auf dem Desktop in der Ausleih-App. Er sieht, das eine neue
Reservierung für Montag um 10:30 Uhr von Mila eingegangen ist (\ref{fig:georg}).

Da sich Mila am Sonntag nicht sicher ist, wo die Materialien abgeholt werden
sollen, schaut sie erneut in der Ausleih-App nach und findet den Abholort auf
der Dashboardansicht.

Nach der Vorlesung am Montag macht sich Mila auf den Weg in das Gebäude 64 zum
Abholort: Techniklabor. Georg wartet dort bereits auf Mila und erklärt ihr, was
sie bei der Nutzung der \textit{GoPro} beachten sollte. Nachdem Mila gegangen ist, trägt
Georg auf seinem Smartphone ein, dass die Materialien abgeholt wurden (\ref{fig:georg2}). Sobald
Georg die Abholung bestätigt hat, wird der Status in der internen Datenbank von
Snipe-IT geändert (\ref{fig:georg2}).

\begin{figure}[p]
    \centering
    \includegraphics[scale=0.25]{Bilder/Dialgobeispiel/Verwaltung.png}
    \caption[Dialogbeispiel 3]{Desktop: Verwaltung}\label{fig:georg}
\end{figure}
\begin{figure}[p]
    \centering
    \includegraphics[scale=0.17]{Bilder/Dialgobeispiel/Reservierung Abholung.png}\hspace{1em}
    \caption[Dialogbeispiel 3]{Verwaltung}\label{fig:georg2}
\end{figure}

\newpage
Laura Eggers ist ebenfalls fiktive \ac{wimi} am \ac{imis} und möchte für eine
Studie ein Mikrofon ausleihen. Da sie Zugriff auf Snipe-IT hat, schaut sie
zunächst dort nach dem gewünschten Material und stellt fest, dass dieses
\textit{herausgegeben} ist. Daraufhin ruft sie die URL
\textit{https://snipe-it-companion.de/} auf und reserviert das Mikrofon im
gewünschten Zeitraum. Am Abholtag entnimmt sie das Asset und bestätigt die
Abholung auf ihrem Smartphone \ref{fig:georg5}. \todo{Hier wird der Use-Case noch nicht so klar}
\begin{figure}[h]
    \centering
    \includegraphics[scale=0.17]{Bilder/Screenshot 2022-10-14 at 11-26-25 Asset ansehen 00087 Ausleihmanagement.png}
    \caption[Assetansicht in Snipe-IT]{Assetansicht in Snipe-IT}\label{fig:georg5}
\end{figure}


Milas Gruppe stellt am Dienstag fest, dass ihnen der Ausleihzeitraum nicht ausreicht und möchte
diesen daher um einen Tag verlängern. Dafür öffnet sie die Ausleih-App und sieht auf dem Dashboard
unter dem Tab \textit{Laufende} ihre Reservierungen. Daraufhin ändert sie die Daten der beiden
Materialien auf Donnerstag um 9:00 Uhr (\ref{fig:andern}).\todo{Noch kurz erwähnen das Georg die Verlängerung auch in
seinem Dashboard sehen kann} 
\begin{figure}[h]
    \centering
    \includegraphics[scale=0.14]{Bilder/Dialgobeispiel/laufende nach änderung.png}\hspace{1em}
    \includegraphics[scale=0.14]{Bilder/Dialgobeispiel/Datum ändern 2.png}\hspace{1em}
    \caption[Dialogbeispiel 3]{Zeitraumverlängern}\label{fig:andern}
\end{figure}

Am Donnerstag um 9:00 Uhr wartet Mila bereits auf Georg, welcher die
Materialien entgegennimmt und in seinem Büro die Rückgabe bestätigt (\ref{fig:zuruck}).

Im vierten Semester möchte Mila das Mikrofon für das Modul \textit{\ac{ide}}
ausleihen, um wieder ein Voiceover aufnehmen zu können. Sie findet das Mikrofon
unter \textit{zurückgegeben} und leiht das Material erneut aus (\ref{fig:zuruck}).
\begin{figure}[h]
    \centering
    \includegraphics[scale=0.14]{Bilder/Dialgobeispiel/Altes Datum.png}\hspace{1em}
    \includegraphics[scale=0.14]{Bilder/Dialgobeispiel/Zurück.png}\hspace{1em}
    \caption[Dialogbeispiel 5]{Zurückgegeben}\label{fig:zuruck}
\end{figure}
%!TEX root = thesis.tex

\chapter{Evaluation}
\label{chapter-evaluation}
Im Rahmen der Forschungsfrage F3 soll herausgearbeitet werden, inwiefern die Gebrauchstauglichkeit
und Nützlichkeit durch das in der vorliegenden Arbeit entwickelte Reservierungssystem gewährleistet
werden kann. Da es sich um ein universitätsinternes Tool handelt, wurde sich
bei den Versuchspersonen, im Rahmen eines Proof of Concepts (N=5) mit Mitarbeitenden des \ac{imis}
und Studierenden im Bereich der Medieninformatik zusammengesetzt, um das konzipierte Studiendesign
zu evaluieren. Die Evaluation wird zusätzlich dafür genutzt, das entwickelte Reservierungssystem zur
Untersuchung von Forschungsfrage F2 hinsichtlich Funktionen, Gestaltung und Nachvollziehbarkeit
bewerten zu lassen. Das Kapitel geht zunächst auf das Vorgehen und daraufhin auf die Benutzergruppen
und Ergebnisse ein.


\section{Vorgehen und Methodik}
- Wording mit Abfragen


Zu Beginn der Studienplanung wurden Evaluationsaufgaben definiert, die die Versuchspersonen Schritt
für Schritt durchführen sollten. Dabei sollten diese Think-Aloud anwenden und stets mitteilen, wenn sie
mit einer Aufgabe fertig sind.

Um die Gebrauchstauglichkeit und Nützlichkeit der Web-App abschließend feststellen zu können, wurde ein
Online-Fragebogen entworfen \todo{Anhang}. Zu Beginn des Fragebogens mussten Proband:innen die Rolle
(Verleihende, Ausleihende) angeben.

Dieser enthielt Fragen zu demografische Daten, UEQ und
den Funktionen der Anwendung (z. B. Ordnen Sie die im System bereits vorhandenen Funktionen nach
Wichtigkeit an.). Abschließend hatten Versuchspersonen in Freitexten weitere Anmerkungen zu geben.

In beiden Fällen wurde zuerst die Nützlichkeit der eingesetzten Funktionalitäten bewertet. Dies geschah mit einer
4-Punkt Likert-Skala, welche mit nicht hilfreich, wenig hilfreich, hilfreich und sehr hilfreich beschriftet
wurde.

Schließlich wurden Proband:innen befragt, wie sie das System in seiner Gesamtheit bewerten würden
und ob sie sich dieses System für den regelmäßigen Gebrauch vorstellen könnten.


\section{Verleihende}
\begin{table}[h]
  \centering
  \caption{Teilnehmende der Interviews, Verleihende}
  \begin{tabular}{lll}
    \arrayrulecolor{maincolor}\hline
    \sffamily\color{maincolor}ID & \sffamily\color{maincolor}Alter &
    \sffamily\color{maincolor}Zuständigkeitsbereich                                  \\
    \arrayrulecolor{maincolor}\hline
    V1                           & 25 - 35 J.                      & Keine direkte
    Zuständigkeit, Zugänge zu verschiedenen Laboren                                  \\
    V2                           & 25 - 35 J.                      & Multimedialabor
    \\
    V3                           & 25 - 35 J.                      & VR-Labor
    \\
    V4                           & 40 - 59 J.                      & Administratives \\
    \arrayrulecolor{maincolor}\hline
  \end{tabular}
  \label{table:vzwei}
\end{table}

\begin{table}[h]
  \centering
  \caption{Werte der \ac{ati}-Skala}
  \begin{tabular}{lccc}
    \arrayrulecolor{maincolor}\hline
    \sffamily\color{maincolor}Benutzergruppe & \sffamily\color{maincolor}Mittelwert
    $(M)$                                    & \sffamily\color{maincolor}Standardabweichung $(SD)$ &
    \sffamily\color{maincolor}Teilnehmende $(N)$                                                          \\
    \arrayrulecolor{maincolor}\hline
    Verleihende                              & 5,00                                                & 0,58
                                             & 3                                                          \\
    Ausleihende                              & 5,13                                                & 0,48
                                             & 6                                                          \\
    \arrayrulecolor{maincolor}\hline
  \end{tabular}
  \label{table:atipartzwei}
\end{table}

\section{Ausleihende}

\begin{table}[h]
  \centering
  \caption{Teilnehmende der Interviews, Ausleihende \\
    (die mit * gekennzeichneten Versuchspersonen waren Mitarbeitende)}
  \begin{tabular}{lll}
    \arrayrulecolor{maincolor}\hline
    \sffamily\color{maincolor}ID & \sffamily\color{maincolor}Alter &
    \sffamily\color{maincolor}Rolle                                                     \\
    \arrayrulecolor{maincolor}\hline
    A1                           & 19 - 25 J.                      & Bachelorstudentin,
    Hilfswissenschaftlerin                                                              \\
    A2                           & 19 - 25 J.                      & Bachelorstudent
    \\
    A3                           & 19 - 25 J.                      & Masterstudent,
    Hilfswissenschaftler                                                                \\
    A4*                          & 19 - 25 J.                      & Bachelorstudentin
    \\
    A5*                          & 19 - 25 J.                      & Bachelorstudentin
    \\
    \arrayrulecolor{maincolor}\hline
  \end{tabular}
  \label{table:azwei}
\end{table}


\begin{table}[h]
  \centering
  \caption{Werte der \ac{ati}-Skala}
  \begin{tabular}{lccc}
    \arrayrulecolor{maincolor}\hline
    \sffamily\color{maincolor}Benutzergruppe & \sffamily\color{maincolor}Mittelwert
    $(M)$                                    & \sffamily\color{maincolor}Standardabweichung $(SD)$ &
    \sffamily\color{maincolor}Teilnehmende $(N)$                                                          \\
    \arrayrulecolor{maincolor}\hline
    Verleihende                              & 5,00                                                & 0,58
                                             & 3                                                          \\
    Ausleihende                              & 5,13                                                & 0,48
                                             & 6                                                          \\
    \arrayrulecolor{maincolor}\hline
  \end{tabular}
  \label{table:atidrie}
\end{table}


\blockquote{App war top...}


\section{Diskussion}
%!TEX root = thesis.tex

\chapter{Zusammenfassung und Ausblick}
\label{chapter-fazit}

Abschließen werden die wischtigsten Ergebnisse sowie Antworten auf die Forschungsfragen
(\ref{sec:Forschungsfragen}) der vorliegenden Arbeit zusammengefasst. Im Anschluss werden offene
Punkte aus dem Konzept, welche nicht wie geplant umgesetzt werden konnten, erläutert. Zum Abschluss
auf die möglichen Weiterentwicklungen des Frameworks eingegangen.


\section{Zusammenfassung}
Für einen Ausleih- und Reservierungsprozess ist die Unterstützung durch ein Reservierungstool eine
zielführende Möglichkeit, um die Planung in die Zukunft zu erleichtern und ermöglichen. Hierbei ist
insbesondere eine strukturierte und übersichtliche Ansicht der auszuleihenden Assets von hoher
Bedeutung.

Um ein solches Konzept entwickeln zu können wuden zunächst aus vorangestellten Untersuchungen die
objektiven Anforderungen an den Snipe-IT Companion formalisiert (\ref{section:anforderung}). Die
wesentliche Informationsquelle stellten hierbei die Stakeholder-Interviews mit Mitarbeitenden des
\ac{imis} und Studierenden der Medieninformatik. Zusätzlich wurde eine Recherche zu vergleichbaren
Systemen vorgenommen, welche Aufschluss über mögliche Fehlerquellen geben sollte. Hierbei wurde sich
nach einer weiteren Recherche auf zwei aus dem Interview genannte Apps beschränkt, da die vorherige
Recherche wenig Aufschluss für den spezifischen Einsatzfall gab. Aus den Qullen konnten zwei
Nutzendengruppen (Verleihende und Ausleihende) festgehalten (\ref{section:Nutzenden}). Für
Forschungsfrage F1 wurden die Probleme und Herausforderungen des aktuellen Vorgehens und den
unterschiedlichen Ausleihprozessen, in einer Problemanalyse, erarbeitet und beantwortet
(\ref{section:iststand}). Anschließend wurden für die Vorbereitung auf Forschungsfrage F2 die
Aufgaben, die Verleihende und Ausleihende im Reservierungsprozess erledigen müssen diskutiert
(\ref{section:aufgaben}). 

In der Spezifikationsphase wurden die Anforderungen an das System weiter spezialisiert
(\ref{chapter-konzept}). Dazu wurden zunächst gemäß der Forschungsfrage F2 Funktionalitäten
definiert. Die Funktionalitäten wurden entsprechend den Anforderungen entwickelt und in einer
priorisierten Feature-Liste festgehalten. Anschließend wurde die Systemarchitektur, aufgeteilt in
Frontend, Reservierungsinterface (Backend) und dem bestehenden Snipe-IT Server, mithilfe des
C4-Models dargestellt (\ref{section:architektur}). Aufbauend darauf wurden passende Frameworks zur
Entwicklung ausgewählt. 

Aufbauend auf dem Konzept wurde das Interface-Design erarbeitet (\ref{chapter-design}). Durch das
regelmäßige einarbeiten von Interviews konnte ein iteratives Vorgehen ermöglicht werden. Daraus
resultierten zentrale Designentscheidungen wie Begrifflichkeiten oder die Navigation, welche in
Folgearbeiten berücksichtigt werden sollten. 

Mithilfe der vorangehstellten Phasen wurde das Reservierungstools entsprechend umgesetzt
(\ref{chapter-implementierung}). Hierbei wurden die in der Konzeptionsphase festgelegten Frameworks
genutzt. \ref{chapter-dialogbeispiel} präsentiert das realisierte System anhand von
Dialogbeispielen.

In der abschließenden Phase wurde das realisierte System mithilfe von Interviews und Umfragen
evaluiert (\ref{chapter-evaluation}). Die Ergebnisse der Phase dienten zur Beantwortung von
Forschungsfrage F3, inwieweit ein aus F2 resultierender Prototyp die in F1 identifizierten
Schwierigkeiten reduzieren kann und gaben Aufschluss über die Wirksamkeit des entwickelten Systems.
Generell wurde die Oberfläche als übersichtlich beschrieben und der Wunsch, die Anwendung im
Universitätsalltag für Projekte nutzen zu können geäußert. Die eingangs formulierte Forschungsfrage
F3 konnte lediglich bedingt beantwortet werden, da das positive Ergebnis der durchgeführten Studie
nicht verglichen werden konnte. Eine weitere ausführliche Studie mit einem Vergleichsystem wird
somit empfohlen. Trotz des fehlenden Vergleichs konnte die Arbeit ein qualitativ hochwertiges System
hervorbringen, welches sich als übersichtlich und unterstützend herausstellt. Für Forschungsfrage 2
konnten weitere Anforderungen an das System herausgearbeitet werden. 

\section{Offene Punkte}
\label{sec:punkte}
Die in den Anforderungen festgelegten Funktionalitäten mit einer hohen Prioriät konnten umgesetzt
werden. Die mittle prioriseirten Funktionalitäten stellten sich als Herausforderung dar.
Insbesondere \anfref{F90}, die Einbindung des LDAP-Systems führte zu Schwierigkeiten und wurde
dementsprechend nicht umgesetzt. Zentrale Gründe hierfür waren der \textit{persönlichen
Zugriffstoken}, welcher die einzige Authentifizierungsmöglichkeit der Snipe-IT API ist und lediglich
manuell im Dashboard generiert werden können. Demzufolge konnte das LDAP-System nicht ohne Umstände
eingebunden und entsprechend genutzt werden. Dis führte auch dazu, dass eine Unterscheidung der
Zugriffsrechte zunächst nicht sichergestellt werden konnten (\anfref{Q20}). Durch die eingeschränkte
Beispieldatenbank konnte eine Nutzen-Suche (\anfref{F100}) nicht ermöglicht werden. Aus der
Evaluation ließ sich jedoch schließen, dass diese Funktion erwünscht sei.

Die erarbeite Funktionalität Ft-VA-7 umfasst das Filtern von Materialen. Das heißt unter anderem,
dass Nutzende verschiebene Preferezen, wie Sortierung, Status, etc. für die Suche einstellen können.
Im Interface-Design wurden bereits Entwürfe für diese Komponente erarbeitet \todo{Anhang}.

Funktionalitäten zur Pflege und Wartung von Assets (F-V-5) ist durch das Assetmangement
Snipe-IT bereits gegeben, diese Prozesse sollten jedoch am \ac{imis} klar kommuniziert werden.


\section{Ausblick}
Die Resultate der Untersuchungen im Rahmen der Arbeit können als Indikator und Grundlage für
weiterführende Arbeiten und Untersuchungen eingesetzt werden. Insbesondere für Erhebungen mit der
Nutzendengruppe wurden zahlreiche Vorarbeiten geleistet, an die es anzuknüpfen gilt. Ebenso
zentrall sind die offenen Punkte für die Weiterentwicklung des Systems.

\subsection{Weiterentwicklung der Funktionalitäten}
Das Filtern von Material (F-VA-7) wurde ebenfalls in den Evaluationsergebnissen herausgearbeitet und
sollte bei der Weiterentwicklung des Systems mit berücksichtigt werden. 

Aus der Evaluation ließen sich die Funktionalität \enquote{Nach Zweck suchen} und
\enquote{Set-Vorschläge} ableiten, welche in der späteren Entwicklung implementiert und evaluiert
werden könnten.

Bevor die ausgeliehenen Assets an Verleihende zurückgeben werden, sollte eine Checkliste
für das jeweilige Asset angezeigt. Dort werden Hinweise angezeigt, wie zum Beispiel
SD-Karte geleert, Assets auf Ursprungseinstellungen zurückgestellt oder Akkus geladen. Diese
Funktionalität soll insbesondere dafür sorgen, dass nachfolgende Ausleihende die Assets
direkt nutzen können.

\subsection{Weiterentwicklung der Kalenderkomponente}
Die im Rahmen dieser Arbeit verwendete Kalenderkomponente
\textit{V-Calendar}\footnote{\url{https://vcalendar.io/layouts.html}} ermöglicht eine leichte
Erweiterbarkeit und sollt insbesondere in der mobile Ansicht beim Bearbeiten des
Reseriveurngszeitraums angepasst werden.

\subsection{LDAP-System Einbindung}
Wie bereits in \ref{sec:punkte} erläutert konnte die Accountbasierte Nutzung des Systems nicht
umgesetzt werden. Für den realen Einsatz des Systems ist diese Funktionalität unabdingbar. Die
Einbindung des  LDAP-Systems deckt zudem die Datenschutz und Zugriffsrechte der Nutzenden ab, sodass
lediglich Nutzende der Universität zugriff auf die Listen der am \ac{imis} bestehenden Assets haben.
Eine mögliche Umsetzung ist in \ref{subsec:heraus} aufgeführt. 

\subsection{Einbindung in das Labormanagementsystem}
Abschließend ist eine Möglicheit, um die Prozesse und die aus der Evaluation erarbeiteten Probleme
der Begrifflichkeiten in Bezug auf den Assetstatus lösen zu können, das erarbeitet Konzept in das
Labormanagementsystem zu \todo{Pabst Zitieren} integrieren. Da sich das Labormanagementsystem
bereits in der zweiten Iteraktion befindet, ist die Integration auf diesem Wege sinnvoll. Dies
ermöglicht ein Reservierungs- und Ausleihsystem für Räume und Assets. Demzufolge können Assets,
welche den Status \enquote{Fest verbaut} aufweisen, über die Raumbuchung genutzt werden. Außerdem
haben Nutzende so nur ein System. Dies löst somit ebenfalls die Funktionalität der Accountbasierten
Nutzung.




\backmatter

\cleardoublepage
\phantomsection
\addcontentsline{toc}{chapter}{Abbildungsverzeichnis}
\listoffigures

\cleardoublepage
\phantomsection
\addcontentsline{toc}{chapter}{Tabellenverzeichnis}
\listoftables

\cleardoublepage
\phantomsection
\addcontentsline{toc}{chapter}{Abkürzungsverzeichnis}
\lstlistoflistings

%!TEX root = thesis.tex

\cleardoublepage
\phantomsection
\pdfbookmark{Abkürzungsverzeichnis}{abbreviations}
\chapter*{Abkürzungsverzeichnis}
\label{section-abbrevs}


\begin{acronym}[CoLab]
  \acro{imis}[IMIS]{Institut für Multimediale und Interaktive Systeme}
  \acro{HTTPS}[HTTPS]{HyperText Transfer Protocol Secure}
  \acro{ati}[ATI]{Fragebogen zur interaktionsbezogenen Technikaffinität}
  \acro{hiwi}[HiWi]{Hilfswissenschaftler:innen}
  \acro{wimi}[WiMi]{wissenschaftliche:r Mitarbeiter:innen}
\end{acronym}


\cleardoublepage
\phantomsection
\bibliography{Verzeichnisse/literature}
%\pdfbookmark{\bibname}{bibliography}
%\bibliography{literature}

\appendix

%!TEX root = thesis.tex

\chapter{Anhang}


\section{Analyse: Interviewleitfaden}
\label{appendix:interview}

Dieser Anhang enthält tiefergehende Informationen, die nicht zur eigentlichen Arbeit gehören.

\begin{itemize}
    \item Begrüßung und Danken für die Zeit
    \item Kurzer Umriss des Themas: Ziel in den Interviews ist es, zu erfahren, wie die Organisation
          und Planung unter den Mitarbeiter:innen untereinander sowie zwischen den Mitarbeiter:innen und
          Studierenden aktuell abläuft und wie diese Kommunikation verbessert werden kann, in Bezug auf
          das Ausleihen von Assets + in 2 Teile geteilt
    \item Tätigkeit
    \item Datenschutz
    \item Wenn einverstanden sind, Interview aufzeichnen und würden Sie bitten, dazu das
          Datenschutzformular auszufüllen. Die Aufzeichnungen werden diskret behandelt und nach der
          Auswertung gelöscht
\end{itemize}

\subsection{Verleihende}
{\sffamily\color{maincolor}{Abschnitt: Jetzt}}
\begin{enumerate}
    \item Interessieren, wie Sie vorgehen, wenn Sie eine Anfrage erhalten zum Ausleihen eines
          Assets.
          \begin{enumerate}
              \item Bekommen Sie anfragen per Mail
              \item oder mündlich
              \item Kalender Eintragen (Outlook)/Zettel
              \item Einfach nehmen
          \end{enumerate}
    \item Gibt es eine öffentliche Übersicht (für Studierende) der auszuleihenden Geräte? Nein: Hat
          das schon einmal für Probleme gesorgt im Ausleihprozess? Im Nachhinein gehört, dass Assets
          benötigt wurden
    \item Können Personen vorläufig Systeme reservieren, z. B. in 2 Wochen für 4 Tage?
          Warum nicht? Probleme
    \item Wird sichergestellt, dass Ausleihende mit dem Gerät umgehen können?
    \item Wie sind die Assets versichert bzw. was passiert, wenn es kaputtgeht?
    \item Sehen Sie Probleme oder sind Sie mit dem derzeitigen Ablauf zufrieden?
    \item Fehlgeleitet Anfragen, direkte Ansprechpartner:innen
\end{enumerate}

{\sffamily\color{maincolor}{Abschnitt: Visionen und Ziele}}
\begin{enumerate}
    \item[8.] Vorstellen, es gibt eine webbasierte Anwendung (System)- Online-Plattform
        \begin{enumerate}
            \item Übersicht – Vorstellung?
                  Informationen werden benötigt
                  Form der Darstellung
            \item Was wären weitere Funktionen?
            \item Vorausplanen? Reservieren?
            \item Fehlgeleitet Anfragen, direkte Ansprechpartner:innen
        \end{enumerate}
    \item[9.] Sind Ihnen bis hierher noch Gedanken gekommen, die Sie gerne mit auf den Weg geben wollen?
    \item[10.] Vielen Dank, dass Sie uns für dieses Interview zur Verfügung gestanden haben, wir wären jetzt am Ende des Interviews angelangt
    \item[11.] Im Nachhinein noch melden können
\end{enumerate}

\subsection{Ausleihende}
\begin{itemize}
    \item Begrüßung und Danken für die Zeit
    \item Kurzer Umriss des Themas
    \item Vorerfahrung (HiWi,...)
    \item Datenschutz
\end{itemize}

{\sffamily\color{maincolor}{Abschnitt: Jetzt}}
\begin{enumerate}
    \item Ist Dir/Ihnen bekannt, welche Assets sie/du am IMIS ausleihen können?
          \begin{enumerate}
              \item Nein, was hätten sie/du dann gebraucht? Jetzt, wo du es weißt, würdest du es gerne
                    nutzen? oder Listenübersicht
              \item Vorgehen, wenn Sie eine Anfrage stellen zum Ausleihen eines Assets.
                    \begin{itemize}
                        \item Schauen Sie auch spontan in den Laboren, nach den Geräten vorbei?
                        \item Wie häufig ist Ihr spontaner Besuch (nicht) erfolgreich?
                        \item Was machen Sie, wenn Sie keine Person antreffen, der sie mitteilen, dass sie das Gerät mitnehmen?
                        \item Was wäre für Sie der einfachste Weg, Informationen zu hinterlassen, was würden Sie sich wünschen?
                    \end{itemize}
              \item Sehen Sie hier Probleme oder sind Sie mit dem derzeitigen Ablauf zufrieden?
                    \begin{itemize}
                        \item Darstellung von auszuleihenden Inhalten hilfreich (Übersicht)
                        \item Wie gehen sie vor, wenn sie etwas in 2 Wochen ausleihen wollen?
                    \end{itemize}
          \end{enumerate}
\end{enumerate}

{\sffamily\color{maincolor}{Abschnitt: Visionen und Ziele}}
\begin{enumerate}
    \item[8.] Vorstellen, es gibt eine webbasierte Anwendung (System)- Online-Plattform
        \begin{enumerate}
            \item Übersicht – Vorstellung?
                  Informationen werden benötigt
                  Form der Darstellung
            \item Was wären weitere Funktionen?
            \item Vorausplanen? Reservieren?
            \item Fehlgeleitet Anfragen, direkte Ansprechpartner:innen
        \end{enumerate}
    \item[9.] Sind Ihnen bis hierher noch Gedanken gekommen, die Sie gerne mit auf den Weg geben wollen?
    \item[10.] Vielen Dank, dass Sie uns für dieses Interview zur Verfügung gestanden haben, wir wären jetzt am Ende des Interviews angelangt
    \item[11.] Im Nachhinein noch melden können
\end{enumerate}

\include{Verzeichnisse/declaration}
\end{document}
